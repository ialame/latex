\documentclass{article}[11pt]
%\documentclass[12pt,twoside, openany]{extbook}
\usepackage[francais]{babel}
\usepackage[utf8]{inputenc} % Required for including letters with accents
\usepackage[T1]{fontenc} % Use 8-bit encoding that has 256 glyphs
\usepackage{pythontex}
\usepackage{amsthm}
\usepackage{amsmath}
\usepackage{amssymb}
\usepackage{mathrsfs}
\usepackage{graphicx}
\usepackage{geometry}
\usepackage{stmaryrd}
\usepackage{tikz}
\usetikzlibrary{patterns}
%\usetikzlibrary{intersections}
\usetikzlibrary{calc} 
%\usepackage{tkz-tab}
\usepackage{stmaryrd}
%\usepackage{tikz}
%\usetikzlibrary{tikzmark}
\usepackage{empheq}
\usepackage{longtable}
\usepackage{booktabs} 
\usepackage{array}
\usepackage{pstricks}
\usepackage{pst-3dplot}
\usepackage{pst-tree}
\usepackage{pstricks-add}
\usepackage{upgreek}
%\usepackage{epstopdf}
\usepackage{eolgrab}
\usepackage{chngpage}
 \usepackage{calrsfs}
 % Appel du package pythontex 
\usepackage{pythontex}
 \usepackage{enumitem}
\usetikzlibrary{decorations.pathmorphing}
\def \de {{\rm d}}
\def \ch {{\rm ch}}
\def \sh {{\rm sh}}
\def \th {{\rm th}}

\usepackage{color}
%\usepackage{xcolor}
%\usepackage{textcomp}
\newcommand{\mybox}[1]{\fbox{$\displaystyle#1$}}
\newcommand{\myredbox}[1]{\fcolorbox{red}{white}{$\displaystyle#1$}}
\newcommand{\mydoublebox}[1]{\fbox{\fbox{$\displaystyle#1$}}}
\newcommand{\myreddoublebox}[1]{\fcolorbox{red}{white}{\fcolorbox{red}{white}{$\displaystyle#1$}}}
\newtheorem{definition}{Définition}
\newtheorem{theorem}{Théorème}

\definecolor{purple2}{RGB}{153,0,153} % there's actually no standard purple
\definecolor{green2}{RGB}{0,153,0} % a darker green
\usepackage{xcolor}
\usepackage{listings}

\lstdefinestyle{Python}{
    language        = Python,
    basicstyle      = \ttfamily,
    keywordstyle    = \color{blue},
    keywordstyle    = [2] \color{teal}, % just to check that it works
    stringstyle     = \color{violet},
    commentstyle    = \color{red}\ttfamily
}

\usepackage{amsmath} 
\renewcommand{\overrightarrow}[1]{\vbox{\halign{##\cr 
  \tiny\rightarrowfill\cr\noalign{\nointerlineskip\vskip1pt} 
  $#1\mskip2mu$\cr}}}

\newcommand{\Coord}[3]{% 
  \ensuremath{\overrightarrow{#1}\, 
    \begin{pmatrix} 
      #2\\ 
      #3 
    \end{pmatrix}}}

\newcommand{\norme}[1]{\left\lVert\overrightarrow{#1}\right\rVert}
\newcommand{\vecteur}[1]{\overrightarrow{#1}}
\title{TECE Projet 5: Calcul matriciel}
\author{ \textsc{Ibrahim ALAME}}
\date{09/11/2023}
\begin{document}
\maketitle
\begin{enumerate}
%\subsection*{1}
\item Soit ${\cal N}$ l'algèbre des matrices triangulaires supérieurs strictes de $\mathcal{M}_3(\mathbb{R})$.
\[N\in \mathcal{N} \Longleftrightarrow N=\left(\begin{array}{ccc} 0&a&b\\ 0&0&c\\0&0&0
\end{array}\right)\]
On associe à $\mathcal{N}$ l'ensemble  $\mathcal{U}$  des matrices $U=I+N$, où $I$ est la matrice identité de $\mathcal{M}_3(\mathbb{R})$.
\begin{enumerate}
\item  Montrer que le produit de trois matrices quelconques de $\mathcal{N}$ est nul. En particulier $N^3=0$ si $N\in\mathcal{N}$.
\item Montrer que $\mathcal{U}$ est un sous groupe non commutatif du groupe linéaire de $\mathcal{M}_3(\mathbb{R})$.
\item Pour tout réel $\alpha$, on définit la matrice $U^\alpha$ par
\[U^\alpha=I+\alpha N+\frac{\alpha(\alpha-1)}{2}N^2\]
Il sera commode de poser $N_\alpha = \alpha N +\frac{\alpha(\alpha-1)}{2}N^2$. Vérifier que pour $\alpha$ et $\beta$ réels arbitraires, on a
\[U^\alpha U^\beta =U^{\alpha+\beta}\quad\mbox{ et }\quad (U^\alpha)^\beta=U^{\alpha\beta}\]
\item Que peut-on dire de $U^\alpha$ pour $\alpha\in\mathbb{Z}$?
\item On définit une application dite exponentielle, noté exp, de $\mathcal{N}$ dans $\mathcal{U}$:
\[\forall N\in\mathcal{N},\qquad \exp(N)=I+N+\frac{N^2}{2}\]
Montrer que l'application exp est une bijection de $\mathcal{N}$ dans $\mathcal{U}$.
\item On définit également l'application dite logarithme notée ln de $\mathcal{U}$ dans $\mathcal{N}$ par:
\[\mbox{Si } U=I+N,\quad \ln(U)=N-\frac{N^2}{2}\]
Prouver que l'application ln est la bijection réciproque de exp.
\item Établir les formules
\[\exp(\alpha N)=(\exp(N))^\alpha,\quad \ln(U^\alpha)=\alpha\ln U,\quad U^\alpha=\exp(\alpha\ln U)\]
\item Application numérique: Soit
\[U=\left(\begin{array}{ccc} 1&2&3\\ 0&1&2\\0&0&1
\end{array}\right)\]
Calculer $\exp(U-I)$, $\ln U$, $U^{-1}$ et $U^n$ pour tout $n\in\mathbb{Z}$.
\end{enumerate}

%\subsection*{2}
\item Soit $A$ la matrice: $\displaystyle A=\left(\begin{array}{cc} 0&-1\\1&0 \end{array}\right)$
\begin{enumerate}
\item Montrer que pour tout entier $k\in\mathbb{N}$
\[\left\{\begin{array}{l}
A^{2k}=(-1)^kI\\
A^{2k+1}=(-1)^kA
\end{array}\right.\]
\item On définit l'exponentielle matricielle par la somme de la série $\displaystyle e^M=\sum_{p=0}^\infty\frac{M^p}{p!}$. Montrer que
\[e^{tA}=\left(\begin{array}{cc} \cos t&-\sin t\\ \sin t&\cos t
\end{array}\right)\]

\end{enumerate}

%\subsection*{3}
\item Montrer que la matrice suivante est diagonalisable:
$ \displaystyle A=\left(\begin{array}{ccc} 0&a&a^2\\ \frac 1a&0&a\\\frac{1}{a^2}&\frac 1a&0
\end{array}\right)\qquad (a\neq 0)$
En déduire $A^{-1}$ et $A^n$ où $n\in\mathbb{Z}$



%\subsection*{3}
%On définit sur $\mathcal{M}_n(\mathbb{R})$ une application dite exponentielle par
%\[e^{M}=I+M+\frac{M^2}{2!}+\cdots + \frac{M^p}{p!}+\cdots \]
%\begin{enumerate}
%\item Calculer la puissance $p^{\mbox{ème}}$ de la matrice suivante:
%\[A=\left(\begin{array}{ccc} 0&1&1\\ 1&0&1\\1&1&0
%\end{array}\right)\qquad (a\neq 0)\]
%et montrer qu'elle peut s'écrire $A^p = a_p A+b_p I$. calculer $a_p$ et $b_p$. En déduire  $e^{tA}$.
%\item Soit $A$ la matrice:
%\[A=\left(\begin{array}{cc} 0&-1\\1&0
%\end{array}\right)\]
%Calculer $A^p$ pour $p\in\mathbb{N}$.  En déduire la somme de la série
%\[e^{tA}=\left(\begin{array}{cc} \cos t&-\sin t\\ \sin t&\cos t
%\end{array}\right)\]
%\end{enumerate}
%

%\subsection*{4}
\item Soit la matrice de  $\mathcal{M}_n(\mathbb{R})$ 
\[A=
\left(\begin{array}{ccccc}
0&-1&0&\cdots&0\\
-1&0&-1&\ddots&\vdots\\
0&  \ddots &\ddots&\ddots&0\\
\vdots &\ddots &-1&0&-1\\
   0&\cdots &0&-1 &0
\end{array}\right)
\] 
\begin{enumerate}
\item Montrer que $A$ est diagonalisable.
\item Montrer que si $\lambda$ est une valeur propre de $A$ alors $0\leq \lambda\leq 2$. On pourra utiliser le théorème d'Hadamart:
\[\lambda \mbox{ est une valeur propre de } A \Longrightarrow \lambda\in \bigcup_{i=1}^n \left\{z\in \mathbb{C}, |z-a_{i,i}|\leq\sum_{j\neq i}|a_{i,j}|\right\}\]
\item On pose $\lambda=2\cos\theta$ où $\theta\in[0,\pi]$. Soit le determinant  $D_n=\det(\lambda I_n-A)$. Montrer que 
\[\left\{\begin{array}{l}
D_n=2\cos\theta \, D_{n-1}-D_{n-2}, \qquad\forall n\geq 2\\
D_0=1,\quad D_1=2\cos\theta
\end{array}\right.\]

\item Calculer $D_n$ en fonction de $n$. En déduire les valeurs propres de $A$.
\end{enumerate}

\end{enumerate}
\end{document}