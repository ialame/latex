\documentclass{article}[11pt]
%\documentclass[12pt,twoside, openany]{extbook}
\usepackage[francais]{babel}
\usepackage[utf8]{inputenc} % Required for including letters with accents
\usepackage[T1]{fontenc} % Use 8-bit encoding that has 256 glyphs
\usepackage{pythontex}
\usepackage{amsthm}
\usepackage{amsmath}
\usepackage{amssymb}
\usepackage{mathrsfs}
\usepackage{graphicx}
\usepackage{geometry}
\usepackage{stmaryrd}
\usepackage{tikz}
\usetikzlibrary{patterns}
%\usetikzlibrary{intersections}
\usetikzlibrary{calc} 
%\usepackage{tkz-tab}
\usepackage{stmaryrd}
%\usepackage{tikz}
%\usetikzlibrary{tikzmark}
\usepackage{empheq}
\usepackage{longtable}
\usepackage{booktabs} 
\usepackage{array}
\usepackage{pstricks}
\usepackage{pst-3dplot}
\usepackage{pst-tree}
\usepackage{pstricks-add}
\usepackage{upgreek}
%\usepackage{epstopdf}
\usepackage{eolgrab}
\usepackage{chngpage}
 \usepackage{calrsfs}
 % Appel du package pythontex 
\usepackage{pythontex}
 \usepackage{enumitem}
\usetikzlibrary{decorations.pathmorphing}
\def \de {{\rm d}}
\def \ch {{\rm ch}}
\def \sh {{\rm sh}}
\def \th {{\rm th}}

\usepackage{color}
%\usepackage{xcolor}
%\usepackage{textcomp}
\newcommand{\mybox}[1]{\fbox{$\displaystyle#1$}}
\newcommand{\myredbox}[1]{\fcolorbox{red}{white}{$\displaystyle#1$}}
\newcommand{\mydoublebox}[1]{\fbox{\fbox{$\displaystyle#1$}}}
\newcommand{\myreddoublebox}[1]{\fcolorbox{red}{white}{\fcolorbox{red}{white}{$\displaystyle#1$}}}
\newtheorem{definition}{Définition}
\newtheorem{theorem}{Théorème}

\definecolor{purple2}{RGB}{153,0,153} % there's actually no standard purple
\definecolor{green2}{RGB}{0,153,0} % a darker green
\usepackage{xcolor}
\usepackage{listings}

\lstdefinestyle{Python}{
    language        = Python,
    basicstyle      = \ttfamily,
    keywordstyle    = \color{blue},
    keywordstyle    = [2] \color{teal}, % just to check that it works
    stringstyle     = \color{violet},
    commentstyle    = \color{red}\ttfamily
}

\usepackage{amsmath} 
\renewcommand{\overrightarrow}[1]{\vbox{\halign{##\cr 
  \tiny\rightarrowfill\cr\noalign{\nointerlineskip\vskip1pt} 
  $#1\mskip2mu$\cr}}}

\newcommand{\Coord}[3]{% 
  \ensuremath{\overrightarrow{#1}\, 
    \begin{pmatrix} 
      #2\\ 
      #3 
    \end{pmatrix}}}

\newcommand{\norme}[1]{\left\lVert\overrightarrow{#1}\right\rVert}
\newcommand{\vecteur}[1]{\overrightarrow{#1}}
\title{TECE Projet 6: Calcul matriciel}
\author{ \textsc{Ibrahim ALAME}}
\date{09/11/2023}
\begin{document}
\maketitle
\begin{enumerate}
%\subsection*{1}
{\color{blue}
\item Soit ${\cal N}$ l'algèbre des matrices triangulaires supérieurs strictes de $\mathcal{M}_3(\mathbb{R})$.
\[N\in \mathcal{N} \Longleftrightarrow N=\left(\begin{array}{ccc} 0&a&b\\ 0&0&c\\0&0&0
\end{array}\right)\]
On associe à $\mathcal{N}$ l'ensemble  $\mathcal{U}$  des matrices $U=I+N$, où $I$ est la matrice identité de $\mathcal{M}_3(\mathbb{R})$.
}
\begin{enumerate}
{\color{blue}
\item  Montrer que le produit de trois matrices quelconques de $\mathcal{N}$ est nul. En particulier $N^3=0$ si $N\in\mathcal{N}$.
}
\[\left(\begin{array}{ccc} 0&a&b\\ 0&0&c\\0&0&0
\end{array}\right)\left(\begin{array}{ccc} 0&a'&b'\\ 0&0&c'\\0&0&0
\end{array}\right) \left(\begin{array}{ccc} 0&a''&b''\\ 0&0&c''\\0&0&0
\end{array}\right)=\left(\begin{array}{ccc} 0&0&ac'\\ 0&0&0\\0&0&0
\end{array}\right) \left(\begin{array}{ccc} 0&a''&b''\\ 0&0&c''\\0&0&0
\end{array}\right)=\left(\begin{array}{ccc} 0&0&0\\ 0&0&0\\0&0&0
\end{array}\right)      \]
{\color{blue}
\item Montrer que $\mathcal{U}$ est un sous groupe non commutatif du groupe linéaire de $\mathcal{M}_3(\mathbb{R})$.
}
\begin{itemize}
\item $I\in\mathcal{U}$ donc $\mathcal{U}$ est non vide.
\item $U=I+N=\left(\begin{array}{ccc} 1&a&b\\ 0&1&c\\0&0&1
\end{array}\right)$ est inversible car $\det(U)=1\neq 0$. 

Soit $U_1=\left(\begin{array}{ccc} 1&a_1&b_1\\ 0&1&c_1\\0&0&1
\end{array}\right)$ et $U_2=\left(\begin{array}{ccc} 1&a_2&b_2\\ 0&1&c_2\\0&0&1
\end{array}\right)$, donc $U_2^{-1}=\left(\begin{array}{ccc} 1&-a_2&a_2c_2-b_2\\ 0&1&-c_2\\0&0&1
\end{array}\right)$ . 

On a $U_1U_2^{-1}=\left(\begin{array}{ccc} 1&a_1-a_2&b_1-b_2-(a_1-a_2)b_2\\ 0&1&c_1-c_2\\0&0&1
\end{array}\right) \in\mathcal{U}$
\end{itemize}
 De plus 
\[U_1U_2=\left(\begin{array}{ccc} 0&0&a_1c_2\\ 0&0&0\\0&0&0
\end{array}\right) \neq \left(\begin{array}{ccc} 0&0&a_2c_1\\ 0&0&0\\0&0&0
\end{array}\right)=U_2U_1\]
Donc $\mathcal{U}$ est un sous groupe non commutatif de $GL_3(\mathbb{R})$.
{\color{blue}
\item Pour tout réel $\alpha$, on définit la matrice $U^\alpha$ par
\[U^\alpha=I+\alpha N+\frac{\alpha(\alpha-1)}{2}N^2\]
Il sera commode de poser $N_\alpha = \alpha N +\frac{\alpha(\alpha-1)}{2}N^2$. Vérifier que pour $\alpha$ et $\beta$ réels arbitraires, on a
\[U^\alpha U^\beta =U^{\alpha+\beta}\quad\mbox{ et }\quad (U^\alpha)^\beta=U^{\alpha\beta}\]
}
%%%%%%%%%%%%%%%%%%%%%%%%%%%%%%%%%%%%%%%%%%%%%%%%%%%%%%%%%%%%%%%%%%%%%%%%%%%%%%%%%%%
On forme $U^\alpha U^\beta $ en utilisant $N^3=0$,
\[\begin{array}{lcl}
U^\alpha U^\beta  &= &\left[I+\alpha N +\frac{\alpha(\alpha -1)}{2}N^2\right] \left[I+\beta N +\frac{\beta(\beta -1)}{2}N^2\right] \\
 & = & I+(\alpha +\beta)N+\left(\frac{\alpha(\alpha -1) +\beta(\beta -1) }{2}+\alpha\beta\right)N^2\\
  & = &  I+(\alpha +\beta)N+\frac{(\alpha +\beta)(\alpha+\beta -1) }{2}N^2\\
  & = &  U^{\alpha+\beta}
\end{array}\]
De même
\[\begin{array}{lcl}
(U^\alpha)^\beta=(I+ N_\alpha)^\beta  &= &I+ \beta N_\alpha +\frac{\beta(\beta -1)}{2}N_\alpha^2\\
 & = & I+\beta\left[ \alpha N+\frac{\alpha(\alpha -1)}{2}N^2\right] +\frac{\alpha^2\beta(\beta-1)}{2}N^2\\
  & = &  I+\alpha\beta N+\frac{\alpha\beta(\alpha\beta -1) }{2}N^2\\
  & = &  U^{\alpha\beta}
\end{array}\]
%%%%%%%%%%%%%%%%%%%%%%%%%%%%%%%%%%%%%%%%%%%%%%%%%%%%%%%%%%%%%%%%%%%%%%%%%%%%%%%%%%%
{\color{blue}
\item Que peut-on dire de $U^\alpha$ pour $\alpha\in\mathbb{Z}$?
}

%%%%%%%%%%%%%%%%%%%%%%%%%%%%%%%%%%%%%%%%%%%%%%%%%%%%%%%%%%%%%%%%%%%%%%%%%%%%%%%%%%%
Pour $\alpha=n\in\mathbb{N}^*$, on peut appliquer la formule de binôme de Newton à $(I+N)^n$. Puisque $N^3=0$, on vérifie que $U^\alpha$ coïncide avec la puissance $n^{\tiny \mbox{ième}}$ de $U$.

Pour $\alpha=-1$, $U^{-1}=I-N+N^2$ est l'inverse de $I+N$ sur $\mathcal{U}$. Pour tout $\alpha\in\mathbb{Z}$, on obtient donc la puissance d'ordre $\alpha$ de $U$.
%%%%%%%%%%%%%%%%%%%%%%%%%%%%%%%%%%%%%%%%%%%%%%%%%%%%%%%%%%%%%%%%%%%%%%%%%%%%%%%%%%%
{\color{blue}
\item On définit une application dite exponentielle, noté exp, de $\mathcal{N}$ dans $\mathcal{U}$:
\[\forall N\in\mathcal{N},\qquad \exp(N)=I+N+\frac{N^2}{2}\]
Montrer que l'application exp est une bijection de $\mathcal{N}$ dans $\mathcal{U}$.
}

%%%%%%%%%%%%%%%%%%%%%%%%%%%%%%%%%%%%%%%%%%%%%%%%%%%%%%%%%%%%%%%%%%%%%%%%%%%%%%%%%%%

Si $\exp N=\exp N'$, on a $N+\frac{N^2}{2}=N'+\frac{N'^2}{2}$ en élevant au carré, on en déduit $N^2=N'^2$, puis $N=N'$. L'application $\exp$ est donc injective.

Elle est aussi surjective. En se donnant $V=I+P\in\mathcal{U}$, on cherche $N$ tel que $P=N+\frac{N^2}{2}$. On a encore
\[P^2=N^2,\quad\mbox{ d'où }\quad N=P-\frac{P^2}{2}\]
%%%%%%%%%%%%%%%%%%%%%%%%%%%%%%%%%%%%%%%%%%%%%%%%%%%%%%%%%%%%%%%%%%%%%%%%%%%%%%%%%%%
{\color{blue}
\item On définit également l'application dite logarithme notée ln de $\mathcal{U}$ dans $\mathcal{N}$ par:
\[\mbox{Si } U=I+N,\quad \ln(U)=N-\frac{N^2}{2}\]
Prouver que l'application ln est la bijection réciproque de exp.
}

%%%%%%%%%%%%%%%%%%%%%%%%%%%%%%%%%%%%%%%%%%%%%%%%%%%%%%%%%%%%%%%%%%%%%%%%%%%%%%%%%%%
On vient de voir que l'antécédent  de  $V=I+P$ dans l'application $\exp$ est $N=P-\frac{P^2}{2}$; c'est $\ln V$, d'où la relation entre applications
\[\ln=\exp^{-1}\]

%%%%%%%%%%%%%%%%%%%%%%%%%%%%%%%%%%%%%%%%%%%%%%%%%%%%%%%%%%%%%%%%%%%%%%%%%%%%%%%%%%%
{\color{blue}
\item Établir les formules
\[\exp(\alpha N)=(\exp(N))^\alpha,\quad \ln(U^\alpha)=\alpha\ln U,\quad U^\alpha=\exp(\alpha\ln U)\]
}
%%%%%%%%%%%%%%%%%%%%%%%%%%%%%%%%%%%%%%%%%%%%%%%%%%%%%%%%%%%%%%%%%%%%%%%%%%%%%%%%%%%
\[\exp(\alpha N)=I+\alpha N+\frac{\alpha^2}{2}N^2\]
\[(\exp(N))^\alpha=I+\alpha N+\frac{\alpha}{2}N^2+\frac{\alpha(\alpha-1)}{2}N^2=\exp(\alpha N)\]
De même $V^\alpha = I+\alpha P +\frac{\alpha(\alpha-1)}{2}P^2$, d'où
\[\ln V^\alpha =\alpha P +\frac{\alpha(\alpha-1)}{2}P^2 -\frac{\alpha^2}{2}P^2=\alpha\left(P-\frac{P^2}{2}\right)=\alpha\ln V\]
Enfin, $V^\alpha =\exp(\ln V^\alpha)=\exp(\alpha\ln V)$.

%%%%%%%%%%%%%%%%%%%%%%%%%%%%%%%%%%%%%%%%%%%%%%%%%%%%%%%%%%%%%%%%%%%%%%%%%%%%%%%%%%%
{\color{blue}
\item Application numérique: Soit
\[U=\left(\begin{array}{ccc} 1&2&3\\ 0&1&2\\0&0&1
\end{array}\right)\]
Calculer $\exp(U-I)$, $\ln U$, $U^{-1}$ et $U^n$ pour tout $n\in\mathbb{Z}$.
}
%%%%%%%%%%%%%%%%%%%%%%%%%%%%%%%%%%%%%%%%%%%%%%%%%%%%%%%%%%%%%%%%%%%%%%%%%%%%%%%%%%%

On trouve 
\[\exp(U-I)=\left(\begin{array}{ccc} 1&2&5\\ 0&1&2\\0&0&1
\end{array}\right), \quad \ln U = \left(\begin{array}{ccc} 0&2&1\\ 0&0&2\\0&0&0
\end{array}\right), \quad U^{-1} = \left(\begin{array}{ccc} 1&-2&1\\ 0&1&-2\\0&0&1
\end{array}\right) 
\]
\[U^{n} = \left(\begin{array}{ccc} 1&2n&2n^2+n\\ 0&1&2n\\0&0&1
\end{array}\right) 
\]
\end{enumerate}

{\color{blue}
%\subsection*{2}
\item Soit $A$ la matrice: $\displaystyle A=\left(\begin{array}{cc} 0&-1\\1&0 \end{array}\right)$
}
\begin{enumerate}
{\color{blue}
\item Montrer que pour tout entier $k\in\mathbb{N}$
\[\left\{\begin{array}{l}
A^{2k}=(-1)^kI\\
A^{2k+1}=(-1)^kA
\end{array}\right.\]
}
On a $A^2=-I$ donc $A^{2k}=(A^2)^k=(-I)^k=(-1)^kI$ et $A^{2k+1}=A^{2k}A=(-1)^kA$

{\color{blue}
\item On définit l'exponentielle matricielle par la somme de la série $\displaystyle e^M=\sum_{p=0}^\infty\frac{M^p}{p!}$. Montrer que
\[e^{tA}=\left(\begin{array}{cc} \cos t&-\sin t\\ \sin t&\cos t
\end{array}\right)\]
}
\[e^A=\sum_{p=0}^\infty\frac{A^p}{p!} =\sum_{k=0}^\infty\frac{A^{2k}}{(2k)!} +\sum_{k=0}^\infty\frac{A^{2k+1}}{(2k+1)!} =I\sum_{k=0}^\infty\frac{(-1)^{k}t^{2k}}{(2k)!} +A\sum_{k=0}^\infty\frac{(-1)^{k}t^{2k+1}}{(2k+1)!}  \]
\[e^A=I\cos t +A\sin t =\left(\begin{array}{cc} \cos t&-\sin t\\ \sin t&\cos t
\end{array}\right)\]
\end{enumerate}

%\subsection*{3}
{\color{blue}
\item Montrer que la matrice suivante est diagonalisable:
$ \displaystyle A=\left(\begin{array}{ccc} 0&a&a^2\\ \frac 1a&0&a\\\frac{1}{a^2}&\frac 1a&0
\end{array}\right)\qquad (a\neq 0)$
En déduire $A^{-1}$ et $A^n$ où $n\in\mathbb{Z}$
}

L'équation caractéristique s'écrit
\[\left|\begin{array}{ccc}
 -\lambda&a&a^2\\ \frac 1a&-\lambda&a\\\frac{1}{a^2}&\frac 1a&-\lambda
\end{array}\right|=-\lambda^3+3\lambda+2=(\lambda+1)^2(2-\lambda)
\]
La valeur propre double $\lambda=-1$, fournit deux vecteurs propres linéairement indépendants $(-a,1,0)$ et $(-a^2,0,1)$. La valeur propre simple $\lambda=2$ fournit le vecteur propre $(a^2,a,1)$. La matrice $P$ ci dessous diagonalise $A$; on a formé $P^{-1}$:
\[P=\left(\begin{array}{ccc}
 -a&-a^2&a^2\\ 1&0&a\\0&1&1
\end{array}\right),\qquad P^{-1}=\left(\begin{array}{ccc}
 -a&2a^2&-a^3\\ -1&-a&2a^2\\1&a&a^2
\end{array}\right)\]
On sait que
\[D=\left(\begin{array}{ccc} -1&0&0\\ 0&-1&0\\0&0&2
\end{array}\right)=P^{-1}AP\]
d'où
\[D^n=\left(\begin{array}{ccc} (-1)^n&0&0\\ 0&(-1)^n&0\\0&0&2^n
\end{array}\right)=P^{-1}A^nP\]
D'où
\[A^n=P\left(\begin{array}{ccc} (-1)^n&0&0\\ 0&(-1)^n&0\\0&0&2^n
\end{array}\right)P^{-1}\]
\[A^n=-\frac{(-1)^n}{3}\left(\begin{array}{ccc} -2&a&a^2\\ \frac 1a&-2&a\\\frac{1}{a^2}&\frac 1a&-2
\end{array}\right)+\frac{2^n}{3}\left(\begin{array}{ccc} 1&a&a^2\\ \frac 1a&1&a\\\frac{1}{a^2}&\frac 1a&1
\end{array}\right)\]


%\subsection*{3}
%On définit sur $\mathcal{M}_n(\mathbb{R})$ une application dite exponentielle par
%\[e^{M}=I+M+\frac{M^2}{2!}+\cdots + \frac{M^p}{p!}+\cdots \]
%\begin{enumerate}
%\item Calculer la puissance $p^{\mbox{ème}}$ de la matrice suivante:
%\[A=\left(\begin{array}{ccc} 0&1&1\\ 1&0&1\\1&1&0
%\end{array}\right)\qquad (a\neq 0)\]
%et montrer qu'elle peut s'écrire $A^p = a_p A+b_p I$. calculer $a_p$ et $b_p$. En déduire  $e^{tA}$.
%\item Soit $A$ la matrice:
%\[A=\left(\begin{array}{cc} 0&-1\\1&0
%\end{array}\right)\]
%Calculer $A^p$ pour $p\in\mathbb{N}$.  En déduire la somme de la série
%\[e^{tA}=\left(\begin{array}{cc} \cos t&-\sin t\\ \sin t&\cos t
%\end{array}\right)\]
%\end{enumerate}
%

%\subsection*{4}
{\color{blue}
\item Soit la matrice de  $\mathcal{M}_n(\mathbb{R})$ 
\[A=
\left(\begin{array}{ccccc}
0&-1&0&\cdots&0\\
-1&0&-1&\ddots&\vdots\\
0&  \ddots &\ddots&\ddots&0\\
\vdots &\ddots &-1&0&-1\\
   0&\cdots &0&-1 &0
\end{array}\right)
\] 
}
\begin{enumerate}
{\color{blue}
\item Montrer que $A$ est diagonalisable.
}

$A$ réelle symétrique donc diagonalisable.
{\color{blue}
\item Montrer que si $\lambda$ est une valeur propre de $A$ alors $-2\leq \lambda\leq 2$. On pourra utiliser le théorème d'Hadamart:
\[\lambda \mbox{ est une valeur propre de } A \Longrightarrow \lambda\in \bigcup_{i=1}^n \left\{z\in \mathbb{C}, |z-a_{i,i}|\leq\sum_{j\neq i}|a_{i,j}|\right\}\]
}

On a $\lambda\in\left\{z\in\mathbb{C}; |z|\leq 1 \mbox{ ou } |z|\leq 2\right\}\bigcap\mathbb{R}=\left\{z\in\mathbb{R}; |z|\leq 2\right\}$ donc $-2\leq \lambda\leq 2$.
{\color{blue}
\item On pose $\lambda=2\cos\theta$ où $\theta\in[0,\pi]$. Soit le determinant  $D_n=\det(\lambda I_n-A)$. Montrer que 
\[\left\{\begin{array}{l}
D_n=2\cos\theta \, D_{n-1}-D_{n-2}, \qquad\forall n\geq 2\\
D_0=1,\quad D_1=2\cos\theta
\end{array}\right.\]
}

On a 
 \[D_n=\det(2\cos\theta I-A)=
\left|\begin{array}{ccccc}
2\cos\theta&1&0&\cdots&0\\
1&2\cos\theta&-1&\ddots&\vdots\\
0&  \ddots &\ddots&\ddots&0\\
\vdots &\ddots &1&2\cos\theta&1\\
   0&\cdots &0&1 &2\cos\theta
\end{array}\right|
\] 
En développant $D_n$ par rapport à la première colonne on a $D_n=2\cos\theta \, D_{n-1}-D_{n-2}$.
$D_0=1$ est une conversion pour tout determinant d'ordre 0. A l'ordre 1 le determinant $D_1$ se réduit à son unique coefficient $2\cos\theta$.
{\color{blue}
\item Calculer $D_n$ en fonction de $n$. En déduire les valeurs propres de $A$.
}

L'équation caractéristique s'écrit: $r^2-2\cos\theta r+1=0$ dont les racines sont $e^{i \theta}$ et $e^{-i \theta}$ et donc 
$D_n=a\cos n \theta+b\sin n \theta $ où $a$ et $b$ solution de
\[\left\{\begin{array}{l}
1=a\cos 0 +b\sin 0\\
2\cos\theta = a\cos\theta+b\sin\theta
\end{array}\right.\]
on a alors $a=1$ et $b=\frac{\cos\theta}{\sin\theta}$. Donc $D_n=\cos n \theta+\frac{\cos\theta}{\sin\theta}\sin n \theta =\frac{\sin\theta\cos n \theta +\cos\theta \sin n\theta}{\sin\theta}$

Donc
\[D_n=\frac{\sin(n+1)\theta}{\sin\theta}\]
On a $D_n=0\Longrightarrow (n+1)\theta=k\pi$ où $k=1,2,...,n$. D'où les valeurs propres
\[\lambda=2\cos\frac{k\pi}{n+1}\qquad k=1,2,...,n\]
\end{enumerate}

\end{enumerate}
\end{document}