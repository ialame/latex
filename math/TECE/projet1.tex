\documentclass{article}
\usepackage[francais]{babel}
\usepackage[utf8]{inputenc} % Required for including letters with accents
\usepackage[T1]{fontenc} % Use 8-bit encoding that has 256 glyphs
\usepackage{pythontex}
\usepackage{amsthm}
\usepackage{amsmath}
\usepackage{amssymb}
\usepackage{mathrsfs}
\usepackage{graphicx}
\usepackage{geometry}
\usepackage{stmaryrd}
\usepackage{tikz}
\usetikzlibrary{patterns}
%\usetikzlibrary{intersections}
\usetikzlibrary{calc} 
%\usepackage{tkz-tab}
\usepackage{stmaryrd}
%\usepackage{tikz}
%\usetikzlibrary{tikzmark}
\usepackage{empheq}
\usepackage{longtable}
\usepackage{booktabs} 
\usepackage{array}
\usepackage{pstricks}
\usepackage{pst-3dplot}
\usepackage{pst-tree}
\usepackage{pstricks-add}
\usepackage{upgreek}
%\usepackage{epstopdf}
\usepackage{eolgrab}
\usepackage{chngpage}
 \usepackage{calrsfs}
 % Appel du package pythontex 
\usepackage{pythontex}

\usetikzlibrary{decorations.pathmorphing}
\def \de {{\rm d}}
\def \ch {{\rm ch}}
\def \sh {{\rm sh}}
\def \th {{\rm th}}
\def \sinc {{\rm sinc}}
\def \cotan {{\rm cotan}}

\usepackage{color}
%\usepackage{xcolor}
%\usepackage{textcomp}
\newcommand\hr{\par\vspace{-.5\ht\strutbox}\noindent\hrulefill\par}
\newcommand{\mybox}[1]{\fbox{$\displaystyle#1$}}
\newcommand{\myredbox}[1]{\fcolorbox{red}{white}{$\displaystyle#1$}}
\newcommand{\mydoublebox}[1]{\fbox{\fbox{$\displaystyle#1$}}}
\newcommand{\myreddoublebox}[1]{\fcolorbox{red}{white}{\fcolorbox{red}{white}{$\displaystyle#1$}}}

\definecolor{purple2}{RGB}{153,0,153} % there's actually no standard purple
\definecolor{green2}{RGB}{0,153,0} % a darker green
\usepackage{xcolor}
%\setbeamercolor{background canvas}{bg=lightgray}
\usepackage{listings}
\definecolor{purple2}{RGB}{153,0,153} % there’s actually no standard purple
\definecolor{green2}{RGB}{0,153,0} % a
\lstset{%
language=Python, % 
basicstyle=\normalsize\ttfamily, % 
% Color settings to match IDLE style 
keywordstyle=\color{orange}, % 
keywordstyle={[2]\color{purple2}}, % 
stringstyle=\color{green2}, 
commentstyle=\color{red}, 
upquote=true, %
}
\lstdefinestyle{Python}{
    language        = Python,
    basicstyle      = \ttfamily,
    keywordstyle    = \color{blue},
    keywordstyle    = [2] \color{teal}, % just to check that it works
    stringstyle     = \color{violet},
    commentstyle    = \color{red}\ttfamily
}

\usepackage{geometry}
 \geometry{
 a4paper,
 total={210mm,297mm},
 left=25mm,
 right=25mm,
 top=20mm,
 bottom=20mm,
 }
\usepackage{wrapfig}
\title{Concours d'entrée à l'ESTP}
%\author{Ibrahim ALAME}
\date{}
  \begin{document}
  \lstset{
    frame       = single,
    numbers     = left,
    showspaces  = false,
    showstringspaces    = false,
    captionpos  = t,
    caption     = \lstname
}
%\maketitle
\begin{center}
\section*{TECE Module 1: Suites et séries numériques}
%\section*{TECE Module 1: Étude d'une suite vectorielle en dim n}
\end{center}
\subsection*{Problème 1}
On désigne par $\mathbb{R}^+$ l'ensemble des nombre réels $x$ tels que $x\geq 0$. Soit $a\in \mathbb{R}^+$ et soit $(u_n)$ la suite de nombres réels définie par la relation de récurrence:
\[\forall n\in \mathbb{N},\quad u_{n+1}=a+\frac{1-e^{-n}}{2}u_n\]
et la condition initiale $u_0=a$.
\begin{enumerate}
\item Montrer que pour tout entier naturel $n$, $u_n\leq 2a$.
\item Montrer que pour tout $k\in \mathbb{N}$, on a
\[2^{k+1}(2a-u_{k+1})=2^k(2a-u_k)+\left(\frac{2}{e}\right)^ku_k\]
\item En déduire que pour tout $n\geq 1$
\[2^n(2a-u_n)=a+\sum_{k=0}^{n-1}\left(\frac{2}{e}\right)^ku_k\]
\item Montrer que la série de terme général $\left(\left(\frac{2}{e}\right)^ku_k\right)_k$ converge. En déduire que la suite $(u_n)$ converge vers $2a$ et que \[2a-u_n\sim \frac{K}{2^n}\] où $K$ est une constante réelle que l'on déterminera.
\end{enumerate}

\subsection*{Problème 2}
\subsubsection*{Partie 1}
Soit $n$ un nombre entier naturel non nul. On considère les fonction numérique $f_n$, $g_n$ et $h_n$ définies sur l'intervalle $[0,n]$ par les relations:
\[f_n(t)=(1-\frac tn)^n,\quad g_n(t)=e^{-t}-(1-\frac tn)^n,\quad h_n(t)=e^tg'_n(t)\] 
\begin{enumerate}
\item Étudier les variations de $h_n$. En déduire les variations de $g_n$. Montrer qu'il existe un élément $x_n$ de $[0,n]$ et un seul tel que, pour tout élément $t$ de $[0,n]$, $g_n(t)\leq g_n(x_n)$.
\item Montrer que $g_n(x_n)\leq \frac{1}{ne}$.
\end{enumerate}
  \end{document}
  
 




