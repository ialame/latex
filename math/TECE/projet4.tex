\documentclass{article}[11pt]
%\documentclass[12pt,twoside, openany]{extbook}
\usepackage[francais]{babel}
\usepackage[utf8]{inputenc} % Required for including letters with accents
\usepackage[T1]{fontenc} % Use 8-bit encoding that has 256 glyphs
\usepackage{pythontex}
\usepackage{amsthm}
\usepackage{amsmath}
\usepackage{amssymb}
\usepackage{mathrsfs}
\usepackage{graphicx}
\usepackage{geometry}
\usepackage{stmaryrd}
\usepackage{tikz}
\usetikzlibrary{patterns}
%\usetikzlibrary{intersections}
\usetikzlibrary{calc} 
%\usepackage{tkz-tab}
\usepackage{stmaryrd}
%\usepackage{tikz}
%\usetikzlibrary{tikzmark}
\usepackage{empheq}
\usepackage{longtable}
\usepackage{booktabs} 
\usepackage{array}
\usepackage{pstricks}
\usepackage{pst-3dplot}
\usepackage{pst-tree}
\usepackage{pstricks-add}
\usepackage{upgreek}
%\usepackage{epstopdf}
\usepackage{eolgrab}
\usepackage{chngpage}
 \usepackage{calrsfs}
 % Appel du package pythontex 
\usepackage{pythontex}
 \usepackage{enumitem}
 \renewcommand{\labelitemi}{$\circ$}
\usetikzlibrary{decorations.pathmorphing}
\def \de {{\rm d}}
\def \ch {{\rm ch}}
\def \sh {{\rm sh}}
\def \th {{\rm th}}

\usepackage{color}
%\usepackage{xcolor}
%\usepackage{textcomp}
\newcommand{\mybox}[1]{\fbox{$\displaystyle#1$}}
\newcommand{\myredbox}[1]{\fcolorbox{red}{white}{$\displaystyle#1$}}
\newcommand{\mydoublebox}[1]{\fbox{\fbox{$\displaystyle#1$}}}
\newcommand{\myreddoublebox}[1]{\fcolorbox{red}{white}{\fcolorbox{red}{white}{$\displaystyle#1$}}}
\newtheorem{definition}{Définition}
\newtheorem{theorem}{Théorème}

\definecolor{purple2}{RGB}{153,0,153} % there's actually no standard purple
\definecolor{green2}{RGB}{0,153,0} % a darker green
\usepackage{xcolor}
\usepackage{listings}

\lstdefinestyle{Python}{
    language        = Python,
    basicstyle      = \ttfamily,
    keywordstyle    = \color{blue},
    keywordstyle    = [2] \color{teal}, % just to check that it works
    stringstyle     = \color{violet},
    commentstyle    = \color{red}\ttfamily
}

\usepackage{amsmath} 
\renewcommand{\overrightarrow}[1]{\vbox{\halign{##\cr 
  \tiny\rightarrowfill\cr\noalign{\nointerlineskip\vskip1pt} 
  $#1\mskip2mu$\cr}}}

\newcommand{\Coord}[3]{% 
  \ensuremath{\overrightarrow{#1}\, 
    \begin{pmatrix} 
      #2\\ 
      #3 
    \end{pmatrix}}}

\newcommand{\norme}[1]{\left\lVert\overrightarrow{#1}\right\rVert}
\newcommand{\vecteur}[1]{\overrightarrow{#1}}
\title{TECE Projet 4: Limites et les développements limités}
\author{Ibrahim ALAME}
\date{19/10/2023}
  \begin{document}
\maketitle
\begin{center}
\end{center}
\subsection*{Exercice 1}
{\color{blue}
A l'aide des accroissements finis, majorer l'erreur faites en prenant 100 comme valeur approchée de $\sqrt{10001}$.}

L'inégalité des accroissements finis appliquée à la fonction $\sqrt x$ entre 10000 et 10001 donne
\[|\sqrt{10001}-\sqrt{10000}|\leq \max_{10000\leq x\leq 10001}\frac{1}{2\sqrt x} \times|10001-10000|=\frac{1}{200}\]
Donc on fait une erreur inférieur à $\frac{1}{200}$ en assimilant $\sqrt{10001}$ à 100.
\subsection*{Exercice 2}
\begin{enumerate}
{\color{blue}
\item Montrer que $\frac{\pi}{4}=4\arctan\frac 15-\arctan\frac{1}{239}$
}

On utilise la formule
\[\arctan x +\arctan y =\arctan\frac{x+y}{1-xy}.\quad \mbox{ Si } x.y < 1\]
\begin{itemize}
\item[$\bullet$] $\displaystyle \arctan \frac 15 +\arctan \frac 15=\arctan\frac{\frac 25}{1-\frac{1}{25}}=\arctan\frac{5}{12}\quad \mbox{ car } \frac 15.\frac 15 < 1$
\item[$\bullet$] $\displaystyle 4\arctan\frac 15=2\arctan\frac{5}{12}=\arctan\frac{\frac{10}{12}}{1-\frac{25}{144}}=\arctan\frac{120}{119}$. D'où
\end{itemize}
\[4\arctan\frac 15-\arctan\frac{1}{239}=\arctan\frac{120}{119}-\arctan\frac{1}{239}=\arctan\frac{\frac{120}{119}-\frac{1}{239}}{1+\frac{120}{119\times 239}}=\arctan 1=\frac{\pi}{4}\]
{\color{blue}
\item Calculer la somme géométrique: $1-x^2+x^4+\cdots +(-1)^nx^{2n}$ où $x$  est un réel tel que $|x|<1$. En déduire que
\[\forall x\in ]-1,1[,\qquad \frac{1}{1+x^2}=1-x^2+x^4+\cdots +(-1)^nx^{2n}+(-1)^{n+1}\frac{x^{2n+2}}{1+x^2}\]
}
Écrivons la somme de la suite géométrique de premier terme 1 et de raison $-x^2$:
\[1-x^2+x^4+\cdots +(-1)^nx^{2n}=\frac{1-(-x^2)^{n+1}}{1-(-x^2)}=\frac{1-(-1)^{n+1}x^{2n+2}}{1+x^2} \]
\[1-x^2+x^4+\cdots +(-1)^nx^{2n}=\frac{1}{1+x^2}-(-1)^{n+1}\frac{x^{2n+2}}{1+x^2} \]
D'où la formule.
{\color{blue}
\item Montrer alors que
\[\forall x\in ]-1,1[,\qquad \arctan x=x-\frac{x^3}3+\frac{x^5}5+\cdots +(-1)^n\frac{x^{2n+1}}{2n+1}+(-1)^{n+1}\int_0^x\frac{t^{2n+2}}{1+t^2}\de t\]
}
Il suffit d'intégrer terme à terme l'égalité de la question précédente.
{\color{blue}
\item On pose $R_n(x)=\int_0^x\frac{t^{2n+2}}{1+t^2}\de t$. Montrer que $R_n(x)\leq\frac{x^{2n+3}}{2n+3}$ pour $x\geq 0$.}
\[|R_n(x)|=\int_0^x\frac{t^{2n+2}}{1+t^2}\de t\leq \int_0^xt^{2n+2}\de t\leq \frac{x^{2n+3}}{2n+3}\]

{\color{blue}
\item Calculer $\arctan\frac 15$ et $\arctan\frac{1}{239}$ à $10^{-9}$ près. En déduire une valeur approchée de $\pi$ à $10^{-8}$ près. 
}
Pour calculer $\arctan\frac 15$ à $10^{-9}$ près, il faut que $\frac{\left(\frac 15\right)^{2n+3}}{2n+3}<10^{-9}$ soit $n\geq 5$. Pour $\arctan\frac{1}{239}$ il faut que $n\geq 1$. Par conséquent, on a:
\begin{itemize}
\item[$\bullet$] $\displaystyle \arctan\frac 15=\frac 15 -\frac{(0.2)^3}{3}+\frac{(0.2)^5}{5}-\frac{(0.2)^7}{7}+\frac{(0.2)^9}{9}-\frac{(0.2)^{11}}{11}=0.197395560$
\item[$\bullet$] $\displaystyle \arctan\frac 1{239}=\frac 1{239}-\frac{\left(\frac 1{239}\right)^3}{3}=0.004184076$
\end{itemize}
On en déduit alors que $\pi=3.14159264$ à $10^{-8}$ près.
\end{enumerate}
\subsection*{Exercice 3}
\begin{enumerate}
{\color{blue}
\item Déterminer la limite, quand $x\to 0$ de :
\[f(x)=\frac{1}{x^2}\left[\frac{\ln(1+x)+\ln(1-x)}{x^2}+1\right]\]}

On divise par $x^4$ donc on fait un DL à l'ordre 4:
\[\ln(1+x)=x-\frac{x^2}{2}+\frac{x^3}{3}-\frac{x^4}{4}+o(x^4),\qquad \ln(1-x)=-x-\frac{x^2}{2}-\frac{x^3}{3}-\frac{x^4}{4}+o(x^4)\]
\[f(x)=\frac{1}{x^2}\left[\frac{1}{x^2}\left(-x^2-\frac{x^4}{2}+o(x^4)\right)+1\right]=\frac{1}{x^2}\left[-1 -\frac{x^2}{2}+o(x^2)+1\right]=-\frac 12+o(1)\]
On a donc $f(x)\to -\frac 12$ quand $x\to 0$.
{\color{blue}
\item Donner le développement limité à l'ordre 2 en 0 de:
\[g(x)=\exp\left[\frac{\sqrt{1+2x^2}-1}{x^2}\right]\]}

On divise par $x^2$ donc on effectue un DL à l'ordre 4. On pose $u=2x^2\to 0$ quand $x\to 0$:
\[\sqrt{1+u}=1+\frac 12x-\frac 18u^2 +o(u^2), \qquad \sqrt{1+2x^2}=1+x^2-\frac {x^4}2 + o(x^4),\] 
\[\frac{\sqrt{1+2x^2}-1}{x^2}=1-\frac{x^2}{2}+o(x^2),\]
\[g(x)=\exp\left(1-\frac{x^2}{2}+o(x^2)\right)=e\times \exp\left(-\frac{x^2}{2}+o(x^2)\right)=e\times \left(1-\frac{x^2}{2}+o(x^2)\right)\]
\end{enumerate}
\subsection*{Problème}
Soit $I$ un intervalle ouvert de $\mathbb{R}$ et $f$ une fonction définie sur $I$ et à valeurs dans $\mathbb{R}$.
\begin{enumerate}
\item Le but de cette première question est de montrer la formule de Taylor avec reste intégral qui assure que toute fonction qui admet une dérivée d'ordre $n+1$ continue satisfait
\[f(a+h)=f(a)+hf'(a)+\frac{h^2}{2}f''(a)+\cdots+\frac{h^n}{n!}f^{(n)}(a)+h^{n+1}\int_0^1\frac{(1-\theta)^n}{n!}f^{(n+1)}(a+\theta h)\de \theta\]
avec $a$ et $a+h$ des éléments de $I$.
\begin{enumerate}
\item  Montrer que la formule est vraie pour $n=0$.
\item  Soit $f$ une fonction qui admet une dérivée d'ordre $n+2$ continue. Montrer que
\[\int_0^1\frac{f^{(n+1)}(a+\theta h)}{n!}(1-\theta)^n\de \theta=\frac{f^{(n+1)}(a)}{(n+1)!}+h\int_0^1\frac{f^{(n+2)}(a+\theta h)}{(n+1)!}(1-\theta)^{n+1}\de \theta\]
\item  En utilisant les questions précédentes, montrer par récurrence la formule de Taylor avec reste intégral.
\item On suppose qu'il existe $M_{n+1}$ tel que $\forall x\in I,  |f^{(n+1)}(x)|\leq M_{n+1}$. Alors on a (la formule de Taylor Lagrange): 
$$\displaystyle{\forall x\in I,\forall a\in I} \quad\displaystyle{\left\vert f(a+h)-f(a)-\sum_{k=1}^{n}\frac{h^k}{k!}f^{(k)}(a)\right\vert\leq \frac{\vert h\vert^{n+1}}{(n+1)!}}M_{n+1}$$
\end{enumerate}

\item  On écrit la formule de Taylor avec reste intégral pour $a=0$ et $h=x$; on l'appelle alors formule de Mac-Laurin:
\[f(x)=f(0)+xf'(0)+\frac{x^2}{2}f''(0)+\cdots+\frac{x^n}{n!}f^{(n)}(0)+h^{n+1}\int_0^1\frac{(1-\theta)^n}{n!}f^{(n+1)}(\theta x)\de \theta\]
\begin{enumerate}
\item Appliquer la formule de Mac-Laurin à la fonction $\sin$ et montrer que pour les valeurs de $x$ entre $0$ et $\frac{\pi}{4}$ et pour tout entier $n$, il existe un polynôme $P_n(x)$ que l'on déterminera, tel que:
\[|\sin x - P_n(x)|\leq \frac{1}{(2n+3)!}\]
\item Comment faut-il choisir $n$ pour que cette dernière approximation donne $\sin x$ à $10^{-3}$ près pour $x$ entre 0 et $\frac{\pi}{4}$?
\item Donner une valeur approchée à 3 décimales pour $\sin \frac 14$.
\end{enumerate}
\end{enumerate}
\end{document}