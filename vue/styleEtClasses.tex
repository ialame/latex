\section{Introduction et portée des styles}
\subsection{Comportement par défaut}
Par défaut, tout le CSS est global car il est réuni dans le {\color{monOrange}bundle} par {\color{monOrange}Vite}. Si vous définissez des classes à l'intérieur des balises {\tt <style>}, ces classes s'appliqueront donc à tous vos composants.

\subsection{L'attribut {\color{monOrange}scoped}}
L'attribut {\color{monOrange}scoped} permet d'encapsuler le CSS d'un composant monofichier ({\color{monOrange}SFC}) en ajoutant, grâce à PostCSS un attribut unique (par exemple data-v-29e5g83) à l'élément en compilant par exempl {\color{monOrange}.ma-classe} en {\color{monOrange}.ma-classe[data-v-29e5g83]}.

Cette encapsulation est effectuée automatiquement par {\color{monOrange}Vite}. Vous n'avez rien à faire d'autre que d'ajouter l'attribut {\color{monOrange}scoped}. Pour utiliser {\color{monOrange}scoped}, il suffit de l'ajouter sur la balise style de votre composant monofichier :
\begin{minted}[
mathescape,
framesep=2mm,
baselinestretch=1.2,
%fontsize=\footnotesize,
bgcolor=LightGray,
%linenos
]{html}
<style scoped>
  .ma-classe {
  }
</style>
\end{minted}

%Exemple exécutable de la vidéo
%Vous pouvez directement utiliser ce code exécutable. N'hésitez pas à l'ouvrir dans un nouvel onglet pour le modifier ou mieux voir :

\section{La liaison de classes}
\subsection{La liaison de classes}
Nous avons vu qu'il était possible de lier une ou plusieurs classes à l'attribut {\color{monOrange}class} en utilisant la directive {\color{monOrange}v-bind}. Pour permettre une utilisation très flexible, {\color{monOrange}Vue.js} permet plus de syntaxes lorsque nous utilisons {\color{monOrange}v-bind} avec les classes ou les styles.

Bien sûr la liaison n'a de sens que si les classes sont réactives suivant la valeur de propriétés. Sinon il suffit d'écrire les classes en dur côté {\color{monOrange}template}. Nous allons récapituler toutes les syntaxes possibles.

\subsubsection{Liaison avec des objets}
La première syntaxe permet d'utiliser un objet :
\begin{minted}[
mathescape,
framesep=2mm,
baselinestretch=1.2,
%fontsize=\footnotesize,
bgcolor=LightGray,
%linenos
]{html}
<div :class="{ actif: estActif }"></div>
\end{minted}
La classe {\color{monOrange}actif} sera ajoutée si la propriété {\tt estActif} vaut {\tt true} :
\begin{minted}[
mathescape,
framesep=2mm,
baselinestretch=1.2,
%fontsize=\footnotesize,
bgcolor=LightGray,
%linenos
]{javascript}
const estActif = ref(true)
\end{minted}
Si la propriété réactive passe à {\tt false}, la classe sera automatiquement retirée.

\subsubsection{Combinaison de l'attribut {\color{monOrange}class} et d'une liaison {\color{monOrange}:class}}
Vous pouvez combiner sans problème l'attribut {\color{monOrange}class} avec une liaison avec la directive {\color{monOrange}v-bind} en utilisant {\color{monOrange}:class} :
\begin{minted}[
mathescape,
framesep=2mm,
baselinestretch=1.2,
%fontsize=\footnotesize,
bgcolor=LightGray,
%linenos
]{html}
<div
  class="classe1"
  :class="{ active: isActive, error: hasError }"
></div>
\end{minted}
Ici nous aurons quoiqu'il arrive la {\tt classe1}.
\begin{itemize}
\item La classe {\color{monOrange}active} sera ajoutée si la propriété {\color{monOrange}isActive} vaut {\tt true}.
\item La classe {\color{monOrange}error} sera ajoutée si la propriété {\color{monOrange}hasError} vaut {\tt true}.
\end{itemize}
Dans tous les cas, Vue.js créera la div sur le DOM avec un seul attribut classe, par exemple :
\begin{minted}[
mathescape,
framesep=2mm,
baselinestretch=1.2,
%fontsize=\footnotesize,
bgcolor=LightGray,
%linenos
]{html}
<div class="classe1 isActive hasError"></div>
\end{minted}

\subsubsection{Utilisation d'un objet réactif}
Vous pouvez utiliser un objet réactif côté {\color{monOrange}script} pour la liaison avec l'attribut {\color{monOrange}class} :
\begin{minted}[
mathescape,
framesep=2mm,
baselinestretch=1.2,
%fontsize=\footnotesize,
bgcolor=LightGray,
%linenos
]{javascript}
const mesClasses = reactive({
  active: true,
  error: false,
  classeX: false
});
\end{minted}
Il suffit ensuite de le passer à {\color{monOrange}v-bind} :
\begin{minted}[
mathescape,
framesep=2mm,
baselinestretch=1.2,
%fontsize=\footnotesize,
bgcolor=LightGray,
%linenos
]{html}
<div :class="mesClasses"></div>
\end{minted}

\subsubsection{Utilisation d'une propriété calculée}
Vous pouvez même utiliser une propriété calculée :
\begin{minted}[
mathescape,
framesep=2mm,
baselinestretch=1.2,
%fontsize=\footnotesize,
bgcolor=LightGray,
%linenos
]{javascript}
const mesClasses = computed(() => ({
  active: isActive.value && !error.value,
  error: error.value && error.value.type === 'fatal'
}))
\end{minted}


\subsubsection{Utilisation d'un tableau de classes}
Vous pouvez utiliser un tableau de classes :
\begin{minted}[
mathescape,
framesep=2mm,
baselinestretch=1.2,
%fontsize=\footnotesize,
bgcolor=LightGray,
%linenos
]{html}
<div :class="[classe1, classe2]"></div>
\end{minted}

%Exemple exécutable de la vidéo
%Vous pouvez directement utiliser ce code exécutable. N'hésitez pas à l'ouvrir dans un nouvel onglet pour le modifier ou mieux voir :

\section{La liaison de styles}
\subsection{Liaison de styles}
Passons maintenant aux différentes syntaxes possibles pour les liaisons de style.

\subsubsection{Liaison avec des objets}
La première syntaxe permet d'utiliser un objet :
\begin{minted}[
mathescape,
framesep=2mm,
baselinestretch=1.2,
%fontsize=\footnotesize,
bgcolor=LightGray,
%linenos
]{html}
<div :style="{ color: activeColor, fontSize: fontSize + 'px' }"></div>
\end{minted}

Avec côté {\color{monOrange}script} :
\begin{minted}[
mathescape,
framesep=2mm,
baselinestretch=1.2,
%fontsize=\footnotesize,
bgcolor=LightGray,
%linenos
]{javascript}
const activeColor = ref('red')
const fontSize = ref(30)
\end{minted}

\subsubsection{Utilisation des propriétés CSS natives}
Vous pouvez également utiliser directement les noms des propriétés CSS en utilisant des guillemets simples :
\begin{minted}[
mathescape,
framesep=2mm,
baselinestretch=1.2,
%fontsize=\footnotesize,
bgcolor=LightGray,
%linenos
]{html}
<div :style="{ 'font-size': fontSize + 'px' }"></div>
\end{minted}

\subsubsection{Utilisation d'un objet réactif}
Vous pouvez utiliser un objet réactif côté script pour la liaison avec l'attribut {\color{monOrange}style} :
\begin{minted}[
mathescape,
framesep=2mm,
baselinestretch=1.2,
%fontsize=\footnotesize,
bgcolor=LightGray,
%linenos
]{javascript}
const mesStyles = reactive({
  color: 'red',
  fontSize: '13px'
})
\end{minted}
Et côté {\color{monOrange}template} :
\begin{minted}[
mathescape,
framesep=2mm,
baselinestretch=1.2,
%fontsize=\footnotesize,
bgcolor=LightGray,
%linenos
]{html}
<div :style="mesStyles"></div>
\end{minted}

\subsubsection{Utilisation d'un tableau}
Vous pouvez enfin utiliser un tableau :
\begin{minted}[
mathescape,
framesep=2mm,
baselinestretch=1.2,
%fontsize=\footnotesize,
bgcolor=LightGray,
%linenos
]{html}
<div :style="[stylesDeBase, stylesParticuliers]"></div>
\end{minted}
Ici les styles contenus dans l'objet {\color{monOrange}stylesDeBase} seront appliqués. Ensuite les styles contenus dans l'objet {\color{monOrange}stylesParticuliers} seront fusionnés. Si certaines propriétés sont dans les deux objets, les valeurs du dernier objet écraseront les valeurs précédentes.

Autrement dit, si par exemple il y a {\tt color:'red'} dans {\color{monOrange}stylesDeBase} et {\tt color:'blue'} dans {\color{monOrange}stylesParticuliers}. La valeur de la propriété {\tt color} sera {\tt blue}.

%Exemple exécutable de la vidéo
%Vous pouvez directement utiliser ce code exécutable. N'hésitez pas à l'ouvrir dans un nouvel onglet pour le modifier ou mieux voir :

\section{Utilisation de {\color{monOrange}Sass}}
L'utilisation de {\color{monOrange}Sass} avec {\color{monOrange}Vite} et {\color{monOrange}Vue.js} est extrêmement simple. Il suffit d'installer {\color{monOrange}sass} :
\begin{minted}[
mathescape,
framesep=2mm,
baselinestretch=1.2,
%fontsize=\footnotesize,
bgcolor=LightGray,
%linenos
]{bash}
npm add -D sass
\end{minted}
Et dans les composants {\color{monOrange}SFC} de préciser, comme pour {\color{monOrange}TypeScript} côté {\color{monOrange}template}, que les styles sont écrits en {\color{monOrange}Sass} :
\begin{minted}[
mathescape,
framesep=2mm,
baselinestretch=1.2,
%fontsize=\footnotesize,
bgcolor=LightGray,
%linenos
]{html}
<template>
  <h1><span>Bonjour</span> le monde !</h1>
</template>

<script setup lang="ts"></script>

<style scoped lang="scss">
h1 {
  color: red;
  span {
    color: blue;
  }
}
</style>
\end{minted}

%Exemple exécutable de la vidéo
%Vous pouvez directement utiliser ce code exécutable. N'hésitez pas à l'ouvrir dans un nouvel onglet pour le modifier ou mieux voir :

