\documentclass{article}
%\documentclass[12pt,twoside, openany]{extbook}
\usepackage[francais]{babel}
\usepackage[utf8]{inputenc} % Required for including letters with accents
\usepackage[T1]{fontenc} % Use 8-bit encoding that has 256 glyphs
\usepackage{pythontex}
\usepackage{amsthm}
\usepackage{amsmath}
\usepackage{amssymb}
\usepackage{mathrsfs}
\usepackage{graphicx}
\usepackage{geometry}
\usepackage{stmaryrd}
\usepackage{tikz}
\usetikzlibrary{patterns}
%\usetikzlibrary{intersections}
\usetikzlibrary{calc} 
%\usepackage{tkz-tab}
\usepackage{stmaryrd}
%\usepackage{tikz}
%\usetikzlibrary{tikzmark}
\usepackage{empheq}
\usepackage{longtable}
\usepackage{booktabs} 
\usepackage{array}
\usepackage{pstricks}
\usepackage{pst-3dplot}
\usepackage{pst-tree}
\usepackage{pstricks-add}
\usepackage{upgreek}
%\usepackage{epstopdf}
\usepackage{eolgrab}
\usepackage{chngpage}
 \usepackage{calrsfs}
 % Appel du package pythontex 
\usepackage{pythontex}
 \usepackage{enumitem}
\usetikzlibrary{decorations.pathmorphing}
\def \de {{\rm d}}
\def \ch {{\rm ch}}
\def \sh {{\rm sh}}
\def \th {{\rm th}}


\usepackage[cache=false]{minted}
 \definecolor{darkWhite}{rgb}{0.94,0.94,0.94}
 \usepackage[cache=false]{minted}
\definecolor{LightGray}{gray}{0.9}
\definecolor{monOrange}{rgb}{0.97,0.35,0.04}

\usepackage{color}
%\usepackage{xcolor}
%\usepackage{textcomp}
\newcommand{\mybox}[1]{\fbox{$\displaystyle#1$}}
\newcommand{\myredbox}[1]{\fcolorbox{red}{white}{$\displaystyle#1$}}
\newcommand{\mydoublebox}[1]{\fbox{\fbox{$\displaystyle#1$}}}
\newcommand{\myreddoublebox}[1]{\fcolorbox{red}{white}{\fcolorbox{red}{white}{$\displaystyle#1$}}}
\newtheorem{definition}{Définition}
\newtheorem{theorem}{Théorème}

\definecolor{purple2}{RGB}{153,0,153} % there's actually no standard purple
\definecolor{green2}{RGB}{0,153,0} % a darker green
\usepackage{xcolor}
\usepackage{listings}

\lstdefinestyle{Python}{
    language        = Python,
    basicstyle      = \ttfamily,
    keywordstyle    = \color{blue},
    keywordstyle    = [2] \color{teal}, % just to check that it works
    stringstyle     = \color{violet},
    commentstyle    = \color{red}\ttfamily
}

\usepackage{amsmath} 
\renewcommand{\overrightarrow}[1]{\vbox{\halign{##\cr 
  \tiny\rightarrowfill\cr\noalign{\nointerlineskip\vskip1pt} 
  $#1\mskip2mu$\cr}}}

\newcommand{\Coord}[3]{% 
  \ensuremath{\overrightarrow{#1}\, 
    \begin{pmatrix} 
      #2\\ 
      #3 
    \end{pmatrix}}}

\newcommand{\norme}[1]{\left\lVert\overrightarrow{#1}\right\rVert}
\newcommand{\vecteur}[1]{\overrightarrow{#1}}
\title{La Méthode des Éléments Finis: TD3}
\author{ \textsc{Ibrahim ALAME}}
\date{23/02/2024}
  \begin{document}
  \lstset{
    frame       = single,
    numbers     = left,
    showspaces  = false,
    showstringspaces    = false,
    captionpos  = t,
    caption     = \lstname
}
\maketitle
\section*{Problème de Poisson en dim 2}
\subsection*{Solution exacte}
On considère une membrane carrée de coté $a=1$ qui se déforme sous l'effet d'une charge surfacique $ f(x,y)$ . La membrane est sous tension et fixée sur les bords.  On note $ u(x,y)$ la déformée

\begin{center}
\begin{tikzpicture}[domain=0:5,scale=0.8]
  \draw[->] (0,0) -- (4.5,0)  node[right] {$x$};
  \draw[->] (0,0) -- (2,1.333) node[left] {$y$};
  \draw[->] (0,0) -- (0,2) node[left] {$z$};
\draw  [line width=1pt](0,0) -- (1.5,1)--++(3,0)--++(-1.5,-1)--cycle;
%\draw [orange,domain=0:2 ] plot(\x,1/4*\x*\x*\x-1/2*\x*\x-5/4*\x+5/2);
\end{tikzpicture}
\end{center}

 La déformée $u$ est solution du problème de Poisson avec conditions de Dirichlet homogènes posé sur le domaine
$\Omega = ]0, 1[ \times ]0, 1[$: 
\begin{equation}
\left\{
\begin{array}{l}
-\Delta u = f \mbox{ dans } \Omega \quad \mbox{ bilan des forces }\\
u=0  \mbox{ sur } \partial \Omega \quad \mbox{ condition limite }
\end{array}
\right.
\label{dirichletDim1}
\end{equation}
On admet que
\begin{enumerate}
\item La solution exacte analytique de l'équation de Poisson \eqref{dirichletDim1}, est donnée,  dans le cas d'un chargement uniforme $f=-1$,  par la somme de la suite double:
\[u(x,y)=\sum_{\ell=1}^{\infty}\sum_{m=1}^{\infty}u_{\ell m}\sin((2\ell-1)\pi x)\sin((2m-1)\pi y)\]
avec  $\displaystyle u_{\ell m}=\frac{-16}{((2\ell-1)^{2}+(2m-1)^{2})\pi^{4}(2\ell-1)(2m-1)}$
\item La flèche au point central de la membrane se calcule à l'aide de la somme:
$$\displaystyle u_{max}=\sum_{\ell=1}^{\infty}\sum_{m=1}^{\infty}\frac{16(-1)^{\ell+1}(-1)^{m+1}}{((2\ell-1)^{2}+(2m-1)^{2})\pi^{4}(2\ell-1)(2m-1)}$$
On peut calculer une valeur approchée très précise de cette série avec Maple, et on trouve (pour $ 1\leq m,\ell \leq 200$):
\[\displaystyle U_{max}=-0.07367135123\]
\end{enumerate}

\subsection*{Solution approchée par éléments finis P1-triangle}
On considère le maillage suivant :

  %Maillage
\begin{center}
\begin{tikzpicture}[domain=0:5,scale=1.5]
 \pgfmathsetmacro{\alpha}{0.05}
  \pgfmathsetmacro{\a}{1.2}
  \pgfmathsetmacro{\N}{4}
  \draw[->] (0,0) -- (\N*\a+\a/2,0)  node[right] {$x$};
  \draw[->] (0,0) -- (0,\N*\a+\a/2) node[left] {$y$};

  \foreach \n in {0,1,...,\N}{
   \draw[blue,thick](\n*\a,0)-- ++(0,\N*\a);
    \draw[blue,thick](0,\n*\a)-- ++(\N*\a,0);
}
  \foreach \n in {0,1,...,\N}{
   \draw[blue,thick](\n*\a,0)-- (\N*\a,\N*\a-\n*\a);
    \draw[blue,thick](0,\n*\a)-- (\N*\a-\n*\a,\N*\a);
}
\pgfmathsetmacro{\Z}{int(\N-1)}
  \foreach \i in {1,...,\Z}{
  		\foreach \j in {1,...,\Z}{
  		\pgfmathsetmacro{\x}{int((\N-1)*(\j-1)+(\i-1))}
   \draw[red,thick](\i*\a,\j*\a) node[below right]{\x};
   }
}
\pgfmathsetmacro{\Z}{int((\N-1)*(\N-1))}
  \foreach \i in {0,1,...,\N}{
  		\pgfmathsetmacro{\x}{int(\Z+\i)}
   		\draw[red,thick](\i*\a,0) node[below]{\x};
}
\pgfmathsetmacro{\Z}{int((\N-1)*(\N-1)+\N)}
  \foreach \i in {1,...,\N}{
  		\pgfmathsetmacro{\x}{int(\Z+\i)}
   		\draw[red,thick](\N*\a,\i*\a) node[right]{\x};
}
\pgfmathsetmacro{\Z}{int((\N-1)*(\N-1)+2*\N)}
  \foreach \i in {1,...,\N}{
  		\pgfmathsetmacro{\x}{int(\Z+\i)}
   		\draw[red,thick](\N*\a-\i*\a,\N*\a) node[above]{\x};
}
\pgfmathsetmacro{\Z}{int((\N-1)*(\N-1)+3*\N)}
\pgfmathsetmacro{\U}{int(\N-1)}
  \foreach \i in {1,...,\U}{
  		\pgfmathsetmacro{\x}{int(\Z+\i)}
   		\draw[red,thick](0,\N*\a-\i*\a) node[left]{\x};
}
\pgfmathsetmacro{\M}{int(\N-1)}
  \foreach \i in {0,1,...,\M}{
  		\foreach \j in {0,1,...,\M}{
  		\pgfmathsetmacro{\x}{int(\N*\j+\j+\i)}
  		\pgfmathsetmacro{\k}{int(2*\i+2*\j*\N)}
   \draw[gray,thick](\i*\a+\a/3,\j*\a+2*\a/3) node{(\k)};
   \pgfmathsetmacro{\k}{int(2*\i+1+2*\j*\N)}
   \draw[gray,thick](\i*\a+2*\a/3,\j*\a+\a/3) node{(\k)};
   }
}

\end{tikzpicture}
\end{center}

\begin{enumerate}
\item Rappeler la formulation variationnelle du problème. Quel est l'ordre de la matrice de rigidité $A$ pour le maillage ci-dessus ?
\item Les 9 sommets intérieurs sont numérotés ligne par ligne de la gauche vers la droite en partant
de la ligne du bas et en remontant jusqu'à la ligne du haut. Les 32 triangles du maillage sont
numérotés de manière analogue. Toutes les mailles sont des triangles rectangles isocèles de
côté $h =\frac 14$ et d'hypoténuse $\ell=\frac{\sqrt 2}{4}$. Calculer les coefficients de la matrice élémentaire de rigidité?
\item Évaluer les coefficients de la matrice de rigidité.
\item On considère maintenant un maillage structuré plus fin de pas $h =\frac 1{n+1}$ dans chaque direction spatiale ($n = 3$ dans l'exemple ci-dessus). Montrer que la matrice de rigidité a une structure
bloc tridiagonale. Quel est le rapport entre le nombre de coefficients non-nuls et le nombre
total de coefficients dans la matrice de rigidité (on retiendra le terme dominant en puissances
de $n$).
\item Calculer le second membre et résoudre le système linéaire. Comparer la flèche du milieu avec la solution exacte.
\end{enumerate}


\subsection*{Eléments finis parallélotopes}
Soit le carré $\Omega=[0,1]\times [0,1]$ et soit une partition en carrés égaux de coté $\frac 1n$.
  %Maillage
\begin{center}
\begin{tikzpicture}[domain=0:5,scale=1]
 \pgfmathsetmacro{\alpha}{0.05}
  \pgfmathsetmacro{\a}{1.2}
  \pgfmathsetmacro{\N}{4}
  \draw[->] (0,0) -- (\N*\a+\a/2,0)  node[right] {$x$};
  \draw[->] (0,0) -- (0,\N*\a+\a/2) node[left] {$y$};

  \foreach \n in {0,1,...,\N}{
   \draw[blue,thick](\n*\a,0)-- ++(0,\N*\a);
    \draw[blue,thick](0,\n*\a)-- ++(\N*\a,0);
}

\pgfmathsetmacro{\Z}{int(\N-1)}
  \foreach \i in {1,...,\Z}{
  		\foreach \j in {1,...,\Z}{
  		\pgfmathsetmacro{\x}{int((\N-1)*(\j-1)+(\i-1))}
   \draw[red,thick](\i*\a,\j*\a) node[below right]{\x};
   }
}
\pgfmathsetmacro{\Z}{int((\N-1)*(\N-1))}
  \foreach \i in {0,1,...,\N}{
  		\pgfmathsetmacro{\x}{int(\Z+\i)}
   		\draw[red,thick](\i*\a,0) node[below]{\x};
}
\pgfmathsetmacro{\Z}{int((\N-1)*(\N-1)+\N)}
  \foreach \i in {1,...,\N}{
  		\pgfmathsetmacro{\x}{int(\Z+\i)}
   		\draw[red,thick](\N*\a,\i*\a) node[right]{\x};
}
\pgfmathsetmacro{\Z}{int((\N-1)*(\N-1)+2*\N)}
  \foreach \i in {1,...,\N}{
  		\pgfmathsetmacro{\x}{int(\Z+\i)}
   		\draw[red,thick](\N*\a-\i*\a,\N*\a) node[above]{\x};
}
\pgfmathsetmacro{\Z}{int((\N-1)*(\N-1)+3*\N)}
\pgfmathsetmacro{\U}{int(\N-1)}
  \foreach \i in {1,...,\U}{
  		\pgfmathsetmacro{\x}{int(\Z+\i)}
   		\draw[red,thick](0,\N*\a-\i*\a) node[left]{\x};
}
\pgfmathsetmacro{\M}{int(\N-1)}
  \foreach \i in {0,1,...,\M}{
  		\foreach \j in {0,1,...,\M}{
  		\pgfmathsetmacro{\x}{int(\N*\j+\j+\i)}
  		\pgfmathsetmacro{\k}{int(\i+\j*\N)}
   \draw[gray,thick](\i*\a+\a/2,\j*\a+\a/2) node{(\k)};
   }
}

\end{tikzpicture}
\end{center}

\begin{enumerate}
\item Calculer la matrice élémentaire d'assemblage associée au problème de Dirichlet et à l'élément fini {\em carré unité de type (1)}.
\item Calculer la matrice assemblée lorsque l'on numérote par lignes.
\end{enumerate}
\subsection*{Éléments triangles équilatéraux}
Soit  $\Omega$ un losange d'angle au sommet $\frac{\pi}{3}$ et de coté 1  et soit une partition en triangles équilatéraux de coté $\frac 1n$.
  %Maillage
\begin{center}
\begin{tikzpicture}[domain=0:5,scale=1]
 \pgfmathsetmacro{\alpha}{0.05}
  \pgfmathsetmacro{\a}{1}
  \pgfmathsetmacro{\h}{1.732}
  \pgfmathsetmacro{\N}{3}
  \draw[->] (0,0) -- (2*\N*\a+\a,0)  node[right] {$x$};
  \draw[->] (0,0) -- (\N*\a+\a/2,\N*\h+\h/2)  node[right] {$y$};


  \foreach \n in {0,1,...,\N}{
   \draw[blue,thick](2*\n*\a,0)-- ++(\N*\a,\N*\h);
    \draw[blue,thick](\n*\a,\n*\h)-- ++(2*\N*\a,0);
}
  \foreach \n in {0,1,...,\N}{
   \draw[blue,thick](2*\n*\a,0)-- (\n*\a,\n*\h);
   \draw[blue,thick](2*\N*\a+\n*\a,\n*\h)-- (\N*\a+2*\n*\a,\N*\h);
}
\pgfmathsetmacro{\Z}{int(\N-1)}
  \foreach \i in {1,...,\Z}{
  		\foreach \j in {1,...,\Z}{
  		\pgfmathsetmacro{\x}{int((\N-1)*(\j-1)+(\i-1))}
   \draw[red,thick](2*\i*\a+\j*\a,\j*\h) node[below right]{\x};
   }
}
\pgfmathsetmacro{\Z}{int((\N-1)*(\N-1))}
  \foreach \i in {0,1,...,\N}{
  		\pgfmathsetmacro{\x}{int(\Z+\i)}
   		\draw[red,thick](2*\i*\a,0) node[below]{\x};
}
\pgfmathsetmacro{\Z}{int((\N-1)*(\N-1)+\N)}
  \foreach \i in {1,...,\N}{
  		\pgfmathsetmacro{\x}{int(\Z+\i)}
   		\draw[red,thick](2*\N*\a+\i*\a,\i*\h) node[right]{\x};
}
\pgfmathsetmacro{\Z}{int((\N-1)*(\N-1)+2*\N)}
  \foreach \i in {1,...,\N}{
  		\pgfmathsetmacro{\x}{int(\Z+\i)}
   		\draw[red,thick](3*\N*\a-2*\i*\a,\N*\h) node[above]{\x};
}
\pgfmathsetmacro{\Z}{int((\N-1)*(\N-1)+3*\N)}
\pgfmathsetmacro{\U}{int(\N-1)}
  \foreach \i in {1,...,\U}{
  		\pgfmathsetmacro{\x}{int(\Z+\i)}
   		\draw[red,thick](\N*\a-\i*\a,\N*\h-\i*\h) node[left]{\x};
}
\pgfmathsetmacro{\M}{int(\N-1)}
  \foreach \i in {0,1,...,\M}{
  		\foreach \j in {0,1,...,\M}{
  		\pgfmathsetmacro{\x}{int(\N*\j+\j+\i)}
  		\pgfmathsetmacro{\k}{int(2*\i+2*\j*\N)}
   \draw[gray,thick](2*\i*\a+\a+\j*\a,\j*\h+\h/2) node{(\k)};
     		\pgfmathsetmacro{\k}{int(\k+1)}
   \draw[gray,thick](2*\i*\a+2*\a+\j*\a,\j*\h+\h/2) node{(\k)};

   }
}

\end{tikzpicture}
\end{center}
\begin{enumerate}
\item Calculer la matrice élémentaire d'assemblage associée au problème de Dirichlet et à l'élément fini {\em triangle de type (1)}.
\item Calculer la matrice assemblée lorsque l'on numérote par lignes.
\end{enumerate}

\end{document}
