\documentclass{article}
%\documentclass[12pt,twoside, openany]{extbook}
\usepackage[francais]{babel}
\usepackage[utf8]{inputenc} % Required for including letters with accents
\usepackage[T1]{fontenc} % Use 8-bit encoding that has 256 glyphs
\usepackage{pythontex}
\usepackage{amsthm}
\usepackage{amsmath}
\usepackage{amssymb}
\usepackage{mathrsfs}
\usepackage{graphicx}
\usepackage{geometry}
\usepackage{stmaryrd}
\usepackage{tikz}
\usetikzlibrary{patterns}
%\usetikzlibrary{intersections}
\usetikzlibrary{calc} 
%\usepackage{tkz-tab}
\usepackage{stmaryrd}
%\usepackage{tikz}
%\usetikzlibrary{tikzmark}
\usepackage{empheq}
\usepackage{longtable}
\usepackage{booktabs} 
\usepackage{array}
\usepackage{pstricks}
\usepackage{pst-3dplot}
\usepackage{pst-tree}
\usepackage{pstricks-add}
\usepackage{upgreek}
%\usepackage{epstopdf}
\usepackage{eolgrab}
\usepackage{chngpage}
 \usepackage{calrsfs}
 % Appel du package pythontex 
\usepackage{pythontex}
 \usepackage{enumitem}
\usetikzlibrary{decorations.pathmorphing}
\def \de {{\rm d}}
\def \ch {{\rm ch}}
\def \sh {{\rm sh}}
\def \th {{\rm th}}

\usepackage{lscape}
\usepackage{color}
%\usepackage{xcolor}
%\usepackage{textcomp}
\newcommand{\mybox}[1]{\fbox{$\displaystyle#1$}}
\newcommand{\myredbox}[1]{\fcolorbox{red}{white}{$\displaystyle#1$}}
\newcommand{\mydoublebox}[1]{\fbox{\fbox{$\displaystyle#1$}}}
\newcommand{\myreddoublebox}[1]{\fcolorbox{red}{white}{\fcolorbox{red}{white}{$\displaystyle#1$}}}
\newtheorem{definition}{Définition}
\newtheorem{theorem}{Théorème}

\definecolor{purple2}{RGB}{153,0,153} % there's actually no standard purple
\definecolor{green2}{RGB}{0,153,0} % a darker green
\usepackage{xcolor}
\usepackage{listings}

\lstdefinestyle{Python}{
    language        = Python,
    basicstyle      = \ttfamily,
    keywordstyle    = \color{blue},
    keywordstyle    = [2] \color{teal}, % just to check that it works
    stringstyle     = \color{violet},
    commentstyle    = \color{red}\ttfamily
}

\usepackage{amsmath} 
\renewcommand{\overrightarrow}[1]{\vbox{\halign{##\cr 
  \tiny\rightarrowfill\cr\noalign{\nointerlineskip\vskip1pt} 
  $#1\mskip2mu$\cr}}}

\newcommand{\Coord}[3]{% 
  \ensuremath{\overrightarrow{#1}\, 
    \begin{pmatrix} 
      #2\\ 
      #3 
    \end{pmatrix}}}

\newcommand{\norme}[1]{\left\lVert\overrightarrow{#1}\right\rVert}
\newcommand{\vecteur}[1]{\overrightarrow{#1}}
\title{La Méthode des Éléments Finis: corrigé du TD2}
\author{ \textsc{Ibrahim ALAME}}
\date{05/10/2021}
  \begin{document}
  \lstset{
    frame       = single,
    numbers     = left,
    showspaces  = false,
    showstringspaces    = false,
    captionpos  = t,
    caption     = \lstname
}
\maketitle
\section*{Problème de Poisson en dim 2}
\subsection*{Solution exacte}


Nous avons le problème
\begin{equation}
\left\{
\begin{array}{l}
-\Delta u = f \mbox{ dans } \Omega \quad \mbox{ bilan des forces }\\
u=0  \mbox{ sur } \partial \Omega \quad \mbox{ condition limite }
\end{array}
\right.
\label{dirichletDim1}
\end{equation}
On suppose que le chargement est uniforme $f(x,y)=-1$ et on admet que la solution analytique du problème est donnée par la somme de la série double:
\[u(x,y)=\sum_{l=1}^{\infty}\sum_{m=1}^{\infty}u_{lm}\sin\left((2l-1)\pi x\right)\sin\left((2m-1)\pi y\right)\]   
avec  \[u_{lm}=\frac{-16}{((2l-1)^{2}+(2m-1)^{2})\pi^{4}(2l-1)(2m-1)}\]  
La valeur maximale $u_{max}$ de la déformation se trouve au centre et a pour expression:
\[u_{max}=\sum_{l=1}^{\infty}\sum_{m=1}^{\infty}\frac{16(-1)^{l+1}(-1)^{m+1}}{((2l-1)^{2}+(2m-1)^{2})\pi^{4}(2l-1)(2m-1)}\]
On peut calculer une valeur approchée très précise de cette série avec Maple, et on trouve (pour $ m=l=200$ ):
\[u_{max}=-0.07367135123\]

\subsection*{Solution approchée par éléments finis P1-triangle}
On rappelle les coordonnées barycentriques dans un triangle rectangle isocèle de côté $h$.
\begin{center}
\includegraphics[width=5cm]{triangleRectangle.png}
\end{center}
On a \[\lambda_1(\mbox{\bf x})=1-\frac{x+y}{h}\quad \lambda_2(\mbox{\bf x})=\frac{x}{h}\quad \lambda_3(\mbox{\bf x})=\frac{y}{h}\]
et 
\[\nabla\lambda_1=\frac{1}{h}\left(\begin{array}{c}
-1 \\ -1 \end{array}\right)
\quad 
\nabla\lambda_2=\frac{1}{h}\left(\begin{array}{c}
1 \\ 0 \end{array}\right)
\quad 
\nabla\lambda_3=\frac{1}{h}\left(\begin{array}{c}
0 \\ 1 \end{array}\right)
\]
On définit les fonctions chapeau $\varphi_i$ pour un élément fini triangle de type (1) par:

Soit $S_i$ un sommet intérieur du maillage, $\forall i \in \{1 . . . N_{so}^{int}\}$
\begin{itemize}
\item ${\cal K}(S_i)$ est l'ensemble des mailles ayant $S_i$ pour sommet
\item $\lambda_{K,S_i}$ est la coordonnée barycentrique de $K$ associée au sommet $S_i$.
\[\varphi_i=\left\{\begin{array}{ll}
\lambda_{K,S_i}& \mbox{si }{\bf x}\in K \mbox{ pour } K\in {\cal K}(S_i)\\
0 & \mbox{sinon }
\end{array}\right.
\]
\begin{center}
\includegraphics[scale=0.3]{fonctionChapeau.png} 
\end{center}
\end{itemize}

\begin{enumerate}
\item Le terme générique de la matrice de rigidité est $A_{ij} =\int_{\Omega}\nabla\varphi_i\cdot\nabla\varphi_j$  où $\varphi_i$ est la fonction chapeau associée au sommet intérieur de numéro $i$. La matrice de rigidité est d'ordre $N_{so}^{int} = 9$.

Une observation essentielle est que
\[(A_{ij} \neq 0) \Longrightarrow (S_i \mbox{ et }S_j \mbox{ sont des sommets d’un même triangle})\]
\begin{center}
\includegraphics[scale=0.3]{assemblage00.png} 
\end{center}
\item En considérant l'intersection des supports des fonctions chapeau, on obtient la disposition suivante des coefficients a priori non-nuls (indiqués par le symbole $\bullet$) :
\begin{center}
\includegraphics[scale=0.3]{profileA.png} 
\end{center}
\item Pour des raisons de symétrie et d’invariance par translation, il vient
\begin{center}
\includegraphics[scale=0.3]{matriceA.png} 
\end{center}
Il nous reste à déterminer les coefficients réels $a$, $b$, $c$ et $d$ donnés par les formules suivantes :
\[\begin{array}{l}
\displaystyle a=\int_{\Omega}\left|\nabla\varphi_1\right|^2,\\
\displaystyle b=\int_{\Omega}\nabla\varphi_1\cdot\nabla\varphi_2,\\
\displaystyle c=\int_{\Omega}\nabla\varphi_1\cdot\nabla\varphi_4,\\
\displaystyle d=\int_{\Omega}\nabla\varphi_1\cdot\nabla\varphi_5.
\end{array}\]
On découpe les intégrales sur Ω en une somme d'intégrales sur les mailles et on ne conserve que les mailles intersectant le support des fonctions chapeau à intégrer. Il vient
\[\begin{array}{l}
\displaystyle a=\int_{K_1}\left|\nabla\varphi_1\right|^2+\int_{K_2}\left|\nabla\varphi_1\right|^2+\int_{K_3}\left|\nabla\varphi_1\right|^2+\int_{K_{10}}\left|\nabla\varphi_1\right|^2+\int_{K_{11}}\left|\nabla\varphi_1\right|^2+\int_{K_{12}}\left|\nabla\varphi_1\right|^2\\
\displaystyle b=\int_{K_3}\nabla\varphi_1\cdot\nabla\varphi_2+\int_{K_{12}}\nabla\varphi_1\cdot\nabla\varphi_2\\
\displaystyle c=\int_{K_{10}}\nabla\varphi_1\cdot\nabla\varphi_4+\int_{K_{11}}\nabla\varphi_1\cdot\nabla\varphi_4\\
\displaystyle d=\int_{K_{11}}\nabla\varphi_1\cdot\nabla\varphi_5+\int_{K_{12}}\nabla\varphi_1\cdot\nabla\varphi_5
\end{array}\]
Considérons le coefficient $a$. La fonction $\varphi_1$ est affine son gradient  $\nabla\varphi_1$ est donc constant, on peut alors le calculer comme taux d'accroissement en choisissant deux points dans $K_1$.


\begin{center}
\begin{tikzpicture}[domain=0:5,scale=1]
  \draw[->] (0,0) -- (3,0)  node[right] {$x$};
  \draw[->] (0,0) -- (0,3) node[left] {$y$};
\draw  [line width=1pt](0,0) -- ++(2,2)--++(-2,0)--cycle;
%\draw [orange,domain=0:2 ] plot(\x,1/4*\x*\x*\x-1/2*\x*\x-5/4*\x+5/2);
 \draw (2,2)node[above left] {$S_1$};
 \draw (2,2)node[right] {$(\varphi_1=1)$};
 \draw (0,2)node[left] {$(\varphi_1=0)$};
 \draw (0,0)node[below] {$(\varphi_1=0)$};
 \draw (0.2,1.8)node[below right] {$K_1$};
\end{tikzpicture}
\end{center}



 Sa composante horizontale est $\frac{1-0}{h}=\frac 1h$. Sa composante verticale est $\frac{0-0}{h}$. Donc
\[\nabla\varphi_1|_{K_1}=\frac{1}{h}\left(\begin{array}{c}1\\0 \end{array}\right)\]
et comme $K_1$ est de mesure égale à $\frac{h^2}{2}$, il vient
\[\int_{K_{1}}\left|\nabla\varphi_1\right|^2=\frac{h^2}{2}\frac{1}{h^2}=\frac 12\]
\begin{center}
\begin{tikzpicture}[domain=0:5,scale=1]
  \draw[->] (0,0) -- (5,0)  node[right] {$x$};
  \draw[->] (0,0) -- (0,3) node[left] {$y$};
\draw  [line width=1pt](2,0) -- ++(2,2)--++(-2,0)--cycle;
%\draw [orange,domain=0:2 ] plot(\x,1/4*\x*\x*\x-1/2*\x*\x-5/4*\x+5/2);
\draw (2,2)node[above] {$S_1$};
 \draw (4,2)node[above] {$S_2$};
 \draw (4,2)node[right] {$(\varphi_1=0)$};
 \draw (2,2)node[left] {$(\varphi_1=1)$};
 \draw (2,0)node[below] {$(\varphi_1=0)$};
 \draw (2.2,1.8)node[below right] {$K_3$};
\end{tikzpicture}
\end{center}
De même,
\[\nabla\varphi_1|_{K_3}=\left(\begin{array}{c}\frac{0-1}{h}\\\frac{1-0}{h} \end{array}\right)=\frac 1h\left(\begin{array}{c}-1\\1 \end{array}\right)\]
si bien que
\[\int_{K_{3}}\left|\nabla\varphi_1\right|^2=\frac{h^2}{2}\frac{2}{h^2}=1\]
Enfin, pour des raisons de symétrie,
\[\int_{K_{1}}\left|\nabla\varphi_1\right|^2=\int_{K_{2}}\left|\nabla\varphi_1\right|^2=\int_{K_{11}}\left|\nabla\varphi_1\right|^2=\int_{K_{12}}\left|\nabla\varphi_1\right|^2\]
et
\[\int_{K_{3}}\left|\nabla\varphi_1\right|^2=\int_{K_{10}}\left|\nabla\varphi_1\right|^2\]
En rassemblant les contributions ci-dessus, on obtient
\[a = 4.\]
Calcul de b:

Par symétrie $\nabla\varphi_2|_{K_3}=\nabla\varphi_1|_{K_1}=\frac{1}{h}\left(\begin{array}{c}1\\0 \end{array}\right)$. Donc sur $K_3$:
\[\nabla\varphi_1\cdot\nabla\varphi_2=\left<\frac 1h\left(\begin{array}{c}-1\\1 \end{array}\right),\frac 1h\left(\begin{array}{c}1\\0 \end{array}\right)\right>=-\frac{1}{h^2}\Longrightarrow \int_{K_3} \nabla\varphi_1\cdot\nabla\varphi_2 =-\frac 12\]
de même sur $K_{12}$
\[\nabla\varphi_1\cdot\nabla\varphi_2=\left<\frac 1h\left(\begin{array}{c}1\\0 \end{array}\right),\frac 1h\left(\begin{array}{c}-1\\1 \end{array}\right)\right>=-\frac{1}{h^2}\Longrightarrow \int_{K_{12}} \nabla\varphi_1\cdot\nabla\varphi_2 =-\frac 12\]
Donc
\[b=\int_{K_3} \nabla\varphi_1\cdot\nabla\varphi_2 + \int_{K_{12}} \nabla\varphi_1\cdot\nabla\varphi_2 =-\frac 12-\frac 12=-1\]
En procédant comme ci-dessus pour les deux autres coefficients $b$, $c$ et $d$, il vient
\[ c = -1\quad\mbox{ et }\quad d = 0.\]
La nullité du coefficient $d$ provient du fait que sur les deux triangles $K_{11}$ et $K_{12}$, les gradients des fonctions chapeau $\varphi_1$ et $\varphi_5$ sont orthogonaux.

\item Il y a $n$ sommets intérieurs dans chaque direction spatiale, donc au total $n^2$ sommets intérieurs dans le maillage. La matrice de rigidité est d'ordre $n^2$ et le nombre total de ses coefficients est $n^4$. En raisonnant sur les supports des fonctions chapeau, on obtient la structure bloc tridiagonale suivante :
\begin{center}
\includegraphics[scale=0.25]{matriceAparBlocs.png} 
\end{center}
Les blocs $B$, $C$ et $O$ sont des matrices d'ordre $n$ et il y a $n$ blocs par ligne dans la structure de la matrice $A$. De plus, $B = \mbox{tridiag}(-1, 4, -1)$, $C = -I$ où $I$ est la matrice identité d'ordre
$n$ et $O$ est le bloc nul d'ordre $n$. La plupart des lignes de $A$ ont 5 coefficients non-nuls. Le rapport demandé est donc de l'ordre de $5n^{-2}\ll 1$. On dit que la matrice de rigidité est creuse.
\end{enumerate}
Pour n=3 nous obtenons la matrice
\[
 A=\left[ 
{\begin{array}{rrrrrrrrr}
4 & -1 & 0 & -1 & 0 & 0 & 0 & 0 & 0 \\
-1 & 4 & -1 & 0 & -1 & 0 & 0 & 0 & 0 \\
0 & -1 & 4 & 0 & 0 & -1 & 0 & 0 & 0 \\
-1 & 0 & 0 & 4 & -1 & 0 & -1 & 0 & 0 \\
0 & -1 & 0 & -1 & 4 & -1 & 0 & -1 & 0 \\
0 & 0 & -1 & 0 & -1 & 4 & 0 & 0 & -1 \\
0 & 0 & 0 & -1 & 0 & 0 & 4 & -1 & 0 \\
0 & 0 & 0 & 0 & -1 & 0 & -1 & 4 & -1 \\
0 & 0 & 0 & 0 & 0 & -1 & 0 & -1 & 4
\end{array}}
 \right] 
\]
Le second membre $F=(f_i)_{i=1,9}$ où 
\[f_i=\int_{supp(\varphi_i)}f(x,y)\varphi_i(x,y)\de x\de y=-\int_{supp(\varphi_i)}\varphi_i\de x\de y=-\frac{1}{3}\times 6\times\frac{h^2}{2}=-h^2=-\frac{1}{16}\]

\[
F := -\left[{\displaystyle \frac {1}{16}} , \,{\displaystyle \frac {1}{
16}} , \,{\displaystyle \frac {1}{16}} , \,{\displaystyle \frac {
1}{16}} , \,{\displaystyle \frac {1}{16}} , \,{\displaystyle 
\frac {1}{16}} , \,{\displaystyle \frac {1}{16}} , \,
{\displaystyle \frac {1}{16}} , \,{\displaystyle \frac {1}{16}} \right]
\]
Le système linéaire $AU=F$ a pour solution
\[
 U=-\left[  \! {\displaystyle \frac {11}{256}} , \,{\displaystyle 
\frac {7}{128}} , \,{\displaystyle \frac {11}{256}} , \,
{\displaystyle \frac {7}{128}} , \,{\displaystyle \frac {9}{128}
} , \,{\displaystyle \frac {7}{128}} , \,{\displaystyle \frac {11
}{256}} , \,{\displaystyle \frac {7}{128}} , \,{\displaystyle 
\frac {11}{256}}  \!  \right] 
\]

La flèche maximale est $\frac{9}{128} = -0.0703125\approx u_{max}=-0.07367135123$. En 9 points de discrétisation nous avons une précision de $3\times 10^{-3}$.

\begin{figure}[!h] %on ouvre l'environnement figure
 
\centering
 
\includegraphics[width=10cm]{poissonSurfaceMEF.png}
 
% Ici on redimensionne une image
 
\caption{La membrane déformée avec 9 points intérieurs (n=3)}
 
\label{membraneDéformée}
 
\end{figure}
\end{document}






%\begin{center}
%\begin{tikzpicture}[domain=0:5,scale=0.8]
% \pgfmathsetmacro{\alpha}{0.05}
%  \pgfmathsetmacro{\a}{2}
%  \pgfmathsetmacro{\N}{4}
%  \draw[->] (0,0) -- (\N*\a+0.4*\a,0)  node[right] {$x$};
%  \draw[->] (0,0) -- (0,\N*\a+0.4*\a) node[left] {$y$};
%
%  \foreach \n in {0,1,...,\N}{
%   \draw[blue,thick](\n*\a,0)-- ++(0,\N*\a);
%    \draw[blue,thick](0,\n*\a)-- ++(\N*\a,0);
%}
%  \foreach \n in {0,1,...,\N}{
%   \draw[blue,thick](\n*\a,0)-- (\N*\a,\N*\a-\n*\a);
%    \draw[blue,thick](0,\n*\a)-- (\N*\a-\n*\a,\N*\a);
%}
%\draw (2*\a,0)  node[below] {$\scriptstyle  \ell h$};
%\draw (0,3*\a)  node[left] {$\scriptstyle  m h$};
%\draw[fill=orange!30, fill opacity=0.6] (2*\a,2*\a)  -- ++(\a,0) -- ++(0,\a) ;
%  \path[fill=gray] (2*\a,2*\a) circle (0.6mm)node[below right] {$ a_i$};
%  \draw[blue] (2.6*\a,2.2*\a) node{$\scriptstyle  K_{l,m,1}$};
%  \draw[fill=olive!30, fill opacity=0.6] (2*\a,2*\a)  -- ++(0,\a) -- ++(\a,0) ;
%  \draw[blue] (2.4*\a,2.8*\a) node{$\scriptstyle  K_{l,m,2}$};
%\end{tikzpicture}
%\end{center}

