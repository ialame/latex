\documentclass[11pt]{article}
\usepackage[francais]{babel}
\usepackage[utf8]{inputenc} % Required for including letters with accents
\usepackage[T1]{fontenc} % Use 8-bit encoding that has 256 glyphs
\usepackage{pythontex}
\usepackage{amsthm}
\usepackage{amsmath}
\usepackage{amssymb}
\usepackage{mathrsfs}
\usepackage{graphicx}
\usepackage{geometry}
\usepackage{stmaryrd}
\usepackage{tikz}
\usetikzlibrary{matrix}
\usetikzlibrary{patterns}
%\usetikzlibrary{intersections}
\usepackage[cache=false]{minted}
 \definecolor{darkWhite}{rgb}{0.94,0.94,0.94}
 \usepackage[cache=false]{minted}
\definecolor{LightGray}{gray}{0.9}
\definecolor{monOrange}{rgb}{0.97,0.35,0.04}
\def \de {{\rm d}}
\usepackage{color}
\usepackage{xcolor}
\newcommand{\mybox}[1]{\fbox{$\displaystyle#1$}}
\newcommand{\myredbox}[1]{\fcolorbox{blue}{white}{$\displaystyle#1$}}
\newcommand{\mybluebox}[1]{\fcolorbox{blue}{white}{$\displaystyle#1$}}
\newcommand{\mygreenbox}[1]{\fcolorbox{green}{white}{$\displaystyle#1$}}
\newcommand{\mydoublebox}[1]{\fbox{\fbox{$\displaystyle#1$}}}
\newcommand{\myreddoublebox}[1]{\fcolorbox{red}{white}{\fcolorbox{red}{white}{$\displaystyle#1$}}}

\usepackage{geometry}
 \geometry{
 a4paper,
 total={210mm,297mm},
 left=20mm,
 right=20mm,
 top=20mm,
 bottom=20mm,
 }
%%%%%%%%%%%%%%%%%%
\usepackage{tikz}
\usetikzlibrary{patterns}
 

%%%%%%%%%%%%%%%%%%%
\title{Méthode des éléments finis: TP1}
\author{Ibrahim ALAME}
\date{4/10/2022}
\begin{document}
\maketitle
  
  %%%%%%%%%%%%%%%%%%%%%%%%%%%%%%%%%%%%%%%%%%%%%%%%%%
  
  Une poutre à treillis est une structure formée par un agencement triangulaire d'éléments linéaires (ou quasi linéaires) dont les extrémités sont reliées au niveau de points d'assemblage appelés nœuds. Les poutres à treillis se composent de triangles, c'est-à-dire des formes géométriquement stables. En effet, un triangle présente des angles fixes qui ne peuvent être ni agrandis, ni rétrécis sans céder au niveau des points d'assemblage, contrairement, notamment, à un rectangle, lequel peut devenir un parallélogramme.
\subsubsection*{Problème variationnelle et matrice élémentaire}
Une barre est une poutre qui ne transmet que des efforts de traction compression à ses extrémités. On rappelle la formulation variationnelle, vue en TD2,  d'une barre $OL$ homogène de module $E$, section constante $A$ et de longueur $L$ soumise à une sollicitation linéïque  $p(x)$ supposée nulle ici et à deux forces aux extrémités $\vec{f}_1$ et $\vec{f}_2$ s'écrit comme un problème abstrait de la forme:
\[
({\cal P}_v)\;\left\{
\begin{array}{l}
\mbox{Trouver } u\in V=H^1(\Omega) \mbox{ vérifiant}\\
a(u,v)=\ell(v)\quad \forall v\in V
\end{array}
\right.
\]

où \[ a(u,v)=\displaystyle EA\int_0^L\frac{\de u}{dx}\cdot \frac{\de v}{dx}\de x\quad \mbox{ et }\quad \ell(v)=f_1 \cdot v(0)+f_2 \cdot v(L)\]
Dans un maillage en $n+1$ points équidistants, si on choisit un élément fini de Lagrange de type (1) de longueur $h$, le système élémentaire s'écrit alors:
\[\frac{EA}{h}\left(\begin{array}{rr} 
1&-1\\-1&1
\end{array}\right) \left(\begin{array}{l} 
u_{1}\\u_{2}
\end{array}\right)=\left(\begin{array}{r} 
f_{1}\\f_{2}
\end{array}\right)
\]

\subsection*{Pont en treillis}
Le pont en treillis que l'on se propose d'étudier dans cette partie est constitué de $N=10$ sommets sur le tablier et $N+1$ sommets sur la poutre supérieur. Les triangles formés par les barres articulées comme indique la figure ci-après sont équilatéraux. Les 4 sommets aux extrémités sont encastrés. les poutres sont de même nature et de même section droite.
Soient $L$ la longueur du pont, $E$ le module de Young du matériau, $A$ l'aire des sections droites et $P$ est la force appliquée verticalement vers le bas en chaque nœud intérieur de la poutre supérieur (en $N-1$ sommets).

 Application numérique : on donne :
$A=100\mbox{mm}^2$ , $L=10 m$ , $E=200000$MPa , $P =-300$N. 



 \begin{center}
\begin{tikzpicture}[scale=0.8]

\pgfmathsetmacro{\X}{0}
\pgfmathsetmacro{\Y}{0}
\foreach \x in {-0.2,-0.1,...,0.3}{
\draw (\X-0.05,\Y+\x) --++ (-0.2,-0.2);
}
\draw[olive] (\X-0.05,\Y-0.2) - - ++(0,0.5);

\pgfmathsetmacro{\X}{0}
\pgfmathsetmacro{\Y}{1.7320508075688772}
\foreach \x in {-0.2,-0.1,...,0.3}{
\draw (\X-0.05,\Y+\x) --++ (-0.2,-0.2);
}
\draw[olive] (\X-0.05,\Y-0.2) - - ++(0,0.5);

\pgfmathsetmacro{\X}{6}
\pgfmathsetmacro{\Y}{0}
\foreach \x in {-0.2,-0.1,...,0.3}{
\draw (\X+0.05,\Y+\x) --++ (0.2,-0.2);
}
\draw[olive] (\X+0.05,\Y-0.2) - - ++(0,0.5);

\pgfmathsetmacro{\X}{6}
\pgfmathsetmacro{\Y}{1.7320508075688772}
\foreach \x in {-0.2,-0.1,...,0.3}{
\draw (\X+0.05,\Y+\x) --++ (0.2,-0.2);
}
\draw[olive] (\X+0.05,\Y-0.2) - - ++(0,0.5);

\draw[orange, double distance = 1pt] (0,1.7320508075688772) - - (1.0,1.7320508075688772);
\draw[orange, double distance = 1pt] (0.0,0) - - (2.0,0);
\draw[orange, double distance = 1pt] (2.0,0) - - (1.0,1.7320508075688772);
\draw[orange, double distance = 1pt] (1.0,1.7320508075688772) - - (0.0,0);
\draw[orange, double distance = 1pt] (1.0,1.7320508075688772) - - (3.0,1.7320508075688772);
\draw[orange, double distance = 1pt] (2.0,0) - - (4.0,0);
\draw[orange, double distance = 1pt] (4.0,0) - - (3.0,1.7320508075688772);
\draw[orange, double distance = 1pt] (3.0,1.7320508075688772) - - (2.0,0);
\draw[orange, double distance = 1pt] (3.0,1.7320508075688772) - - (5.0,1.7320508075688772);
\draw[orange, double distance = 1pt] (4.0,0) - - (6.0,0);
\draw[orange, double distance = 1pt] (6.0,0) - - (5.0,1.7320508075688772);
\draw[orange, double distance = 1pt] (5.0,1.7320508075688772) - - (4.0,0);
\draw[orange, double distance = 1pt] (5.0,1.7320508075688772) - - (6.0,1.7320508075688772);

\path[fill=gray]  (0,1.7320508075688772) circle (.75mm);
\path[fill=gray]  (0.0,0) circle (.75mm) ;
\path[fill=gray]  (1.0,1.7320508075688772) circle (.75mm) ;
\path[fill=gray]  (2.0,0) circle (.75mm) ;
\path[fill=gray]  (3.0,1.7320508075688772) circle (.75mm) ;
\path[fill=gray]  (4.0,0) circle (.75mm) ;
\path[fill=gray]  (5.0,1.7320508075688772) circle (.75mm) ;
\path[fill=gray]  (6.0,0) circle (.75mm) ;
\path[fill=black]  (6.0,1.7320508075688772) circle (.75mm) ;

\foreach \x in {1,3,5}{
\draw[very thick,olive,latex-] (\x,1.7320508075688772)--++(0,1);
}

\end{tikzpicture}
\end{center}
On note $\overrightarrow{F_i}=(f_{2i},f_{2i+1})$ la force appliquée au sommet $i$ et $\overrightarrow{U_i}=(u_{2i},u_{2i+1})$ le déplacement du sommet $i$.
On admet que le système élémentaire d'une barre $(q,r)$ faisant un angle $\theta$ avec l'horizontal est donnée par:
\[\frac{EA}{L}\left(\begin{array}{rrrr} 
C^2&CS&-C^2&-CS\\
CS&S^2&-CS&-S^2\\
-C^2&-CS&C^2&CS\\
-CS&-S^2&CS&S^2
\end{array}\right)\left(\begin{array}{l} 
u_{2q}\\u_{2q+1}\\u_{2r}\\u_{2r+1}
\end{array}\right) =\left(\begin{array}{l} 
f_{2q}\\f_{2q+1}\\f_{2r}\\f_{2r+1}
\end{array}\right)  \]
où  $C=\cos\theta$ et $S=\sin\theta$.

%\input{mef3.tex} 
\begin{enumerate}
\item {\bf Données et constantes physiques}
\begin{itemize}
\item
  Les poutres ont une longueur: \(L=1\)m, ou $2L$ et une même section
  rectangulaire 10 mm \(\times\) 20 mm, soit
  \(A=200\times 10^{-6}\mbox{ m}^2\).
  \item Les triangles sont équilatéraux de hauteur $H=L\sqrt 3$.
\item
  On applique en chaque noeud du tablier une charge \(P=-3000\) N.
\item
  Le module de Young est \(E=200000\) MPa.
\end{itemize}

\begin{minted}[
mathescape,
framesep=2mm,
baselinestretch=1.2,
%fontsize=\footnotesize,
bgcolor=LightGray,
%linenos
]{python}
import numpy as np
import math
import matplotlib.pyplot as plt

L=1.
H=L*math.sqrt(3)
A=200* 1E-6
E=200* 1E9
P=30000
\end{minted}

\item Écrire les coordonnées des sommets dans une liste 
\[ \mbox{Sommets} = \left[[x_0,y_0],[x_1,y_1],\cdots [x_{N_s-1},y_{N_s-1}]\right]\]
\item Écrire la liste de connectivité définissant les éléments du maillage:
\[ \mbox{Elements} = \left[[\mbox{origine}_i,\mbox{extremité}_i], \; i = 0 \cdots N_e-1\right]\]
\item Tracer la structure au repos à l'aide de la boucle:
\begin{minted}[
mathescape,
framesep=2mm,
baselinestretch=1.2,
%fontsize=\footnotesize,
bgcolor=LightGray,
%linenos
]{python}
for q,r in Elements:
    x1,y1 = Sommets[q]
    x2,y2 = Sommets[r]
    plt.plot([x1,x2],[y1,y2],color='gray',linestyle='solid')
    plt.axis('equal') # repère orthonormé
plt.show()
\end{minted}
\item {\bf Charges externes}
La charge appliquée en un noeud est représentée par un triplet:
\([n^{(i)},F^{(i)}_x,F^{(i)}_y]\) 
\begin{itemize}
\item  \(n_i\) est le numéro du nœud
\(i\). 
\item  \(F^{(i)}_x\) est la composante horizontale de la force
appliquée au noeud \(i\). 
\item \(F^{(i)}_y\) est la composante verticale de
la force appliquée au noeud \(i\).
\end{itemize}

\begin{minted}[
mathescape,
framesep=2mm,
baselinestretch=1.2,
%fontsize=\footnotesize,
bgcolor=LightGray,
%linenos
]{python}
Forces = [[3,0,-P],[5,0,-P],[7,0,-P]]
\end{minted}
Le second membre $B$ du système linéaire globale regroupe toutes les forces extérieures 
\begin{minted}[
mathescape,
framesep=2mm,
baselinestretch=1.2,
%fontsize=\footnotesize,
bgcolor=LightGray,
%linenos
]{python}
B=np.zeros(9*2,dtype=float)
for i,fx,fy in Forces:
    B[2*i]=fx
    B[2*i+1]=fy
\end{minted}

\item {\bf Matrice d'assemblage}
Pour chaque élément $[p,q]$ calculer sa matrice élémentaire $m$. Puis injecter $m$ dans la matrice globale $M$. (cf Chapitre3 page 43) où $M$ est une matrice {\tt float} 18$\times$18.
\begin{minted}[
mathescape,
framesep=2mm,
baselinestretch=1.2,
%fontsize=\footnotesize,
bgcolor=LightGray,
%linenos
]{python}
M=np.zeros((9*2,9*2),dtype=float)

def n(q,r):
    return [2*q,2*q+1,2*r,2*r+1]

for q,r in Elements:
    x1,y1 = Sommets[q]
    x2, y2 = Sommets[r]
    ell = math.sqrt((x2-x1)**2+(y2-y1)**2)
    c=(x2-x1)/ell
    s=(y2-y1)/ell
    CS = np.mat([c,s,-c,-s],dtype=float)
    CSt= np.transpose(CS)
    m = np.dot(CSt,CS)*A*E/ell
    for i in range(4):
        I=n(q,r)[i]
        for j in range(4):
            J=n(q,r)[j]
            M[I,J]+=m[i,j]
\end{minted}


\item {\bf Conditions aux appuis :} Une condition d'appui est un triplet:
\([n^{(i)},\varepsilon_y,\varepsilon_\theta]\) 
\begin{itemize}
\item \(n_i\) est le numéro
du noeud \(i\). 
\item \(\varepsilon_y=0\) si le déplacement suivant \(x\)
est bloqué sinon \(\varepsilon_y=1\). 
\item \(\varepsilon_\theta=0\) si
l'angle \(\theta\) est bloqué sinon \(\varepsilon_\theta=1\).
\end{itemize}
\begin{minted}[
mathescape,
framesep=2mm,
baselinestretch=1.2,
%fontsize=\footnotesize,
bgcolor=LightGray,
%linenos
]{python}
Conditions = [[0,0,0],[1,0,0],[6,0,0],[8,0,0]]
\end{minted}
Regrouper dans une liste $\ell$ les indices où les déplacements nuls sont appliqués.


\item Simplifier le système linéaire en éliminant les déplacements connus. Supprimer, pour cela, les lignes et colonnes correspondants aux indices des contraintes nulles:
\begin{minted}[
mathescape,
framesep=2mm,
baselinestretch=1.2,
%fontsize=\footnotesize,
bgcolor=LightGray,
%linenos
]{python}
l.sort()
l.reverse()
for i in l:
    M = np.delete(M, i, axis=0)
    M = np.delete(M, i, axis=1)
    B = np.delete(B, i)
\end{minted}
\item Résoudre le système linéaire $MU=B$:
\begin{minted}[
mathescape,
framesep=2mm,
baselinestretch=1.2,
%fontsize=\footnotesize,
bgcolor=LightGray,
%linenos
]{python}
U=np.linalg.solve(M,B)
\end{minted}
\item Réinsérer les déplacements nuls dans la solution $U$. Puis tracer la nouvelle position de la structure en amplifiant $U$, 50 fois.
\begin{minted}[
mathescape,
framesep=2mm,
baselinestretch=1.2,
%fontsize=\footnotesize,
bgcolor=LightGray,
%linenos
]{python}
l.sort()
for i in l:
    U = np.insert(U, i,0)

U=50*U
\end{minted}
\item Déterminer les réactions des appuis en calculant le produit de la matrice $M$ initiale (avant la suppression des lignes et colonnes) par le déplacement $U$ après insertion des zéros.
\end{enumerate}

  %%%%%%%%%%%%%%%%%%%%%%%%%%%%%%%%%%%%%%%%%%%%%%%%%%  
 \subsection*{Plaque rectangulaire}   
On considère le maillage suivant:    
  %Maillage
\begin{center}
\begin{tikzpicture}[domain=0:5,scale=2]
 \pgfmathsetmacro{\alpha}{0.05}
  \pgfmathsetmacro{\a}{1.2}
  \pgfmathsetmacro{\N}{4}
  \draw[->] (0,0) -- (\N*\a+\a/2,0)  node[right] {$x$};
  \draw[->] (0,0) -- (0,\N*\a+\a/2) node[left] {$y$};

  \foreach \n in {0,1,...,\N}{
   \draw[blue,thick](\n*\a,0)-- ++(0,\N*\a);
    \draw[blue,thick](0,\n*\a)-- ++(\N*\a,0);
}
  \foreach \n in {0,1,...,\N}{
   \draw[blue,thick](\n*\a,0)-- (\N*\a,\N*\a-\n*\a);
    \draw[blue,thick](0,\n*\a)-- (\N*\a-\n*\a,\N*\a);
}
\pgfmathsetmacro{\Z}{int(\N-1)}
  \foreach \i in {1,...,\Z}{
  		\foreach \j in {1,...,\Z}{
  		\pgfmathsetmacro{\x}{int((\N-1)*(\j-1)+(\i-1))}
   \draw[red,thick](\i*\a,\j*\a) node[below right]{\x};
   }
}
\pgfmathsetmacro{\Z}{int((\N-1)*(\N-1))}
  \foreach \i in {0,1,...,\N}{
  		\pgfmathsetmacro{\x}{int(\Z+\i)}
   		\draw[red,thick](\i*\a,0) node[below]{\x};
}
\pgfmathsetmacro{\Z}{int((\N-1)*(\N-1)+\N)}
  \foreach \i in {1,...,\N}{
  		\pgfmathsetmacro{\x}{int(\Z+\i)}
   		\draw[red,thick](\N*\a,\i*\a) node[right]{\x};
}
\pgfmathsetmacro{\Z}{int((\N-1)*(\N-1)+2*\N)}
  \foreach \i in {1,...,\N}{
  		\pgfmathsetmacro{\x}{int(\Z+\i)}
   		\draw[red,thick](\N*\a-\i*\a,\N*\a) node[above]{\x};
}
\pgfmathsetmacro{\Z}{int((\N-1)*(\N-1)+3*\N)}
\pgfmathsetmacro{\U}{int(\N-1)}
  \foreach \i in {1,...,\U}{
  		\pgfmathsetmacro{\x}{int(\Z+\i)}
   		\draw[red,thick](0,\N*\a-\i*\a) node[left]{\x};
}
\pgfmathsetmacro{\M}{int(\N-1)}
  \foreach \i in {0,1,...,\M}{
  		\foreach \j in {0,1,...,\M}{
  		\pgfmathsetmacro{\x}{int(\N*\j+\j+\i)}
  		\pgfmathsetmacro{\k}{int(2*\i+2*\j*\N)}
   \draw[gray,thick](\i*\a+\a/3,\j*\a+2*\a/3) node{(\k)};
   \pgfmathsetmacro{\k}{int(2*\i+1+2*\j*\N)}
   \draw[gray,thick](\i*\a+2*\a/3,\j*\a+\a/3) node{(\k)};
   }
}

\end{tikzpicture}
\end{center}

On admet que la matrice élémentaire est la suivante:
\[m=\frac 12 \left(\begin{array}{ccc} 1&0&-1\\0&1&-1\\-1&-1&2\end{array}\right)\]  
\begin{enumerate}
\item Donner la liste des coordonnées {\tt Sommets} et la liste des éléments ${\tt Elements}$.
\item Déterminer la matrice d'assemblage $M$.
\item Determiner la sous matrice des 9 premières lignes et 9 premières colonnes.
\item Pour un second membre uniforme ($F(x,y)=1$), déterminer et tracer la solution de $Mu=F$.
\end{enumerate}


    
    
    
    
\end{document}
