\documentclass[a4paper]{article} 
\usepackage[francais]{babel}
\usepackage[utf8]{inputenc} % Required for including letters with accents
\usepackage[T1]{fontenc} % Use 8-bit encoding that has 256 glyphs

\usepackage{amsthm}
\usepackage{amsmath}
\usepackage{amssymb}
\usepackage{mathrsfs}
\usepackage{graphicx}
\usepackage{geometry}
\usepackage{stmaryrd}
\usepackage{tikz}

\def \de {{\rm d}}

\title{TD 4 Analyse numérique (B1-TP1)}
\author{Ibrahim ALAME}
\date{22/3/2024}
\begin{document}
\maketitle



\section{Méthode des différences finies}
%%%%%%%%%%%%%%%%%%%%%%%%%%%%%%%%%%%%%%%%%%%%%%%%%%%%%%%%%%%%%%%%%%%%%%%%%%%%%%%%
 On considère le problème aux limites suivant
\[\left\{\begin{array}{l}
-\frac{d^2u(x)}{dx^2}+c u(x)=f(x)\quad\mbox{dans }]0,1[\\
u(0)=u(1)=0
\end{array}\right.
\]
où $c$ est une constante positive, $f(x)$ une fonction continue. Cette équation modélise de nombreux problèmes de statique, par exemple la diffusion de la chaleur dans une barre.

On divise l'intervalle $[0; 1]$ selon un pas $h = \frac 1{N+1}$. On appelle nœuds les points de coordonnées $x_i = ih$ où $0\leq i\leq N+1$. L'objectif est de calculer une approximation $u_i$ des valeurs de la solution du problème aux points $x_i$. Il y a donc $N$ valeurs à calculer $u(x_i)$;  $1\leq i\leq N$. 

\begin{enumerate}
\item La solution $u$ étant de classe $C^4$, en faisant un développement de Taylor à l'ordre 4 de $u$, montrer que l'approximation 
\[u''(x)\simeq \frac{u(x+h)-2u(x)+u(x-h)}{h^2}\] 
 est  d'ordre 2. 

\item En déduire que la suite $(u(x_i))$ pour $1\leq i\leq N$ peut être approchée par une suite $(u_i)$ vérifiant la relation de récurrence:
\[\left\{\begin{array}{l}
\displaystyle -\frac{u_{i-1}-2u_i+u_{i+1}}{h^2}+c\,u_i =f_i \quad1\leq i\leq N\\
u_0=u_{N+1}=0
\end{array}\right.
\]
où  $f_i=f(x_i)$.
\item On pose $U=\left(\begin{array}{c} u_1\\ u_2 \\ \vdots \\ u_N \end{array}\right)$ et $b=\left(\begin{array}{c} f_1\\ f_2 \\ \vdots \\ f_N \end{array}\right)$. Montrer que le problème approché se ramène à une résolution d'un système linéaire $A U = b$ où $A$ est une matrice tridiagonale que l'on déterminera.
\item Montrer que $A$ est une matrice symétrique définie positive.

\end{enumerate}

\section{Problème aux limites d'ordre 4}
%%%%%%%%%%%%%%%%%%%%%%%%%%%%%%%%%%%%%%%%%%%%%%%%%%%%%%%%%%%%%%%%%%%%%%%%%%%%%%%%
 On considère le problème aux limites d'ordre 4 suivant
\[\left\{\begin{array}{l}
\frac{d^4u(x)}{dx^4}=f(x)\quad\mbox{dans }]0,1[\\
u(0)=u(1)=0\\
u'(0)=u'(1)=0\\
\end{array}\right.
\]
où $f(x)$ une fonction continue. Cette équation modélise de nombreux problèmes de statique, par exemple une poutre en flexion simple.

\begin{enumerate}
\item On subdivise l'intervalle $[0; 1]$ en $N+3$ points d'abscisses $x_i = ih$ où $0\leq i\leq N+2$ et $h = \frac 1{N+2}$.  L'objectif est de calculer une approximation $(u_i)$ des valeurs de la solution du problème aux points $x_i$. Montrer qu'il y a  $N-1$ valeurs à calculer $u(x_i)$;  $2\leq i\leq N$. 

\item On définit l'opérateur linéaire $T_h$ par $T_hf(x)=f(x+h)$ et on approche alors la dérivée par l'opérateur $\Delta$:
\[\Delta = \frac{T_{\frac h2}-T_{-\frac h2}}{h}\]
Justifier rapidement l'approximation $\Delta u\simeq u'$ et calculer $\Delta^4$. En déduire une approximation de $\frac{\de^4 u}{\de x^4}(x_i)$.Montrer que cette dernière approximation est  d'ordre 2. 

\item En déduire que la suite $(u(x_i))$ pour $2\leq i\leq N$ peut être approchée par une suite $(u_i)$ vérifiant la relation de récurrence d'ordre 4:
\[\left\{\begin{array}{l}
\displaystyle \frac{u_{i-2}-4u_{i-1}+6u_i-4u_{i+1}+u_{i+2}}{h^4}=f_i \quad 2\leq i\leq N\\
u_0=u_{N+2}=0\\
u_1=u_{N+1}=0
\end{array}\right.
\]
où  $f_i=f(x_i)$.
\item On pose $U=\left(\begin{array}{c}  u_2 \\ \vdots \\ u_N \end{array}\right)$ et $b=\left(\begin{array}{c} f_2 \\ \vdots \\ f_N \end{array}\right)$. Montrer que le problème approché se ramène à une résolution d'un système linéaire $A U = b$ où $A$ est une matrice pentadiagonale que l'on déterminera.

\end{enumerate}

\section{Équation de la chaleur}
%%%%%%%%%%%%%%%%%%%%%%%%%%%%%%%%%%%%%%%%%%%%%%%%%%%%%%%%%%%%%%%%%%%%%%%%%%%%%%%%
 On considère l'équation aux dérivées partielles de la chaleur en dimension 1   suivante
\[\left\{\begin{array}{ll}
\displaystyle \frac{\partial u(x,t)}{\partial t} -\nu \frac{\partial^2u(x,t)}{\partial x^2}=0 & \forall (x,t)\in ]0,1[\times \mathbb{R}_+^*\\
u(0,t)=u(1,t)=0&\\
u(x,0)=u_0(x) & \forall x \in [0,1]
\end{array}\right.
\]
 On subdivise l'intervalle $[0; 1]$ en $N+2$ points d'abscisses $x_i = ih$ où $0\leq i\leq N+1$ avec $h = \frac 1{N+1}$ et soit $\tau$ le pas de temps.  On note $u^n_i$ la valeur approchée de $u(x_i,t_n)$ avec $t_n=n\tau$ et on considère le schéma dit d'Euler implicite qui est le suivant:
 \[ \frac{u_{i}^{n+1}-u_{i}^n}{\tau}-\nu \frac{u_{i-1}^{n+1}-2u_i^{n+1}+u_{i+1}^{n+1}}{h^2}=0
\]
  

\begin{enumerate}
\item Montrer que ce schéma est précis d'ordre 1 en temps et d'ordre 2 en espace.

\item On pose $u^n=\left(\begin{array}{c}  u_1^n  \\ u_2^n \\\vdots \\ u_N^n \end{array}\right)$. Montrer que le schéma se traduit sous la forme matricielle:
\[\forall n\in \mathbb{N},\quad M u^{n+1} = u^n\]

\item Expliquer rapidement comment on programmerait l'algorithme.
\item Soit A la matrice
\[A=
\left(\begin{array}{ccccc}
2&-1&0&\cdots&0\\
-1&2&-1&\ddots&\vdots\\
0&  \ddots &\ddots&\ddots&0\\
\vdots &\ddots &-1&2&-1\\
   0&\cdots &0&-1 &2
\end{array}\right)
\] 
Montrer que si $\lambda$ est valeur propre de $A$ alors $0<\lambda<4$. En déduire que le rayon spectral de $M^{-1}$ vérifie $\rho(M^{-1})\leq 1$.
\item On dit que le schéma est stable ssi pour tout entier $n$, $\|u^n\|\leq \|u^0\|$. Montrer que 
\[\|u^{n+1}\|\leq \|u^n\|\]
En déduire que le schéma est stable.
\end{enumerate}



%%%%%%%%%%%%%%%%%%%%%%%%%%%%%%%%%%%%%%%%%%%%%%%%%%%%%%%%%%%%%%%%%%%%%%%%%%%%%%%%%%%%%%%%%%%%%
\section{Problème de Dirichlet}
\begin{enumerate}
\item En dimension deux, soit le domaine  $D = [0, a] \times [0, b]\subset \mathbb{R}^2$, on désigne par $\partial D$ le bord de $D$. On considère le
problème
\[\left\{\begin{array}{l}
-\Delta u = f \mbox{ dans }D\\
u =0 \mbox{ sur }\partial D
\end{array}\right.
\]
La discrétisation par différences finies sur un quadrillage uniforme de pas $\delta x = \frac{a}{N+1}$ et $\delta y = \frac{b}{M+1}$
est : 

\begin{equation}
\begin{array}{l}
\displaystyle \frac {-u_{i-1,j} + 2u_{i,j}-u_{i+1,j}}{\delta x^2} +
\frac {-u_{i,j-1} + 2u_{i,j}-u_{i,j+1}}{\delta y^2}= f_{i,j} ;
 \qquad 1 \leq i  \leq N, 1 \leq j  \leq M \\
u_{i,0} = u_{i,M+1} = u_{0,j} = u_{N+1,j}=0.
\end{array}
\end{equation}

\item On note $u_{i,j}=u_m$ avec $m = i + (j-1)N$. Soit le vecteur $U = (u_m)$.  Pour $N = M = 3$, écrire le système
sous la forme $AU = F$. Montrer $A$ est inversible et définie positive.
\item Vérifier que les vecteurs propres de la matrice sont les vecteurs $u(p,q)$ définis par
\[(u(p,q))_{i,j} = \sin\left(p\pi \frac{i\delta x}{a}\right)\sin\left(q\pi \frac{j\delta y}{b}\right)
\]

Les valeurs propres associées sont
\[\lambda_{p,q} =\frac 4{\delta x^2} \sin^2\left(p\pi \frac{\delta x}{2a}\right)+\frac 4{\delta y^2} \sin^2\left(q\pi \frac{\delta y}{2b}\right)\]

\item Pour $N$ et $M$ assez grands, calculer une valeur approchée de $\mbox{Cond}_2A$.
\item On suppose $M=N$, et $a=b$ calculer un équivalent de  $\mbox{Cond}_2A$ quant $N\to \infty$.

\end{enumerate}

%%%%%%%%%%%%%%%%%%%%%%%%%%%%%%%%%%%%%%%%%%%%%%%%%%%%%%%%%%%%%%%%%%%%%%%%%%%%%%%%%%%%%%%%%%%%%
\section{Problème de Dirichlet}
\begin{enumerate}
\item En dimension deux, soit le domaine  $D = [0, a] \times [0, b]\subset \mathbb{R}^2$, on désigne par $\partial D$ le bord de $D$. On considère le
problème
\[\left\{\begin{array}{l}
-\Delta u + u= f \mbox{ dans }D\\
u =0 \mbox{ sur }\partial D
\end{array}\right.
\]
La discrétisation par différences finies sur un quadrillage uniforme de pas $\delta x = \frac{a}{N+1}$ et $\delta y = \frac{b}{M+1}$
est : 

\begin{equation}
\begin{array}{l}
\displaystyle \frac {-u_{i-1,j} + 2u_{i,j}-u_{i+1,j}}{\delta x^2} +
\frac {-u_{i,j-1} + 2u_{i,j}-u_{i,j+1}}{\delta y^2}+u_{i,j}= f_{i,j} ;
 \qquad 1 \leq i  \leq N, 1 \leq j  \leq M \\
u_{i,0} = u_{i,M+1} = u_{0,j} = u_{N+1,j}=0.
\end{array}
\end{equation}

\item On note $u_{i,j}=u_m$ avec $m = i + (j-1)N$. Soit le vecteur $U = (u_m)$  Pour  $N = M = 4$, écrire le système
sous la forme $AU = F$. Montrer $A$ est inversible et définie positive.
\item Vérifier que les vecteurs propres de la matrice sont les vecteurs $u(p,q)$ définis par
\[(u(p,q))_{i,j} = \sin\left(p\pi \frac{i\delta x}{a}\right)\sin\left(q\pi \frac{j\delta y}{b}\right)
\]

Déterminer les valeurs propres associées.
\item Pour $N$ et $M$ assez grands, calculer une valeur approchée de $\mbox{Cond}_2A$.
\item On suppose $M=N$, et $a=b$ calculer un équivalent de  $\mbox{Cond}_2A$ quant $N\to \infty$.

\item Donner la matrice $A$ associée au graphe suivant :

\begin{center}

 \begin{tikzpicture}[scale=1]
\draw  [ gray] [->]  (-0.2,0) -- (6.2,0);
 \draw  [very thin, gray] [-]  (-0.2,1) -- (5.2,1);
  \draw  [very thin, gray] [-]  (-0.2,2) -- (5.2,2);
  \draw  [very thin, gray] [-]  (-0.2,3) -- (5.2,3);
  \draw  [very thin, gray] [-]  (-0.2,4) -- (5.2,4);
   \draw  [very thin, gray] [-]  (-0.2,5) -- (5.2,5);
%      \draw  [very thin, gray] [-]  (-0.2,6) -- (6.2,6);
\draw  [very thin, gray] [->] (0,-0.2) -- (0,5.2);
\draw  [very thin, gray] [-] (1,-0.2) -- (1,5.2);
\draw  [very thin, gray] [-] (2,-0.2) -- (2,5.2);
\draw  [very thin, gray] [-] (3,-0.2) -- (3,5.2);
\draw  [very thin, gray] [-] (4,-0.2) -- (4,5.2);
\draw  [very thin, gray] [-] (5,-0.2) -- (5,5.2);
%\draw  [very thin, gray] [-] (6,-0.2) -- (6,6.2);

\node at (0.8,1.2) {1};
\node at (0.8,2.2) {2};
\node at (1.8,1.2) {3};

\node at (0.8,3.2) {4};
\node at (1.8,2.2) {5};
\node at (2.8,1.2) {6};

\node at (0.8,4.2) {7};
\node at (1.8,3.2) {8};
\node at (2.8,2.2) {9};
\node at (3.8,1.2) {10};

\node at (1.8,4.2) {11};
\node at (2.8,3.2) {12};
\node at (3.8,2.2) {13};

\node at (2.8,4.2) {14};
\node at (3.8,3.2) {15};
\node at (3.8,4.2) {16};


\end{tikzpicture} 


\end{center}


\end{enumerate}















%%%%%%%%%%%%%%%%%%%%%%%%%%%%%%%%%%%%%%%%%%%%%%%%%%%%%%%%%%%%%%%%%%%%%%%%%%%%%%%



\end{document}