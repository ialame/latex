\documentclass[a4paper,11pt]{article} 
\usepackage[francais]{babel}
\usepackage[utf8]{inputenc} % Required for including letters with accents
\usepackage[T1]{fontenc} % Use 8-bit encoding that has 256 glyphs

\usepackage{amsthm}
\usepackage{amsmath}
\usepackage{amssymb}
\usepackage{mathrsfs}
\usepackage{graphicx}
\usepackage{geometry}
\usepackage{stmaryrd}
\usepackage{tikz}

\def \de {{\rm d}}
\usepackage{algorithm2e}
\RestyleAlgo{algoruled}
  \SetKw{KwFrom}{from} 
\newenvironment{algo}{
\begin{algorithm}[H]
\DontPrintSemicolon \SetAlgoVlined}
{\end{algorithm}}
\usepackage{color}
%\usepackage{xcolor}
%\usepackage{textcomp}
\newcommand{\mybox}[1]{\fbox{$\displaystyle#1$}}
\newcommand{\myredbox}[1]{\fcolorbox{red}{white}{$\displaystyle#1$}}
\newcommand{\mydoublebox}[1]{\fbox{\fbox{$\displaystyle#1$}}}
\newcommand{\myreddoublebox}[1]{\fcolorbox{red}{white}{\fcolorbox{red}{white}{$\displaystyle#1$}}}
%%%%%%%%   bleclercq@adm.estp.fr
\title{TD4 d'Analyse Numérique (B1-TP1) }
\author{Ibrahim ALAME}
\date{19/03/2024}
\begin{document}
\maketitle

\section{Méthode des différences finies}
\begin{enumerate}
\item Si $u$ est de classe $C^4$ au voisinage de $x$, on fait un développement de Taylor à l'ordre 4 de $u$:
\[u(x+h)=u(x)+h u'(x)+\frac{h^2}2 u''(x)+\frac{h^3}6 u'''(x)+\frac{h^4}{24} u^{(4)}(\xi_1) ,\]
\[u(x-h)=u(x)-h u'(x)+\frac{h^2}2 u''(x)-\frac{h^3}6 u'''(x)+\frac{h^4}{24} u^{(4)}(\xi_2) ,\]
Donc
\[u(x+h)+u(x-h)=2u(x)+h^2 u''(x)+\frac{h^4}{24} \left(u^{(4)}(\xi_1) + u^{(4)}(\xi_2)\right) ,\]
D'où
\[\left| \frac{u(x+h)-2u(x)+u(x-h)}{h^2}- u''(x) \right|\leq \frac{h^2}{12}\sup_{y\in[0,1]} |u^{(4)}(y)| \]
pour tout $x \in [h, 1 - h]$. L'approximation 

 \[\myredbox{u''(x)\simeq \frac{u(x+h)-2u(x)+u(x-h)}{h^2}}\] 
 est donc consistante d'ordre 2. 
\item Approximation de la dérivée seconde:
\[u''(x_i)\simeq \frac{u(x_i+h)-2u(x_i)+u(x_i-h)}{h^2}\simeq \frac{u_{i+1}-2u_i+u_{i-1}}{h^2}\] 


D'où le problème approché
\begin{equation}
\left\{\begin{array}{l}
-{\displaystyle \frac{u_{i-1}-2u_i+u_{i+1}}{h^2}+c u_i }= f_i, \mbox { pour } i\in\{1, \cdots , N\},\\
u_0=0,\quad u_{N+1}=0
\end{array}
\right.
\end{equation}
où $c_i=c(x_i)$ et $f_i=f(x_i)$.

On obtient $N$ équations à $N$ inconnues $u_1, \cdots , u_N$ . 

\[\begin{array}{lllll}
i=1:\quad -{\displaystyle \frac{u_{0}-2u_1+u_{2}}{h^2} }&+&c u_1&=& f_1, \\
i=2:\quad -{\displaystyle \frac{u_{1}-2u_2+u_{3}}{h^2} }&+&c u_2 &= &f_2, \\
\cdots&&&&\\
%i=N-1:\quad -{\displaystyle \frac{u_{N-2}-2u_{N-1}+u_{N}}{h^2} }+c u_{N-1} = f_{N-1}, \\
i=N:\quad -{\displaystyle \frac{u_{N-1}-2u_{N}+u_{N+1}}{h^2} }&+&c u_{N} &=& f_{N},

\end{array}
\]
D'où le système
\[\myredbox{\left\{\begin{array}{lllll}
-{\displaystyle \frac{-2u_1+u_{2}}{h^2}}&+&c u_1 &= &f_1, \\
-{\displaystyle \frac{u_{1}-2u_2+u_{3}}{h^2}}&+&c u_2 &=& f_2, \\
\cdots&\cdots&\cdots&\cdots\\
-{\displaystyle \frac{u_{N-1}-2u_{N}}{h^2}}&+&c u_{N} &=& f_{N},

\end{array}\right.}
\]
\item  Matriciellement, le problème s'écrit : 
 
 \begin{equation}
A U =b
 \end{equation}
 
  avec
 
 
 
 \[A=\frac{1}{h^2}
\left(\begin{array}{ccccc}
2&-1&0&\cdots&0\\
-1&2&-1&\ddots&\vdots\\
0&  \ddots &\ddots&\ddots&0\\
\vdots &\ddots &-1&2&-1\\
   0&\cdots &0&-1 &2
\end{array}\right)+c I_n
\] 
\item Soit \[A_0=
\left(\begin{array}{ccccc}
2&-1&0&\cdots&0\\
-1&2&-1&\ddots&\vdots\\
0&  \ddots &\ddots&\ddots&0\\
\vdots &\ddots &-1&2&-1\\
   0&\cdots &0&-1 &2
\end{array}\right)
\] 
On 
\[<A_0x\,,\, x> = 2\sum_{i=1}^nx_i^2-2\sum_{i=1}^{n-1}x_i x_{i+1}=x_1^2+\sum_{i=1}^{n-1}(x_i-x_{i+1})^2+x_n^2\]
On a alors $<A_0x\,,\, x>\;\geq 0$ et 
\[<A_0x\,,\, x>\;= 0\Longrightarrow x_1^2=x_n^2=0 \mbox{ et }(x_i-x_{i+1})^2=0 \;\forall i=1,n-1\]
Donc $\Longrightarrow x=0$ et $A_0$ définie positive donc $A$ est définie positive.

\end{enumerate}

\section{Problème aux limites d'ordre 4}
%%%%%%%%%%%%%%%%%%%%%%%%%%%%%%%%%%%%%%%%%%%%%%%%%%%%%%%%%%%%%%%%%%%%%%%%%%%%%%%%

\[\left\{\begin{array}{l}
\frac{d^4u(x)}{dx^4}=f(x)\quad\mbox{dans }]0,1[\\
u(0)=u(1)=0\\
u'(0)=u'(1)=0\\
\end{array}\right.
\]

\begin{enumerate}
\item On a $0=u'(0)\simeq \frac{u(h)-u(0)}{h}\Longrightarrow u_1=u(h)=0$ de même $u_{N+1}=0$. Nous avons donc 4 valeurs de $u$ connues, il reste $N-1$ valeurs de $(u_i)$ à déterminer pour $i$ allant de 2 à $N$. 

\item On a
\[\Delta u(x)= \frac{T_{\frac h2}u(x)-T_{-\frac h2}u(x)}{h}= \frac{u(x+\frac h2)-u(x-\frac h2)}{h}\simeq u'(x)\quad\mbox{ (cf cours)}\]
On a
\[\Delta^4=\left(\frac 1h(T_{h/2}-T_{-h/2})\right)^4\]\[=\frac 1{h^4}(T^4_{h/2}-4T^3_{h/2}T_{-h/2}+6T^2_{h/2}T^2_{-h/2}-4T_{h/2}T^3_{-h/2}+T^4_{-h/2})\]
\[=\frac 1{h^4}(T_{2h}-4T_{3h/2}T_{-h/2}+6T_{h}T_{-h}-4T_{h/2}T_{-3h/2}+T_{-2h})\]
\[=\frac 1{h^4}(T_{2h}-4T_{h}+6I-4T_{-h}+T_{-2h})\]
\[\Delta^4u(x_i)=\frac 1{h^4}(u(x_i+2h)-4u(x_i+h)+6u(x_i)-4u(x_i-h)+u(x_i-2h))\]
\[\Delta^4u(x_i)=\frac 1{h^4}(u(x_{i+2})-4u(x_{i+1})+6u(x_i)-4u(x_{i-1})+u(x_{i-2}))\]
\[\myredbox{\frac{\de^4 u_i}{\de x^4}=\frac 1{h^4}(u_{i+2}-4u_{i+1}+6u_i-4u_{i-1}+u_{i-2})}\]


Cette approximation est  d'ordre 2: On fait un développement de Taylor à l'ordre 6 de $u$:
\[u(x+h)=u(x)+h u'(x)+\frac{h^2}2 u''(x)+\frac{h^3}6 u'''(x)+\frac{h^4}{24} u^{(4)}(x) +\frac{h^5}{120} u^{(5)}(x) +\frac{h^6}{720} u^{(6)}(x)+o(h^6)  ,\]
\[u(x-h)=u(x)-h u'(x)+\frac{h^2}2 u''(x)-\frac{h^3}6 u'''(x)+\frac{h^4}{24} u^{(4)}(x) -\frac{h^5}{120} u^{(5)}(x) +\frac{h^6}{720} u^{(6)}(x) +o(h^6) ,\]
\[u(x+2h)=u(x)+2h u'(x)+2h^2 u''(x)+\frac{4h^3}3 u'''(x)+\frac{2h^4}{3} u^{(4)}(x) +\frac{4h^5}{15} u^{(5)}(x) +\frac{4h^6}{45} u^{(6)}(x) +o(h^6) ,\]
\[u(x-2h)=u(x)-2h u'(x)+2h^2 u''(x)-\frac{4h^3}3 u'''(x)+\frac{2h^4}{3} u^{(4)}(x) -\frac{4h^5}{15} u^{(5)}(x) +\frac{4h^6}{45} u^{(6)}(x) +o(h^6) ,\]
En faisant une combinaison des 4 dernières lignes; on trouve:
\[-4u(x+h)-4u(x-h)+u(x+h)+u(x-h)=6u(x)+h^4 u^{(4)}(x)+\frac{h^6}{6} u^{(6)}(x) +o(h^6) ,\]
D'où
\[ \frac{u(x-2h)-4u(x-h)+6u(x)-4u(x+h)+u(x+2h)}{h^4}= u^{(4)}(x) +O(h^2)\]
\item 
\[\left\{\begin{array}{l}
\displaystyle \frac{u_{i-2}-4u_{i-1}+6u_i-4u_{i+1}+u_{i+2}}{h^4}=f_i \quad 2\leq i\leq N\\
u_0=u_{N+2}=0\\
u_1=u_{N+1}=0
\end{array}\right.
\]
où  $f_i=f(x_i)$.
\item On pose $U=\left(\begin{array}{c}  u_2 \\ \vdots \\ u_N \end{array}\right)$ et $b=\left(\begin{array}{c} f_2 \\ \vdots \\ f_N \end{array}\right)$. Le problème approché se ramène à une résolution d'un système linéaire $A U = b$ où $A$ est la matrice pentadiagonale:
Soit \[A=
\left(\begin{array}{ccccccc}
6&-4&1&0&0&\cdots&0\\
-4&6&-4&1&0&\cdots&0\\
1&-4&6&-4&1& \ddots  &\vdots\\
0& 1 &\ddots &\ddots &\ddots&\ddots&0 \\
\vdots &  \ddots &\ddots &\ddots &\ddots&\ddots&1\\
0 &\cdots &0&1 &-4&6&-4\\
   0&\cdots&0  &0&1&-4  &6
\end{array}\right)
\] 
\end{enumerate}

\section{Équation de la chaleur}
%%%%%%%%%%%%%%%%%%%%%%%%%%%%%%%%%%%%%%%%%%%%%%%%%%%%%%%%%%%%%%%%%%%%%%%%%%%%%%%%
\[\left\{\begin{array}{ll}
\displaystyle \frac{\partial u(x,t)}{\partial t} -\nu \frac{\partial^2u(x,t)}{\partial x^2}=0 & \forall (x,t)\in ]0,1[\times \mathbb{R}_+^*\\
u(0,t)=u(1,t)=0&\\
u(x,0)=u_0(x) & \forall x \in [0,1]
\end{array}\right.
\]
 On subdivise l'intervalle $[0; 1]$ en $N+2$ points d'abscisses $x_i = ih$ où $0\leq i\leq N+1$ avec $h = \frac 1{N+1}$ et soit $\tau$ le pas de temps.  On note $u^n_i$ la valeur approchée de $u(x_i,t_n)$ avec $t_n=n\tau$ et on considère le schéma dit d'Euler implicite qui est le suivant:
 \[ \frac{u_{i}^{n+1}-u_{i}^n}{\tau}-\nu \frac{u_{i-1}^{n+1}-2u_i^{n+1}+u_{i+1}^{n+1}}{h^2}=0
\]
  

\begin{enumerate}
\item Posons $t=t_{n+1}$ et $x=x_i$ pour alléger les notations.
Pour déterminer l'erreur de troncature, on va dans le schéma remplacer $u_i^n$ par $u(x,t-\tau)$, $u_{i-1}^{n+1}$ par $u(x-h,t)$, $u_i^{n+1}$ par $u(x,t)$ et $u_{i+1}^{n+1}$ par $u(x+h,t)$. A l'aide de la formule de Taylor à l'ordre 2 de $u$ suivant $t$, on a
\[u(x,t-\tau)=u(x,t)-\tau\frac{\partial u}{\partial t}+\frac{\tau^2}{2}\frac{\partial^2u}{\partial t^2}+o(\tau^2)\]
D'où
\[\frac{u(x,t) - u(x,t-\tau)}{\tau}  =\frac{\partial u}{\partial t}-\frac{\tau}{2}\frac{\partial^2u}{\partial t^2}+o(\tau)\]
Par ailleurs, en considérant la variation de $x$, on a
\[u(x+h,t)=u(x,t)+h\frac{\partial u}{\partial x}+\frac{h^2}{2}\frac{\partial^2u}{\partial x^2}+\frac{h^3}{6}\frac{\partial^3u}{\partial x^3}+\frac{h^4}{24}\frac{\partial^4u}{\partial x^4}+o(h^4)\]
et de la même façon
\[u(x-h,t)=u(x,t)-h\frac{\partial u}{\partial x}+\frac{h^2}{2}\frac{\partial^2u}{\partial x^2}-\frac{h^3}{6}\frac{\partial^3u}{\partial x^3}+\frac{h^4}{24}\frac{\partial^4u}{\partial x^4}+o(h^4)\]
Donc
\[\frac{u(x-h,t) - 2u(x,t)+u(x+h,t)}{h^2}  =\frac{\partial^2 u}{\partial x^2}+\frac{h^2}{12}\frac{\partial^4u}{\partial x^4}+o(h^2)\]
En remplaçant $u_i^n$ dans le schéma par $u(x_i,t_n)$, on se retrouve avec l'erreur de troncature:
\[\frac{\partial u}{\partial t}-\frac{\tau}{2}\frac{\partial^2u}{\partial t^2}+o(\tau) -\nu\left(\frac{\partial^2 u}{\partial x^2}+\frac{h^2}{12}\frac{\partial^4u}{\partial x^4}+o(h^2)\right)=-\frac{\tau}{2}\frac{\partial^2u}{\partial t^2}-\nu \frac{h^2}{12}\frac{\partial^4u}{\partial x^4} +o(\tau+h^2)\]
qui converge bien vers 0 pour $h\to 0$, $\tau\to 0$. Le schéma est bien consistant, et précis d'ordre 1 en temps et d'ordre 2 en espace.
\item On pose $u^n=\left(\begin{array}{c}  u_1^n  \\ u_2^n \\\vdots \\ u_N^n \end{array}\right)$. Dans le schéma, isolons $u_i^n$, en posant $c=\frac{\nu\tau}{h^2}$:
\[u_i^n=-c u_{i-1}^{n+1}+(1+2c)u_i^{n+1}-cu_{i+1}^{n+1}\]
Précisons les équations, en notant leur forme particulières pour $i=1$ et $i=N$ lorsque l'on prend en compte les conditions aux limites:
\[\begin{array}{rcl}
u_1^n & = & (1+2c)u_1^{n+1}-cu_2^{n+1}\\
u_2^n & = & -cu_1^{n+1}+(1+2c)u_2^{n+1}-cu_3^{n+1}\\
u_3^n & = & -cu_2^{n+1}+(1+2c)u_3^{n+1}-cu_4^{n+1}\\
\vdots &\vdots &\vdots \\
u_{N-1}^n & = & -cu_{N-2}^{n+1}+(1+2c)u_{N-1}^{n+1}-cu_{N}^{n+1}\\
u_N^n & = & -cu_{N-1}^{n+1}+(1+2c)u_N^{n+1}
\end{array}\]
On a bien $ M u^{n+1} = u^n$ avec
\[M=
\left(\begin{array}{ccccc}
1+2c&-c&0&\cdots&0\\
-c&1+2c&-c&\ddots&\vdots\\
0&  \ddots &\ddots&\ddots&0\\
\vdots &\ddots &-c&1+2c&-c\\
   0&\cdots &0&-c &1+2c
\end{array}\right)
\] 

\item Expliquer rapidement comment on programmerait l'algorithme:

$u^0$ étant connu par la condition initiale, on calcule de proche en proche les $u^n$
successifs. $u^1$ est la solution du système $Mu^1=u^0$. Puis $u^2$ est la solution du système $Mu^2=u^1$ et ainsi de suite...

Il faut donc résoudre successivement des systèmes décrits par $Mu^{n+1}=u^n$ 
\item On a
\[A=
\left(\begin{array}{ccccc}
2&-1&0&\cdots&0\\
-1&2&-1&\ddots&\vdots\\
0&  \ddots &\ddots&\ddots&0\\
\vdots &\ddots &-1&2&-1\\
   0&\cdots &0&-1 &2
\end{array}\right)
\] 
Soit $\lambda$ une valeur propre de $A$ et $v$ un vecteur propre associé. On développe la composante $i_0$ de l'égalité matricielle: $Av=\lambda v$ où $i_0$ est un indice tel que $v_{i_0}=max_{i}|v_i|$:

\[-v_{i_0-1}+2v_{i_0}-v_{i_0+1}=\lambda v_{i_0}\]
\[(2-\lambda)v_{i_0} = v_{i_0-1}+v_{i_0+1}\]
\[|2-\lambda|\times |v_{i_0}| \leq |v_{i_0-1}|+|v_{i_0+1}|\]
\[|2-\lambda| \leq 2\]
\[0<\lambda<4\]
Comme $M$ s'écrit $M=I+cA$ et que toutes les valeurs de $A$ sont entre 0 et 4, il s'avère que $M$ a ses valeurs propres comprises entre 1 et $1+4c>1$ donc $M^{-1}$ a ses valeurs propres comprises entre $\frac{1}{1+4c}$ et 1. Comme $M^{-1}$ est symétrique définie positive, sa norme 2 est la plus grande valeur propre (cf cours) et donc $\|M^{-1}\|\leq 1$. 
\item On a $u^{n+1}=M^{-1}u^n$ donc $\|u^{n+1}\|\leq \|M^{-1}\| \|u^n\|\leq \|u^n\|$ et par récurrence $\|u^n\|\leq \|u^0\|$. Donc le schéma est stable.
\end{enumerate}



\section{Problème de Dirichlet}
\begin{enumerate}
\item
\item Le système avec $u$ en deux indices:
\begin{center}
 \begin{tikzpicture}[domain=0:5]
  \draw[->] (-0.1,0) -- (4.5,0)  node[right] {$\scriptstyle x$};
  \draw[->] (0,-0.1) -- (0,4.3) node[above] {$\scriptstyle y$};
  \path[fill=black]  (0,0) circle (.4mm) [fill=orange] node[below] {$\scriptstyle O$};


   \path[fill=black]  (4,0) circle (.4mm) [fill=orange] node[below] {$\scriptstyle A$};

   \path[fill=black]  (0,4) circle (.4mm) [fill=orange] node[left] {$\scriptstyle C$};
      \path[fill=black]  (4,4) circle (.4mm) [fill=orange] node[above right] {$\scriptstyle B$};
\draw[dotted,blue] (1,0) -- ++(0,4) ;
\draw[dotted,blue] (2,0) -- ++(0,4) ;
\draw[dotted,blue] (3,0) -- ++(0,4) ;
\draw[dotted,blue] (0,1) -- ++(4,0) ;
\draw[dotted,blue] (0,2) -- ++(4,0) ;
\draw[dotted,blue] (0,3) -- ++(4,0) ;
\draw[orange] (0.02,0.02) -- ++(4,0)  -- ++(0,4) -- ++(-4,0)  -- ++(0,-4)  ;
 \path[fill=red]  (1,1) circle (.5mm)node[below] {$\scriptstyle (1,1)$};
 \path[fill=red]  (2,1) circle (.5mm) node[below] {$\scriptstyle (2,1)$};
 \path[fill=red]  (3,1) circle (.5mm)node[below] {$\scriptstyle (3,1)$};
  \path[fill=red]  (1,2) circle (.5mm) node[below] {$\scriptstyle  (1,2)$};
  \path[fill=red]  (2,2) circle (.5mm)  node[ below] {$\scriptstyle (2,2)$};
   \path[fill=red]  (3,2) circle (.5mm)  node[ below] {$\scriptstyle (3,2) $};
   \path[fill=red]  (1,3) circle (.5mm) node[below] {$\scriptstyle (1,3)$};
  \path[fill=red]  (2,3) circle (.5mm)  node[ below] {$\scriptstyle (2,3)$};
   \path[fill=red]  (3,3) circle (.5mm)  node[ below] {$\scriptstyle (3,3)$};
  
\end{tikzpicture}
 \end{center}
 
\[\begin{array}{l}
i=1, j=1:\quad {\displaystyle -\frac{u_{0,1}-2u_{1,1}+u_{2,1}}{\delta x^2}-\frac{u_{1,0}-2u_{1,1}+u_{1,2}}{\delta y^2} }=f_{1,1} \\
i=2, j=1:\quad {\displaystyle -\frac{u_{1,1}-2u_{2,1}+u_{3,1}}{\delta x^2}-\frac{u_{2,0}-2u_{2,1}+u_{2,2}}{\delta y^2} }=f_{2,1} \\
i=3, j=1:\quad {\displaystyle -\frac{u_{2,1}-2u_{3,1}+u_{4,1}}{\delta x^2}-\frac{u_{3,0}-2u_{3,1}+u_{3,2}}{\delta y^2} }=f_{3,1} \\
i=1, j=2:\quad {\displaystyle -\frac{u_{0,2}-2u_{1,2}+u_{2,2}}{\delta x^2}-\frac{u_{1,1}-2u_{1,2}+u_{1,3}}{\delta y^2} }=f_{1,2} \\
i=2, j=2:\quad {\displaystyle -\frac{u_{1,2}-2u_{2,2}+u_{3,2}}{\delta x^2}-\frac{u_{2,1}-2u_{2,2}+u_{2,3}}{\delta y^2} }=f_{2,2} \\
i=3, j=2:\quad {\displaystyle -\frac{u_{2,2}-2u_{3,2}+u_{4,2}}{\delta x^2}-\frac{u_{3,1}-2u_{3,2}+u_{3,3}}{\delta y^2} }=f_{3,2} \\
i=1, j=3:\quad {\displaystyle -\frac{u_{0,3}-2u_{1,3}+u_{2,3}}{\delta x^2}-\frac{u_{1,2}-2u_{1,3}+u_{1,4}}{\delta y^2} }=f_{1,3} \\
i=2, j=3:\quad {\displaystyle -\frac{u_{1,3}-2u_{2,3}+u_{3,3}}{\delta x^2}-\frac{u_{2,2}-2u_{2,3}+u_{2,4}}{\delta y^2} }=f_{2,3} \\
i=3, j=3:\quad {\displaystyle -\frac{u_{2,3}-2u_{3,3}+u_{4,3}}{\delta x^2}-\frac{u_{3,2}-2u_{3,3}+u_{3,4}}{\delta y^2} }=f_{3,3} \\
\end{array}
\]
Avec les conditions aux bords:
\[\begin{array}{l}
i=1, j=1:\quad {\displaystyle -\frac{0-2u_{1,1}+u_{2,1}}{\delta x^2}-\frac{0-2u_{1,1}+u_{1,2}}{\delta y^2} }=f_{1,1} \\
i=2, j=1:\quad {\displaystyle -\frac{u_{1,1}-2u_{2,1}+u_{3,1}}{\delta x^2}-\frac{0-2u_{2,1}+u_{2,2}}{\delta y^2} }=f_{2,1} \\
i=3, j=1:\quad {\displaystyle -\frac{u_{2,1}-2u_{3,1}+0}{\delta x^2}-\frac{0-2u_{3,1}+u_{3,2}}{\delta y^2} }=f_{3,1} \\
i=1, j=2:\quad {\displaystyle -\frac{0-2u_{1,2}+u_{2,2}}{\delta x^2}-\frac{u_{1,1}-2u_{1,2}+u_{1,3}}{\delta y^2} }=f_{1,2} \\
i=2, j=2:\quad {\displaystyle -\frac{u_{1,2}-2u_{2,2}+u_{3,2}}{\delta x^2}-\frac{u_{2,1}-2u_{2,2}+u_{2,3}}{\delta y^2} }=f_{2,2} \\
i=3, j=2:\quad {\displaystyle -\frac{u_{2,2}-2u_{3,2}+0}{\delta x^2}-\frac{u_{3,1}-2u_{3,2}+u_{3,3}}{\delta y^2} }=f_{3,2} \\
i=1, j=3:\quad {\displaystyle -\frac{0-2u_{1,3}+u_{2,3}}{\delta x^2}-\frac{u_{1,2}-2u_{1,3}+0}{\delta y^2} }=f_{1,3} \\
i=2, j=3:\quad {\displaystyle -\frac{u_{1,3}-2u_{2,3}+u_{3,3}}{\delta x^2}-\frac{u_{2,2}-2u_{2,3}+0}{\delta y^2} }=f_{2,3} \\
i=3, j=3:\quad {\displaystyle -\frac{u_{2,3}-2u_{3,3}+0}{\delta x^2}-\frac{u_{3,2}-2u_{3,3}+0}{\delta y^2} }=f_{3,3} \\
\end{array}
\]

Le système avec $u$ en seul  indice:
\begin{center}
 \begin{tikzpicture}[domain=0:5]
  \draw[->] (-0.1,0) -- (4.5,0)  node[right] {$\scriptstyle x$};
  \draw[->] (0,-0.1) -- (0,4.3) node[above] {$\scriptstyle y$};
  \path[fill=black]  (0,0) circle (.4mm) [fill=orange] node[below] {$\scriptstyle O$};


   \path[fill=black]  (4,0) circle (.4mm) [fill=orange] node[below] {$\scriptstyle A$};

   \path[fill=black]  (0,4) circle (.4mm) [fill=orange] node[left] {$\scriptstyle C$};
      \path[fill=black]  (4,4) circle (.4mm) [fill=orange] node[above right] {$\scriptstyle B$};
\draw[dotted,blue] (1,0) -- ++(0,4) ;
\draw[dotted,blue] (2,0) -- ++(0,4) ;
\draw[dotted,blue] (3,0) -- ++(0,4) ;
\draw[dotted,blue] (0,1) -- ++(4,0) ;
\draw[dotted,blue] (0,2) -- ++(4,0) ;
\draw[dotted,blue] (0,3) -- ++(4,0) ;
\draw[orange] (0.02,0.02) -- ++(4,0)  -- ++(0,4) -- ++(-4,0)  -- ++(0,-4)  ;
 \path[fill=red]  (1,1) circle (.5mm)node[below] {$\scriptstyle 1$};
 \path[fill=red]  (2,1) circle (.5mm) node[below] {$\scriptstyle 2$};
 \path[fill=red]  (3,1) circle (.5mm)node[below] {$\scriptstyle 3$};
  \path[fill=red]  (1,2) circle (.5mm) node[below] {$\scriptstyle 4$};
  \path[fill=red]  (2,2) circle (.5mm)  node[ below] {$\scriptstyle 5$};
   \path[fill=red]  (3,2) circle (.5mm)  node[ below] {$\scriptstyle 6$};
   \path[fill=red]  (1,3) circle (.5mm) node[below] {$\scriptstyle 7$};
  \path[fill=red]  (2,3) circle (.5mm)  node[ below] {$\scriptstyle 8$};
   \path[fill=red]  (3,3) circle (.5mm)  node[ below] {$\scriptstyle 9$};
  
\end{tikzpicture}
\[(i,j)\mapsto i+(j-1)N \]
 \end{center}
 
\[\begin{array}{l}
m=1:\quad {\displaystyle -\frac{0-2u_{1}+u_{2}}{\delta x^2}-\frac{0-2u_{1}+u_{4}}{\delta y^2} }=f_{1} \\
m=2:\quad {\displaystyle -\frac{u_{1}-2u_{2}+u_{3}}{\delta x^2}-\frac{0-2u_{2}+u_{5}}{\delta y^2} }=f_{2} \\
m=3:\quad {\displaystyle -\frac{u_{2}-2u_{3}+0}{\delta x^2}-\frac{0-2u_{3}+u_{6}}{\delta y^2} }=f_{3} \\
m=4:\quad {\displaystyle -\frac{0-2u_{4}+u_{5}}{\delta x^2}-\frac{u_{1}-2u_{4}+u_{7}}{\delta y^2} }=f_{4} \\
m=5:\quad {\displaystyle -\frac{u_{4}-2u_{5}+u_{6}}{\delta x^2}-\frac{u_{2}-2u_{5}+u_{8}}{\delta y^2} }=f_{5} \\
m=6:\quad {\displaystyle -\frac{u_{5}-2u_{6}+0}{\delta x^2}-\frac{u_{3}-2u_{6}+u_{9}}{\delta y^2} }=f_{6} \\
m=7:\quad {\displaystyle -\frac{0-2u_{7}+u_{8}}{\delta x^2}-\frac{u_{4}-2u_{7}+0}{\delta y^2} }=f_{7} \\
m=8:\quad {\displaystyle -\frac{u_{7}-2u_{8}+u_{9}}{\delta x^2}-\frac{u_{5}-2u_{8}+0}{\delta y^2} }=f_{8} \\
m=9:\quad {\displaystyle -\frac{u_{8}-2u_{9}+0}{\delta x^2}-\frac{u_{6}-2u_{9}+0}{\delta y^2} }=f_{9} \\
\end{array}
\]

\[\begin{array}{l}
m=1:\quad {\displaystyle 2\left(\frac{1}{\delta x^2}+\frac{1}{\delta y^2}\right)u_1-\frac{1}{\delta x^2}u_2-\frac{1}{\delta y^2}u_4}=f_{1} \\
m=2:\quad {\displaystyle -\frac{1}{\delta x^2}u_1+ 2\left(\frac{1}{\delta x^2}+\frac{1}{\delta y^2}\right)u_2-\frac{1}{\delta x^2}u_3-\frac{1}{\delta y^2}u_5 }=f_{2} \\
m=3:\quad {\displaystyle -\frac{1}{\delta x^2}u_2+ 2\left(\frac{1}{\delta x^2}+\frac{1}{\delta y^2}\right)u_3-\frac{1}{\delta y^2}u_6}=f_{3} \\
m=4:\quad {\displaystyle -\frac{1}{\delta y^2}u_1+ 2\left(\frac{1}{\delta x^2}+\frac{1}{\delta y^2}\right)u_4-\frac{1}{\delta x^2}u_5-\frac{1}{\delta y^2}u_7 }=f_{4} \\
m=5:\quad {\displaystyle -\frac{1}{\delta y^2}u_2-\frac{1}{\delta x^2}u_4+ 2\left(\frac{1}{\delta x^2}+\frac{1}{\delta y^2}\right)u_5-\frac{1}{\delta x^2}u_6-\frac{1}{\delta y^2}u_8}=f_{5} \\
m=6:\quad {\displaystyle -\frac{1}{\delta y^2}u_3-\frac{1}{\delta x^2}u_5+ 2\left(\frac{1}{\delta x^2}+\frac{1}{\delta y^2}\right)u_6-\frac{1}{\delta y^2}u_9}=f_{6} \\
m=7:\quad {\displaystyle -\frac{1}{\delta y^2}u_4+ 2\left(\frac{1}{\delta x^2}+\frac{1}{\delta y^2}\right)u_7-\frac{1}{\delta x^2}u_8 }=f_{7} \\
m=8:\quad {\displaystyle -\frac{1}{\delta y^2}u_5-\frac{1}{\delta x^2}u_7+ 2\left(\frac{1}{\delta x^2}+\frac{1}{\delta y^2}\right)u_8-\frac{1}{\delta x^2}u_9}=f_{8} \\
m=9:\quad {\displaystyle -\frac{1}{\delta y^2}u_6-\frac{1}{\delta x^2}u_8+ 2\left(\frac{1}{\delta x^2}+\frac{1}{\delta y^2}\right)u_9 }=f_{9} \\
\end{array}
\]
Pour simplifier, on pose $\beta = \frac{1}{\delta x^2}$, $\gamma = \frac{1}{\delta y^2}$ et $\alpha =2(\beta+\gamma)$:
\[\myredbox{\begin{array}{l}
m=1:\quad {\displaystyle \alpha u_1-\beta u_2-\gamma u_4}=f_{1} \\
m=2:\quad {\displaystyle -\beta u_1+ \alpha  u_2-\beta u_3-\gamma u_5 }=f_{2} \\
m=3:\quad {\displaystyle -\beta u_2+ \alpha u_3-\gamma u_6}=f_{3} \\
m=4:\quad {\displaystyle -\gamma u_1+ \alpha u_4-\beta u_5-\gamma u_7 }=f_{4} \\
m=5:\quad {\displaystyle -\gamma u_2-\beta u_4+ \alpha u_5-\beta u_6-\gamma u_8}=f_{5} \\
m=6:\quad {\displaystyle -\gamma u_3-\beta u_5+ \alpha u_6-\gamma u_9}=f_{6} \\
m=7:\quad {\displaystyle -\gamma u_4+ \alpha u_7-\beta u_8 }=f_{7} \\
m=8:\quad {\displaystyle -\gamma u_5-\beta u_7+ \alpha u_8-\beta u_9}=f_{8} \\
m=9:\quad {\displaystyle -\gamma u_6-\beta u_8+ \alpha u_9 }=f_{9} \\
\end{array}}
\]
\[
\left[\begin{array}{ccc|ccc|ccc}
\alpha &-\beta &0&-\gamma &&\\
-\beta &\alpha &-\beta &&-\gamma &\\
0&-\beta &\alpha &&&-\gamma \\  \hline
-\gamma &&&\alpha &-\beta &0&-\gamma &&\\
&-\gamma &&-\beta &\alpha &-\beta &&-\gamma &\\
&&-\gamma &0&-\beta &\alpha &&&-\gamma \\  \hline
&&&-\gamma &&&\alpha &-\beta &0\\
&&&&-\gamma &&-\beta &\alpha &-\beta \\
&&&&&-\gamma &0&-\beta &\alpha 
\end{array}\right]
\left[\begin{array}{c}
u_{1}\\u_{2}\\ u_{3} \\  \hline
u_{4}\\u_{5}\\ u_{6} \\  \hline
u_{7}\\u_{8}\\ u_{9}
\end{array}\right] =h^2\left[\begin{array}{c}
f_{1}\\f_{2}\\ f_{3} \\ \hline
 f_{4}\\f_{5}\\ f_{6} \\  \hline
 f_{7}\\f_{8}\\ f_{9}
\end{array}\right]
\] 
La matrice $A$ peut s'écrire en blocs de la façon suivante:
\[
\left[\begin{array}{c|c|c}
\beta K+2\gamma I_3 &-\gamma I_3 &\\ \hline
 -\gamma I_3 &\beta K+2\gamma I_3 &-\gamma I_3 \\ \hline
 &-\gamma I_3 &\beta K+2\gamma I_3  
\end{array}\right]
\]
où $K$ est la matrice symétrique définie positive suivante:
\[
K=\left[\begin{array}{ccc}
2 &-1 &0\\ 
 -1 &2&-1 \\ 
 0&-1 &2  
\end{array}\right]
\]
Soit le vecteur en blocs:
\[X=\left[\begin{array}{c}
X_1 \\ \hline
 X_2 \\  \hline
 X_3
\end{array}\right]
\]
On a
\[
\begin{array}{lcl}
<A X\,,\, X>&=&<
\left[\begin{array}{c|c|c}
\beta K+2\gamma I_3 &-\gamma I_3 &\\ \hline
 -\gamma I_3 &\beta K+2\gamma I_3 &-\gamma I_3 \\ \hline
 &-\gamma I_3 &\beta K+2\gamma I_3  
\end{array}\right]\, \left[\begin{array}{c}
X_1 \\ \hline
 X_2 \\  \hline
 X_3
\end{array}\right]\;,\; \left[\begin{array}{c}
X_1 \\ \hline
 X_2 \\  \hline
 X_3
\end{array}\right]
>\\
&&\\
&=&{\displaystyle \beta\left(^tX_1KX_1+^tX_2KX_2+^tX_3KX_3\right)}\\
&&+\gamma\left(2\|X_1\|^2+2\|X_2\|^2+2\|X_3\|^2-2^tX_1X_2-2^tX_2X_3\right)\\
&&\\
&=&{\displaystyle \beta\left(^tX_1KX_1+^tX_2KX_2+^tX_3KX_3\right)}\\
&&+\gamma\left(\|X_1\|^2+\|X_1-X_2\|^2+\|X_2-X_3\|^2+\|X_3\|^2\right)\\
&&\\
&\geq 0&
\end{array}
\]
et $<A X\,,\, X>=0 \Longrightarrow X=0$ et donc $A$ est bien une matrice symétrique définie positive.
\item Appliquons la matrice $A$ au vecteur ligne par ligne:
\[
\begin{array}{l}
\displaystyle (A u(p,q))_{i,j}=\underbrace{\frac {-u(p,q)_{i-1,j} + 2u(p,q)_{i,j}-u(p,q)_{i+1,j}}{\delta x^2} }_{\color{red}{A_1}}+
\underbrace{\frac {-u(p,q)_{i,j-1} + 2u(p,q)_{i,j}-u(p,q)_{i,j+1}}{\delta y^2}}_{\color{red}{ A_2}}
\end{array}
\]
\[
\begin{array}{lcl}
A_1&=&\displaystyle \frac {-u(p,q)_{i-1,j} + 2u(p,q)_{i,j}-u(p,q)_{i+1,j}}{\delta x^2} \\
      &= &\displaystyle  \frac {-\sin\left(p\pi \frac{(i-1)\delta x}{a}\right) + 2\sin\left(p\pi \frac{i\delta x}{a}\right)-\sin\left(p\pi \frac{(i+1)\delta x}{a}\right)}{\delta x^2} \sin\left(q\pi \frac{j\delta y}{b}\right)\\
      &= &\displaystyle  \frac {-2\sin\left(p\pi \frac{i\delta x}{a}\right)\cos\left(p\pi \frac{\delta x}{a}\right)  + 2\sin\left(p\pi \frac{i\delta x}{a}\right)}{\delta x^2} \sin\left(q\pi \frac{j\delta y}{b}\right)\\
      &= &\displaystyle 2 \frac {1-\cos\left(p\pi \frac{\delta x}{a}\right)}{\delta x^2}\sin\left(p\pi \frac{i\delta x}{a}\right) \sin\left(q\pi \frac{j\delta y}{b}\right)\\
      &= &\displaystyle \frac{4}{\delta x^2} \sin^2\left(p\pi \frac{\delta x}{2a}\right)\sin\left(p\pi \frac{i\delta x}{a}\right) \sin\left(q\pi \frac{j\delta y}{b}\right)\\
      &= &\displaystyle \frac{4}{\delta x^2} \sin^2\left(p\pi \frac{\delta x}{2a}\right)u(p,q)_{i,j}
\end{array}
\]

\[\myredbox{A_1=\displaystyle \frac{4}{\delta x^2} \sin^2\left(p\pi \frac{\delta x}{2a}\right)u(p,q)_{i,j}}
\]

Il en est de même pour $A_2$:
\[\myredbox{A_2=\displaystyle \frac{4}{\delta y^2} \sin^2\left(q\pi \frac{\delta y}{2b}\right)u(p,q)_{i,j}}
\]
D'où
\[(A u(p,q))_{i,j} =A_1+A_2=\left[\frac{4}{\delta x^2} \sin^2\left(p\pi \frac{\delta x}{2a}\right)+ \frac{4}{\delta y^2} \sin^2\left(q\pi \frac{\delta y}{2b}\right)\right]u(p,q)_{i,j}\]
et
\[\myredbox{\lambda_{p,q}=\frac{4}{\delta x^2} \sin^2\left(p\pi \frac{\delta x}{2a}\right)+ \frac{4}{\delta y^2} \sin^2\left(q\pi \frac{\delta y}{2b}\right)}\]
\item \[\max \lambda_{p,q} = \lambda_{N,M}\simeq \frac{4}{\delta x^2} \times 1+ \frac{4}{\delta y^2}  \times 1 = 4\left(\frac{1}{\delta x^2} +\frac{1}{\delta y^2} \right)\]
 \[\min \lambda_{p,q}= \lambda_{1,1} \simeq \frac{4}{\delta x^2} \times \left(\pi \frac{\delta x}{2a}\right)^2+ \frac{4}{\delta y^2}  \times \left(\pi \frac{\delta y}{2b}\right)^2 = \pi^2\left(\frac{1}{a^2} +\frac{1}{b^2} \right)\]
 D'où 
 \[\myredbox{\mbox{cond}_2A=\frac{4}{\pi^2}\frac{\frac{1}{\delta x^2} +\frac{1}{\delta y^2}}{\frac{1}{a^2} +\frac{1}{b^2} }}\]
 Si $M=N$ et $a=b$ alors
 \[\myredbox{\mbox{cond}_2A \sim \frac{4N^2}{\pi^2}}\]
\end{enumerate}

\end{document}