\documentclass[a4paper]{article} 
\usepackage[francais]{babel}
\usepackage[utf8]{inputenc} % Required for including letters with accents
\usepackage[T1]{fontenc} % Use 8-bit encoding that has 256 glyphs

\usepackage{amsthm}
\usepackage{amsmath}
\usepackage{amssymb}
\usepackage{mathrsfs}
\usepackage{graphicx}
\usepackage{geometry}
\usepackage{stmaryrd}
\usepackage{tikz}

\def \de {{\rm d}}



\title{TD 1 Analyse numérique (B1-TP1)}
\author{Ibrahim ALAME}
\date{25/01/2024}
\begin{document}
\maketitle
\section{}
{\bf Interpolation: } On veut calculer, par trois méthodes distinctes, le polynôme d'interpolation de de la fonction $f(x) = \sin(x)$ en les $3$ points $x_i =i\frac{\pi}{2}$ pour $i = 0, 1 , 2$. On cherche donc $p(x)\in\mathbb{R}[x]$ tel que $p(x_i) = \sin(x_i )$ pour $i = 0, 1 , 2$. Utiliser pour cela les trois méthodes suivantes:
\begin{enumerate}
\item Méthode directe: Poser $P(x)=ax^2+bx+c$ puis résoudre le système linéaire $p(x_i) = \sin(x_i )$ pour $i = 0, 1 , 2$.
\item Méthode de Lagrange: Appliquer $P(x)=\sum_{i=0}^2\sin(x_i)L_i(x)$ après avoir calculer les polynôme $L_i(x)$ pour $i = 0, 1 , 2$.
\item Méthode de Newton: Appliquer $P(x)=\sum_{i=0}^2\Delta_i\Pi_{k=0}^{i-1}(x-x_k)$ après avoir calculer les différences divisées $\Delta_i$ pour $i = 0, 1 , 2$.
\end{enumerate}
\begin{center}
 \begin{tikzpicture}[scale=1]
\draw  [very thin, gray] [->]  (-0.2,0) -- (6.5,0); 
\draw  [very thin, gray] [->] (0,-1.2) -- (0,1.2);
\draw  [dashed] (1.57,0) -- (1.57,1)--(0,1);
\path[fill=black] (0,0) circle (1.2mm) [fill=gray];
\path[fill=black]  (1.57,1) circle (1.2mm) [fill=gray];
\path[fill=black]  (3.142,0) circle (1.2mm) [fill=gray];
\draw [domain=0:3.5][line width=1] plot(\x,{-.4052*\x*(\x-3.142)});
\draw [domain=0:6] plot(\x,{sin(57.29*\x)});
\end{tikzpicture} 

\end{center}
Maintenant on veut calculer le polynôme d'interpolation en ajoutant le nouveau point d'interpolation $x_3=\frac{3\pi}2$. Appliquer les trois méthodes précédentes et conclure.

\begin{center}
 \begin{tikzpicture}[scale=1]
\draw  [very thin, gray] [->]  (-0.2,0) -- (6.5,0); 
\draw  [very thin, gray] [->] (0,-1.2) -- (0,1.2);
\draw  [dashed] (1.57,0) -- (1.57,1)--(0,1);
\draw  [dashed] (4.71,0) -- (4.71,-1)--(0,-1);
\path[fill=black] (0,0) circle (1.2mm) [fill=gray];
\path[fill=black]  (1.57,1) circle (1.2mm) [fill=gray];
\path[fill=black]  (3.142,0) circle (1.2mm) [fill=gray];
\path[fill=black]  (4.713,-1) circle (1.2mm) [fill=gray];

\draw [domain=0:5][line width=1] plot(\x,{.8598e-1*\x*\x*\x-.8104*\x*\x+1.697*\x});
\draw [domain=0:6] plot(\x,{sin(57.29*\x)});
\end{tikzpicture} 

\end{center}

\section{} {\bf Convergence de l'interpolation de Lagrange : } Soit $L_n$ le polynôme d'interpolation de Lagrange de la fonction définie par :
\[f(x) = \frac 1{x-\alpha},\qquad -1\leq x \leq 1 \]
aux $n + 1$ points distincts $x_0, ..., x_n$ de l'intervalle $[-1,1]$.
\begin{enumerate}
\item Calculer les dérivées successives de la fonction $f$.
\item  Montrer que si $\alpha > 3$, et si les $n + 1$ points $x_0, ..., x_n$ sont équidistants, nous avons alors
\[\lim_{n\to \infty}\|f-L_n\|_{\infty}=0\]
\item Écrire l'approximation spline linéaire, $f_n$ de $f$.
\item  Montrer que si $\alpha \notin [-1,1]$, nous avons
\[\|f-f_n\|_{ \infty}\leq \frac c{n^2}\]
et donc que $f_n$ converge uniformément vers $f$ lorsque $n$ tend vers l'infini.
\end{enumerate}

\section{} {\bf Moindre carrés discrets : }
Nous recherchons le polynôme $P$ de degré 2 qui approche le mieux le nuage de points $(x_i, y_i)_{1\leq i\leq 6}$  suivant :
\[(-2,5),\quad(-1,1),\quad (0, -3),\quad (1, -3),\quad (2, -2),\quad (3, 1)\]

\begin{enumerate}
\item  Tracer le nuage de points.
\item On pose $P(x)=ax^2+bx+c$. les inconnues $a$ , $b$ et $c$ sont solution du système:

\[\left(\begin{array}{ccc}
1 & \overline{x} & \overline{x^2}\\
\overline{x} & \overline{x^2} & \overline{x^3} \\
 \overline{x^2} & \overline{x^3} & \overline{x^4} 
\end{array}\right) \left(\begin{array}{c}
c \\
b \\
a 
\end{array}
\right) =\left(\begin{array}{c}
\overline{y} \\
\overline{x y} \\
\overline{x^2 y} 
\end{array}
\right) 
\]

\item  Résoudre le système et tracer la solution du problème.
\end{enumerate}

\section{} {\bf Intégration numérique : }
  On considère l'intégrale
\[ I = \int_1^2\frac 1x {\mbox d}x \]
\begin{enumerate}
\item  Calculer la valeur exacte de $I$.
\item  Évaluer numériquement cette intégrale par la méthode des trapèzes avec n = 3 sous-intervalles.
\item  Pourquoi la valeur numérique obtenue à la question précédente est-elle supérieure à $\ln(2)$? Est-ce vrai quelque soit $n$ ? Justifier la réponse. (On pourra s'aider par un dessin.)
\item  Quel nombre de sous-intervalles n faut-il choisir pour avoir une erreur inférieure à $10^{-4}$? On rappelle que l'erreur de quadrature associée s'écrit, si $f\in C^2([a,b])$.
\[ E =\left|\frac{(b-a)^4}{12 n^2} f''(\xi)\right| \qquad \xi \in  ]a,b[\]
\end{enumerate}

\section{}
Voici le relevé de la vitesse d'écoulement de l'eau $v$ dans
un conduit cylindrique en fonction du temps $t$ :
\[
\begin{array}{|c||c|c|c|c|c|c|c|}\hline
t \mbox{(s)} & 0 & 10 & 20 &30 &40 &50 &60\\ \hline
v \mbox{(m/s)} &2 &1.98 &1.7 &1.44 &1.32 &1.20 &1.02\\ \hline
\end{array}
\]
La vitesse moyenne de l'eau en écoulement dans le conduit cylindrique peut être calculée par la relation suivante :
\[\overline{v} =\frac 1{60}\int_0^{60}v(t)\de t\]
\begin{enumerate}
\item Calculer la vitesse moyenne de l'eau $\overline{v} $ par la méthode des rectangles à droite.
\item  Calculer la vitesse moyenne de l'eau $\overline{v} $ par la méthode des trapèzes.
\item Peut-on déterminer la vitesse moyenne de l'eau $\overline{v} $ par la méthode de Simpson ? Justifier rigoureusement votre réponse.
\end{enumerate}

\section{}
 On souhaite déterminer une valeur approchée de 
\[I =\int_0^1 e^{-x^2}\de x\]
 en subdivisant l'intervalle $[0, 1]$ en $N = 10$ sous-intervalles.
 \begin{enumerate}
\item  Majorer l'erreur commise en utilisant les différentes méthodes usuelles (rectangle à
gauche, rectangle à droite, point milieu, trapèzes, Simpson).
\item Proposer une approche permettant de déterminer une valeur approchée de $I$ à $10^{-10}$ près.
(On pourra envisager une autre valeur de $N$).
 \end{enumerate}
 \section{} 
On rappelle que l'erreur commise par la méthode des trapèzes pour une fonction $f$
de classe $C^2([a, b])$ est majorée ainsi :
\[
|I(f, a, b) - I_N(f, a, b)| \leq \frac{(b-a)M_2}{12}h^2\]
où $M_2 =\displaystyle \sup_{a\leq x\leq b}|f(2)(x)|$. Combien faut-il de subdivisions de $[0, 1]$ pour évaluer l'intégrale
\[I =\int_0^1 xe^{-x}\de x\]
à $10^{-6}$ près ?


\section{} 
 On considère $f$ une fonction de classe $C^2$ sur un intervalle $J = [a, b]$, que l'on
subdivise en $N$ sous-intervalles.
On note respectivement $I_{RG}$, $I_{RD}$, $I_T$ , $I_{PM}$, $I_S$ les approximations données par les méthodes usuelles
(rectangles à gauche, rectangle à droite, trapèzes, point milieu, Simpson) de l'intégrale $I = \int_a^bf(x)\de x$.
\begin{enumerate}
\item Montrer que $I_T =\frac{I_{RG}+I_{RD}}2$
\item  Exprimer $I_S$ en fonction de $I_{PM}$ et $I_T$ .
\item Lorsque l'on connaît $I_{RG}$ comment calculer rapidement $I_{RD}$ ?
\item On suppose que $f$ est une fonction croissante. Montrer que l'on a les inégalités
$I_{RG} \leq I \leq I_{RD}$.
\item On suppose maintenant que $f$ est une fonction convexe. Montrer que $I_{PM} \leq I \leq I_T$.

\end{enumerate}

\section{} 
 \begin{enumerate}
\item Déterminer par la méthode des rectangles à gauche et à droite puis celle des trapèzes la valeur de $I = \int_a^bf(x)\de x$.
sur la base des valeurs données dans le tableau suivant
\[
\begin{array}{|c||c|c|c|c|c|c|}\hline
x & 0 & 0,1 & 0,2 &0,3 &0,4 &0,5\\ \hline
f(x)&0 &0,0993346 &0,1947091 &0,2823212 &0,3586780 &0,4207354\\ \hline
\end{array}
\]

\item Ces points sont ceux donnant $f(x) = \sin (t) \cos (t)$. Comparer alors les résultats obtenus avec la valeur exacte. Conclure.
\end{enumerate} 

\section{} 
 Le but de l'exercice est de calculer une valeur approchée de l'intégrale $I = \int_0^1x^3e^{-x}\de x$. et la comparer avec sa valeur exacte. Pour cela, on découpe l'intervalle $[0, 1]$ en $N = 10$ sous intervalle
de même longueur.
\begin{enumerate}
\item Calculer la valeur du pas $h$.
\item Donner la valeur exacte de l'intégrale $I$.
\item \begin{enumerate}
\item Calculer $I$ par la méthode des rectangles à gauche.
\item Calculer $I$ par la méthode des rectangles du point milieu.
\item Calculer $I$ par la méthode des trapèzes.
\end{enumerate} 
\item Déterminer numériquement l'erreur relative commise par chaque méthode. Conclure.
\end{enumerate} 


\section{}{\bf Formule de Quadrature : } Considérons la formule de quadrature suivante
\[\int_{-1}^1f(x)\de x=\omega_1f\left(-\alpha \right)+\omega_2f\left(\alpha\right)\]
Où $\alpha \in ]0; 1]$.
\begin{enumerate}
\item  Déterminer les poids pour que cette formule de quadrature soit exacte pour les polynômes de $\mathbb{R}_1[X]$
\item On adopte désormais les poids déterminés à la question précédente. Quelle est la formule obtenue lorsque $\alpha=1$ ? Est-elle exacte pour les polynômes de $\mathbb{R}_2[X]$ ?
\item Montrer que cette formule de quadrature est exacte sur $\mathbb{R}_2[X]$ pour une et une seule valeur de $\alpha$, à déterminer.
\item On adopte la valeur de déterminée à la question précédente. La formule est-elle exacte sur $\mathbb{R}_3[X]$ ? sur $\mathbb{R}_4[X]$ ?
\item  Adapter la formule obtenue à une intégrale $\int_{0}^1f(x)\de x$, puis à une intégrale $\int_{a}^bf(x)\de x$.
\item Exprimer la formule composite obtenue lorsque l'on subdivise l'intervalle $[a; b]$ en $N$ sous-intervalles.
\end{enumerate}

%%%%%%%%%%%%%%%%%%%%%%%%%%%%%%%%%%%%%%%%%%%%%%%%%%%%%%%%%%%%%%%%%%%%%%%%%%%%%%%


\section{} On propose dans un premier temps de construire la formule de quadrature à deux points
\[\int_{-1}^1f(x)\de x=\frac 43f\left(-\frac{\xi}{2}\right)+\frac 23f\left(\xi\right)\]
\begin{enumerate}
\item Déterminer $\xi$ pour que la formule de quadrature soit exacte pour toute fonction $f$ polynomiale de degré $m \geq 1$ et donner la plus grande valeur de $m$.
\item À l'aide d'un changement de variable affine, en déduire une formule de quadrature pour l'intégrale :
\[\int_{a}^bf(x)\de x\]
\item En déduire une formule de quadrature composite à $2n+1$ points, pour le calcul approché de $\int_{a}^bf(x)\de x$. 
\end{enumerate}

%%%%%%%%%%%%%%%%%%%%%%%%%%%%%%%%%%%%%%%%%%%%%%%%%%%%%%%%%%%%%%%%%%%%%%%%%%%%%%%



\section{} Soit $f \in C^1([0; 1])$. On considère la formule de quadrature élémentaire \[\int_0^1 f(x) \de x \simeq \omega_0 f(0) +  \omega_1f'(0) +  \omega_2f'(\xi)\]
où $\xi\in ]0; 1]$ et $\omega_0$, $\omega_1$ et $\omega_2$ sont des réels. On pose
\[E(f) =\int_0^1 f(x) \de x -\left[ \omega_0 f(0) +  \omega_1f'(0) +  \omega_2f'(\xi)\right]\]
\begin{enumerate}
\item Déterminer les paramètres $\xi$, $\omega_0$, $\omega_1$ et $\omega_2$ pour que la formule de quadrature soit exacte si $f$ est un polynôme de degré inférieur ou égal à $3$.
\item  Les paramètres $\xi$, $\omega_0$, $\omega_1$ et $\omega_2$ étant ainsi fixés, calculer $E(x \mapsto x^4)$ et en déduire l'ordre de la
méthode. 
\item  A l'aide d'un changement de variable, construire une méthode de quadrature élémentaire sur un intervalle $[a, b]$.
\end{enumerate}

%%%%%%%%%%%%%%%%%%%%%%%%%%%%%%%%%%%%%%%%%%%%%%%%%%%%%%%%%%%%%%%%%%%%%%%%%%%%%%%
\section{}  \begin{enumerate}
\item Soit la formule de quadrature
\[\int_{0}^1f(x)\de x=\omega_1f(0)+\omega_2f(\alpha)\]
Où $\alpha \in ]0; 1]$.
\begin{enumerate}
\item Déterminer les poids pour que la formule soit exacte sur $\mathbb{R}_1[X]$.
\item Déterminer $\alpha$ pour que la méthode soit exacte pour les polynômes de degré $\leq 2$.
\item Adapter la formule obtenue à un intervalle $[0; h]$.
\end{enumerate}
\item  On va dans le cas de cette formule pour une intégrale $\int_{0}^hf(x)\de x$ prouver une estimation de l'erreur commise par la formule de quadrature pour une fonction $f$ supposée de classe $C^3$ sur $[0; h]$.

On notera $I(f) =\int_{0}^hf(x)\de x$, $Q(f)$ l'approximation donnée par la formule de quadrature, et enfin $E(f) = I(f) - Q(f)$.
\begin{enumerate}
\item  Soit $M_3 =\displaystyle \sup_{0\leq x\leq h}\left|f^{(3)}(x)\right|$.
Montrer que l'on peut écrire $f(x) = P(x) + R(x)$ avec $P$ un polynôme de degré 2,
que l'on précisera et $R$ une fonction vérifiant
\[\forall x\in [0,h],\quad |R(x)|\leq \frac{M_3}6x^3\]
\item  Majorer en fonction de $M_3$ et de $h$ les valeurs de $|I(R)|$, de $|Q(R)|$, de $|E(R)|$.
\item En déduire une majoration de $|E(f)|$.
\end{enumerate}

\item  \begin{enumerate}
\item   Donner la formule composite pour le calcul d'une intégrale $\int_{a}^bf(x)\de x$ associée à la formule de quadrature élémentaire étudiée.
\item  Donner une majoration de l'erreur lorsqu'on utilise cette formule composite.
\end{enumerate}
\end{enumerate}

%%%%%%%%%%%%%%%%%%%%%%%%%%%%%%%%%%%%%%%%%%%%%%%%%%%%%%%%%%%%%%%%%%%%%%%%%%%%%%%

\section{} Soit une formule de quadrature élémentaire à $p$ points. Montrer que cette formule ne peut pas être exacte pour tous les polynômes de $\mathbb{R}_{2p}[X]$.
(Indication : considérer le polynôme $Q(x) =\prod_{i=1}^p(x-\alpha_i)^2$)

%%%%%%%%%%%%%%%%%%%%%%%%%%%%%%%%%%%%%%%%%%%%%%%%%%%%%%%%%%%%%%%%%%%%%%%%%%%%%%%


\section{} Soit $p$ un entier avec $p > 1$.
La formule à $p$ points définie par
\[\int_{0}^1f(x)\de x=\frac 1p\sum_{i=1}^p\left(\frac{i-1}{p-1}\right)\]
est-elle exacte sur $\mathbb{R}_{1}[X]$ ? Sur $\mathbb{R}_{2}[X]$ ?

\section{Polynômes orthogonaux}
Soit $E = \mathbb{R}[X]$. On munit $E$ du produit scalaire 
\[(P|Q) = \int_{-1}^{+1} P(t)Q(t) \mbox{d}t\]

\begin{enumerate}
\item  Pour $n\in \mathbb{N}$, on pose 
\[P_n(X) =\frac 1{n! 2^n}\frac{\mbox{d}^n}{\mbox{d}X^n}\left[\left(X^2-1\right)^n\right]\]

\begin{enumerate}
\item  Montrer que la famille $(P_n)_{n\in \mathbb{N}}$ est une famille orthogonale de $E$.
\item  Déterminer $\|P_n\|$ pour $n\in \mathbb{N}$.
\end{enumerate}
\item   Déterminer $P_0$, $P_1$, $P_2$ et $P_3$.
\item Déterminer les paramètres $\xi_i$ et $\omega_i$, $i=1,2,3$ de la formule de quadrature
\[\int_{-1}^1f(x)\de x \simeq \omega_1f(\xi_1)+\omega_2f(\xi_2)+\omega_3f(\xi_3)\]
\item En déduire la formule composite de la méthode de Gauss-Legendre pour approcher à l'ordre 3  la valeur de l'intégral \[\int_{a}^bf(x)\de x\]

\end{enumerate}


\end{document}