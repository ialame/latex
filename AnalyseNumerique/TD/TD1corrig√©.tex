\documentclass[a4paper]{article} 
\usepackage[francais]{babel}
\usepackage[utf8]{inputenc} % Required for including letters with accents
\usepackage[T1]{fontenc} % Use 8-bit encoding that has 256 glyphs

\usepackage{amsthm}
\usepackage{amsmath}
\usepackage{amssymb}
\usepackage{mathrsfs}
\usepackage{graphicx}
\usepackage{geometry}
\usepackage{stmaryrd}
\usepackage{tikz}

\def \de {{\rm d}}

\newcommand{\mybox}[1]{\fbox{$\displaystyle#1$}}
\newcommand{\myredbox}[1]{\fcolorbox{red}{white}{$\displaystyle#1$}}
\newcommand{\mydoublebox}[1]{\fbox{\fbox{$\displaystyle#1$}}}
\newcommand{\myreddoublebox}[1]{\fcolorbox{red}{white}{\fcolorbox{red}{white}{$\displaystyle#1$}}}

\title{TD 1 Analyse numérique (B1-TP1)}
\author{Ibrahim ALAME}
\date{23/03/2024}
\begin{document}
\maketitle
\section{Interpolation}

\begin{enumerate}
\item 
\begin{enumerate}
\item Méthode directe: On cherche $a$, $b$ et $c$ solution de
\[\left(\begin{array}{ccc}
0&0&1\\
\frac{\pi^2}{4}&\frac{\pi}{2}&1\\
\pi^2&\pi&1
\end{array}\right)\left(\begin{array}{c} a\\b\\c \end{array}\right)=\left(\begin{array}{c} 0\\1\\0 \end{array}\right)\]

On trouve \[P(x)=-\frac{4}{\pi^2}x^2+\frac{4}{\pi}x\]

\item Méthode de Lagrange: 
\[P(x)=\sin(0) L_0(x)+\sin(\frac{\pi}{2})L_1(x)+\sin(\pi)L_2(x)=L_1(x)=\frac{(x-0)(x-\pi)}{(\frac{\pi}{2}-0)(\frac{\pi}{2}-\pi)}=\frac{-4}{\pi^2}x(x-\pi)\]
\item Méthode de Newton: 
\[\begin{array}{cc|cc}\hline
x_i&y_i&\Delta[x_{i-1},x_i]&\Delta[x_{i-2},x_{i-1},x_i] \\ \hline\hline
0&\boxed{0}&& \\
\frac{\pi}{2}&1&\boxed{\frac{2}{\pi}}& \\
\pi&0&-\frac{2}{\pi}&\boxed{-\frac{4}{\pi^2}}
\end{array}
\]
D'où
\[\begin{array}{ccl}
P(x)&=&\Delta[x_0]+\Delta[x_0,x_1](x-x_0)+\Delta[x_0,x_1,x_2](x-x_0)(x-x_1)\\
&=&0+\frac{2}{\pi}(x-0)-\frac{4}{\pi^2}(x-0)(x-\frac{\pi}{2})\\
&=&-\frac{4}{\pi^2}x^2+\frac{4}{\pi}x
\end{array}
\]



\end{enumerate}
\item 
\begin{enumerate}
\item Méthode directe: On cherche $a$, $b$ , $c$ et $d$ solution de
\[\left(\begin{array}{cccc}
0&0&0&1\\
\frac{\pi^3}{8}&\frac{\pi^2}{4}&\frac{\pi}{2}&1\\
\pi^3&\pi^2&\pi&1\\
\frac{27\pi^3}{8}&\frac{9\pi^2}{4}&\frac{3\pi}{2}&1
\end{array}\right)\left(\begin{array}{c} a\\b\\c\\d \end{array}\right)=\left(\begin{array}{c} 0\\1\\0\\-1 \end{array}\right)\]

On trouve \[P(x)=\frac{8}{3\pi^3}x^3-\frac{8}{\pi^2}x^2+\frac{16}{3\pi}x\]
                       


\item Méthode de Lagrange: 
\[\begin{array}{ccl}
P(x)&=&\sin(0) L_0(x)+\sin(\frac{\pi}{2})L_1(x)+\sin(\pi)L_2(x)+\sin(\frac{3\pi}{2})L_3(x)\\
&=&L_1(x)-L_3(x)\\
&=&\frac{(x-0)(x-\pi)(x-\frac{3\pi}{2})}{(\frac{\pi}{2}-0)(\frac{\pi}{2}-\pi)(\frac{\pi}{2}-\frac{3\pi}{2})}-\frac{(x-0)(x-\frac{\pi}{2})(x-\pi)}{(\frac{3\pi}{2}-0)(\frac{3\pi}{2}-\pi)(\frac{3\pi}{2}-\frac{\pi}{2})}\\
&=&\frac{4}{\pi^3}x(x-\pi)(x-\frac{3\pi}{2})-\frac{4}{3\pi^3}x(x-\frac{\pi}{2})(x-\pi)
\end{array}
\]
\item Méthode de Newton: 
\[\begin{array}{cc|ccc}\hline
x_i&y_i&\Delta[x_{i-1},x_i]&\Delta[x_{i-2},x_{i-1},x_i] &\Delta[x_{i-3},x_{i-2},x_{i-1},x_i]\\ \hline\hline
0&\boxed{0}&&& \\
\frac{\pi}{2}&1&\boxed{\frac{2}{\pi}}&& \\
\pi&0&-\frac{2}{\pi}&\boxed{-\frac{4}{\pi^2}}&\\
\frac{3\pi}{2}&-1&-\frac{2}{\pi}&0&\boxed{\frac{8}{3\pi^3}}\\
\end{array}
\]
D'où
\[\begin{array}{ccl}
P(x)&=&y_0+\Delta[x_0,x_1](x-x_0)+\Delta[x_0,x_1,x_2](x-x_0)(x-x_1)\\
&&+\Delta[x_0,x_1,x_2,x_3](x-x_0)(x-x_1)(x-x_2)\\
&=&0+\frac{2}{\pi}(x-0)-\frac{4}{\pi^2}(x-0)(x-\frac{\pi}{2})+\frac{8}{3\pi^3}(x-0)(x-\frac{\pi}{2})(x-\pi)\\
&=&\frac{8}{3\pi^3}x^3-\frac{8}{\pi^2}x^2+\frac{16}{3\pi}x
\end{array}
\]
La méthode de Newton est la plus simple, car elle réutilise les coefficients déjà calculés.


\end{enumerate}
\end{enumerate}

\section{Convergence de l'interpolation de Lagrange} 
\begin{enumerate}
\item \[f^{(n)}(x)=\frac{(-1)^n n!}{(x-\alpha)^{n+1}}\]
\item Nous avons
\[f(x) - P_n(x) =\frac{\prod_{i=0}^n\left(x-x_i\right)}{(n + 1)!}f^{(n+1)}(\xi)\]
où  $-1 < \xi < 1$. Donc
\[|f(x)-P_n(x)|\leq \frac{\Lambda_n }{(n + 1)!}.\frac{ (n + 1)!}{|\xi-\alpha|^{n+2}}\sim \frac{2^{n+1}}{\mbox{e}\, n\ln(n)}.\frac{ 1}{|\xi-\alpha|^{n+2}}\leq \frac{1}{2\mbox{e}\, n\ln(n)}\]
car d'après le cours  la constante de Lebesgue $\Lambda_n \sim \frac{2^{n+1}}{\mbox{e}\, n\ln(n)} $ et $|\xi-\alpha|>2$.  
D'où
\[\lim_{n\to \infty}\|f-P_n\|_{\infty}=0\]
\item  \[\left\{\begin{array}{rl}
 \mbox{Pour } &x_i<x<x_{i+1}\\
s_n(x)&=f(x_i)+\frac{x-x_i}{x_{i+1}-x_{i}}\left(f(x_{i+1})-f(x_{i})\right)\\
 &=\frac{1}{x_i-\alpha}-\frac{1}{(x_i-\alpha)(x_{i+1}-\alpha)}(x-x_i)
\end{array}\right.
\]
\item On fait un développement de Taylor à l'ordre 1 de $f$ en $x=x_i$:
\[f(x)=f(x_i)+(x-x_i)f'(x_i)+\frac{(x-x_i)^2}{2}f''(\xi)\]
soit
\[f(x)=\frac{1}{x_i-\alpha}+(x-x_i)\frac{-1}{(x_i-\alpha)^2}+\frac{(x-x_i)^2}{2}\frac{2}{(\xi-\alpha)^{3}}\]
Donc
\[\begin{array}{ll}
f(x)-s_n(x)  &= (x-x_i)\left[\frac{-1}{(x_i-\alpha)^2}+\frac{1}{(x_i-\alpha)(x_{i+1}-\alpha)}\right]+\frac{(x-x_i)^2}{(\xi-\alpha)^{3}}\\
&= \frac{(x_{i+1}-x_i)(x-x_i)}{(x_i-\alpha)^2(x_{i+1}-\alpha)}+\frac{(x-x_i)^2}{(\xi-\alpha)^{3}}
\end{array}
\]
\[|f(x)-s_n(x)|\leq (c_1\,h^2+c_2\,h^2)=\frac{4(c_1+c_2)}{n^2}=\frac{c}{n^2}\]
car $h=\frac 2n$, ainsi
\[\|f-s_n\|_{ \infty}\leq \frac c{n^2}\]
et donc que $s_n$ converge uniformément vers $f$ lorsque $n$ tend vers l'infini.

\end{enumerate}


\section{Moindre carrés discrets}
On a
\[\overline{x^k}=\frac{\sum_{i=1}^n{x_i^k}}n\qquad\mbox{et}\qquad\overline{x^ky}=\frac{\sum_{i=1}^n{x_i^ky_i}}n\]
On trouve 
\[\overline{x}=\frac 12\quad \overline{y}=-\frac 16\quad \overline{x^2}=\frac {19}6\quad   \overline{x^3}=\frac {9}2\quad \overline{x^4}=\frac {115}6\quad   \overline{xy}=-\frac {5}2\quad    \overline{x^2y}=\frac {19}6   \]
D'où le système
\[\left(\begin{array}{ccc}
1 & \frac 12 & \frac {19}6\\
\frac 12 & \frac {19}6 & \frac {9}2 \\
 \frac {19}6&\frac {9}2 & \frac {115}6 
\end{array}\right) \left(\begin{array}{c}
c \\
b \\
a 
\end{array}
\right) =\left(\begin{array}{c}
-\frac 16 \\
-\frac {5}2 \\
\frac {19}6
\end{array}
\right) 
\]
On trouve
\[a=\frac{55}{56}\qquad b=\frac{-507}{280}\qquad c=\frac{-83}{35}\]
D'où le polynôme:
\[P(x)=\frac{55}{56}x^2-\frac{507}{280}x-\frac{83}{35}\]
\begin{center}
 \begin{tikzpicture}[scale=1]
\draw  [very thin, gray] [->]  (-2.2,0) -- (3.2,0); 
\draw  [very thin, gray] [->] (0,-3.2) -- (0,5);
\draw  [dashed] (-2,0) -- (-2,5)--(0,5);
\draw  [dashed] (-1,0) -- (-1,1)--(0,1);
\draw  [dashed] (1,0) -- (1,-3)--(0,-3);
\draw  [dashed] (2,0) -- (2,-2)--(0,-2);
\draw  [dashed] (3,0) -- (3,1)--(0,1);
\path (-2,5) circle (.5mm) [fill=orange];
\path (-1,1) circle (.5mm) [fill=orange];
\path (0,-3) circle (.5mm) [fill=orange];
\path (1,-3) circle (.5mm) [fill=orange];
\path (2,-2) circle (.5mm) [fill=orange];
\path (3,1) circle (.5mm) [fill=orange];
\draw [domain=-2:3][line width=1] plot(\x,{55/56*\x*\x-507/280*\x-83/35});
\end{tikzpicture} 

\end{center}

%\section{Intégration numérique}
%
%\begin{enumerate}
%\item \[\int_1^2\frac 1x\de x=\ln 2\simeq 0.693\]
%\item On a $h=\frac 13$ et $x_i=1+i/3$ donc
%\[x_0=1,\, x_1=\frac 43,\,x_2=\frac 53,\, x_3=2\]
%\[\int_a^bf(x)\de x=\frac{b-a}{n}\left(\frac{f(a)+f(b)}{2}+\sum_{i=1}^{n-1}f(x_i)\right)\]
%\[\int_1^2\frac 1x\de x\simeq h\left(\frac{f(1)+f(2)}{2}+f(\frac 43)+f(\frac 53)\right)=\frac 13\left(\frac{1+\frac 12}{2}+\frac 34+\frac 35\right)=\frac 13(\frac 34+\frac 34+\frac 35)=\frac 7{10}=0.7\]
%\item La fonction $x\mapsto \frac 1x$ étant convexe, la méthode donne alors une valeur approchée par excès.
%\item \[\left|\frac{(b-a)^4}{12n^2}f''(\xi)\right|\leq 10^{-4}\]
%Donc 
% \[\left|\frac{1}{12n^2}2\right|\leq 10^{-4}\]
% \[n^2\geq 10000/6\]
% \[n\geq 100/\sqrt 6\]
% \[n\geq 41\]
%\end{enumerate}

\section*{Exercice n°4 : Intégration numérique (retranscrit par Léo Shi)}

On considère l'intégrale : $$I = \int^2_1 \dfrac{1}{x}\text{d}x$$

\begin{enumerate}
\item On calcule la valeur exacte de $I$ : $$ I = \int^2_1 \dfrac{1}{x}\text{d}x = \left[\ln (x)\right]^2_1 = \ln 2
$$
\item Par la méthode des trapèzes : 
\begin{center}
\begin{tikzpicture}
\draw[thick,-] (0,0) -- (6,0);

\draw (0,-1pt) -- (0,1pt) node[anchor=north] {$x_0=1$};
\draw (2,-1pt) -- (2,1pt) node[anchor=north] {$\dfrac{4}{3}$};
\draw (4,-1pt) -- (4,1pt) node[anchor=north] {$\dfrac{5}{3}$};
\draw (6,-1pt) -- (6,1pt) node[anchor=north] {$x_3=2$};
\draw[thick,<->] (2,0.5) -- (4,0.5) node[anchor=south] {$h = \dfrac{1}{3}$};
\end{tikzpicture}
\end{center}

$$
\int^b_a f(x) \text{d}x = h\left[ \dfrac{f(a)+f(b)}{2} + \sum^2_{i=1} f(x_i) \right] = \dfrac{1}{3}\left[ \dfrac{1+\frac{1}{2}}{2} + \dfrac{3}{4} + \dfrac{3}{5} \right] = \dfrac{7}{10} = 0,7
$$

\item On s'aide d'un dessin : 

\begin{center}
\begin{tikzpicture}
%\draw[step=1cm,gray,very thin] (-1,-1) grid (5,4);
\draw[thick,->] (0,0) -- (4,0) node[anchor=north] {$x$};
\draw[thick,->] (0,0) -- (0,3) node[anchor=west] {$y$};
\filldraw [fill=gray!20,draw=black]
(1,0) -- (1,1)
-- plot [domain=1:2] (\x,1/\x)
-- (2,0) -- cycle;
\draw [domain=0.4:3] plot(\x,1/\x) node[anchor=south west] {$y=\dfrac{1}{x}$};
\draw (0,-1pt) -- (0,1pt) node[anchor=north] {0};
\draw (1,-1pt) -- (1,1pt) node[anchor=north] {1};
\draw (2,-1pt) -- (2,1pt) node[anchor=north] {2};
\draw[thick,-, draw=blue] (1,1) -- (2,0.5);
\draw[thick,-, draw=blue] (2,0.5) -- (2,0);
\draw[thick,-, draw=blue] (1,1) -- (1,0);

\end{tikzpicture}

On a $x \mapsto \dfrac{1}{x}$ qui est \textbf{convexe}, donc la courbe est en dessous de la droite.
\end{center}

\item \fbox{$E=\dfrac{(b-a)^4}{12n^2}f''(\xi)$} avec $\xi\in ]a,b[$, on cherche $n$ tel que $E\leq 10^{-4}$. On a $b=2$ et $a=1$ :
$$
1<\xi<2 \implies \dfrac{1}{2}<\dfrac{1}{\xi} <1 \implies \dfrac{1}{8}<\dfrac{1}{\xi^3} <1
$$
$$
\dfrac{(2-1)^4}{12n^2}\times \dfrac{2}{\xi^3}\leq 10^{-4} \implies \dfrac{2}{8} \times \dfrac{1}{12n^2} \leq \dfrac{1}{12n^2} \times \dfrac{2}{\xi^3} \leq 10^{-4}
$$
D'où $\dfrac{1}{96n^2}\leq 10^{-4}$, soit \fbox{$n\geq 10$}
\end{enumerate}
\section*{5}
\section*{6}
\section*{7}
\section*{8}
\section*{9}
\section*{10}
\section*{Exercice n°11 : Formule de Quadrature(retranscrit par Léo Shi)}

Soit la formule de quadrature, avec $\alpha \in ]0,1[$ : 

\begin{align*}
\int^1_{-1} f(x) \text{d}x = w_1 f(-\alpha)+w_2 f(\alpha) \qquad (R)
\end{align*}

\begin{enumerate}
\item $(R)$ vraie sur $\mathbb{R}_n[X]$ si et seulement si elle est vraie sur la base $(1,X,X^2, ..., X^n)$, donc pour $n=1$, la relation $(R)$ est vraie si et seulement si elle est vraie sur la base $(1,X)$. 
\begin{itemize}
\item Pour 1, on a alors : 
\begin{align*}
\int^1_{-1}1 \text{d}x = w_1 + w_2 \implies w_1 + w_2 = 2
\end{align*}
\item Pour $X$, on a : 
\begin{align*}
\int^1_{-1} x \text{d}x = w_1(-\alpha) + w_2(\alpha) \implies w_1(-\alpha) + w_2(\alpha) = 0 \implies w_1 = w_2
\end{align*}
\end{itemize}

Soit finalement, on obtient $w_1 = w_2 = 1$, d'où: 
\begin{align*}
  \int^1_{-1} f(x) \text{d}x = f(-\alpha) + f(\alpha)
\end{align*}

\item Lorsque $\alpha = 1$, on a : 
\begin{align*}
\int^1_{-1} f(x) \text{d}x = f(-1) + f(1)
\end{align*}
Pour les polynômes de $\mathbb{R}_2[X] = \mathbb{R}_1[X] + X^2$, d'où pour $f(x)=x^2$ : 
\begin{align*}
\int^1_{-1} x^2 \text{d}x = \dfrac{2}{3}
\end{align*}
Or $f(-1) + f(1) = 2$ et $2 \neq  \dfrac{2}{3}$, donc \underline{la formule n'est pas exacte sur $\mathbb{R}_2[X]$}


\item On a $\dfrac{2}{3} = \alpha^2 \times 2 \implies \alpha = + \sqrt{\dfrac{1}{3}}$ car $\alpha > 0$ ($\alpha \in ]0,1[$). La formule exacte est alors : 
\begin{align*}
\int^1_{-1} f(x) \text{d}x = f\left(\dfrac{-1}{\sqrt{3}} \right) + f\left(\dfrac{1}{\sqrt{3}}\right)
\end{align*}

\item Si $f(x)=x^3$, alors 
\begin{align*}
\int^1_{-1} x^3 \text{d}x = 0 \qquad \text{et} \qquad \left( \dfrac{-1}{\sqrt{3}}\right)^3 + \left( \dfrac{1}{\sqrt{3}}\right)^3 = 0
\end{align*}

Donc \underline{la formule est exacte sur $\mathbb{R}_3[X]$}. La dernière question n'est pas toujours valable, vérifions donc la formule sur $\mathbb{R}_4[X]$ : 

\begin{align*}
\int^1_{-1} x^4 \text{d}x = \dfrac{2}{5} \qquad \text{et} \qquad \left( \dfrac{-1}{\sqrt{3}}\right)^4 + \left( \dfrac{1}{\sqrt{3}}\right)^4 = \dfrac{2}{9}
\end{align*}
Etant donné que $\dfrac{2}{9}\neq \dfrac{2}{5}$, la formule \textbf{n'est pas exacte} pour $\mathbb{R}_4[X]$ (comme prédit ...)

\item \textbf{On cherche une relation entre $\xi$ sur l'intervalle $[-1,1]$ et $x$ sur l'intervalle $[a,b]$}. On suppose que la relation entre $x$ et $\xi$ est \textbf{affine}: 

\begin{align*}
x=\lambda \xi + \mu
\end{align*}



$$
\left\{
    \begin{array}{l}
        a= -\lambda + \mu   \\ 
		b=  \lambda + \mu 
    \end{array}
\right. 
\Longleftrightarrow
\left\{
    \begin{array}{ll}
        \mu = \dfrac{a+b}{2} \\ 
		\lambda = \dfrac{b-a}{2}
    \end{array}
\right.
$$



On pose $x = \dfrac{a+b}{2} + \dfrac{b-a}{2} \xi$, on a donc avec $\xi = \pm \dfrac{1}{\sqrt{3}} $
\begin{align*}
\int^b_a f(x) \text{d}x & = \int^1_{-1} \left( \dfrac{b-a}{2} \right) f\left( \dfrac{a+b}{2} + \dfrac{b-a}{2} \xi\right) \text{d}\xi \\
& = \dfrac{b-a}{2} \left[ f\left( \dfrac{a+b}{2} + \dfrac{b-a}{2\sqrt{3}}\right) + f\left( \dfrac{a+b}{2} - \dfrac{b-a}{2\sqrt{3}}\right) \right]
\end{align*}

\end{enumerate}

\section*{12}{\bf Intégration numérique : }
\[\int_{-1}^1f(x)\de x\simeq\frac 43f\left(-\frac{\xi}{2}\right)+\frac 23f\left(\xi\right)\]
\begin{enumerate}
\item La formule est exacte pour $f(x)=1$ et $f(x)=x$. Pour $f(x)=x^2$ nous avons
\[\frac 23=\xi^2\]
donc $\xi=\sqrt{\frac{2}{3}}$ et avec cette valeur la formule n'est plus exacte pour $f(x)=x^3$ donc la méthode est d'ordre $2$.
\item On fait $x=\lambda t+\mu$. En $x=a$, $t=-1$ et en $x=b$, $t=1$:
\[a=-\lambda+\mu\]
\[b=\lambda +\mu\]
D'où le changement de variable $x=\frac{a+b}{2}+t\frac{b-a}{2}$:
\[\int_{a}^bf(x)\de x=\frac{b-a}{2}\int_{-1}^1f(\frac{a+b}{2}+t\frac{b-a}{2})\de t\]
\[\int_{a}^bf(x)\de x\simeq\frac{b-a}{2}\left[\frac 43f\left(\frac{a+b}{2}-\frac{\xi}{2}\frac{b-a}{2}\right)+\frac 23f\left(\frac{a+b}{2}+\xi\frac{b-a}{2}\right)\right]\]
\item Nous avons
\[\int_{x_{2i-1}}^{x_{2i+1}}f(x)\de x\simeq h\left[\frac 43f\left(x_{2i}-\frac{\xi}{2}h\right)+\frac 23f\left(x_{2i}+\xi h\right)\right]\]
\[\int_{x_{2i-1}}^{x_{2i+1}}f(x)\de x\simeq\frac{2h}{3}\left[2f\left(x_{2i}-\frac{h}{\sqrt 6}\right)+f\left(x_{2i}+\sqrt{\frac 23} h \right)\right]\]
où $x_k=a+kh=-1+kh$, $h=\frac{b-a}{2n}$, 
$x_{2i+1}=-1+(2i+1)h$ et $x_{2i-1}=-1+(2i-1)h$


D'où
\[\int_{a}^bf(x)\de x=\sum_{i=1}^{n}\int_{x_{2i-1}}^{x_{2i+1}}f(x)\de x\simeq\frac{b-a}{3n}\sum_{i=1}^{n}\left[2f\left(a+(i-\frac 1{2\sqrt 6})\frac{b-a}{n}\right)+f\left(a+(i+\frac 1{\sqrt{6}})\frac{b-a}{n}\right)\right]\]
\end{enumerate}
%%%%%%%%%%%%%%%%%%%%%%%%%%%%%%%%%%%%%%%%%%%%%%%%%%%%%%%%%%%%%%%%%%%%%%%%%%%%%%%

\section*{13}{\bf Formule de Quadrature : }

\[\int_0^1 f(x) \de x \simeq \omega_0 f(0) +  \omega_1f'(0) +  \omega_2f'(\xi)\]
\[E(f) =\int_0^1 f(x) \de x -\left[ \omega_0 f(0) +  \omega_1f'(0) +  \omega_2f'(\xi)\right]\]
\begin{enumerate}
\item On trouve $\xi=\frac 12$, $\omega_0=1$, $\omega_1=\frac 16$ et $\omega_2=\frac 13$.
\item  
\[E(f) =\int_0^1 f(x) \de x -\left[  f(0) +  \frac 16 f'(0) +  \frac 13f'(\frac 12)\right]\]
 $E(x \mapsto x^4)=\frac 1{30}$ donc la méthode est d'ordre 3. 

\item  On fait le changement de variable $x=a+t(b-a)$:
\[\int_{a}^bf(x)\de x=(b-a)\int_{0}^1f(a+t(b-a))\de t\]
\[\int_{a}^bf(x)\de x\simeq(b-a)\left[f(a) +  \frac 16 f'(a) +  \frac 13f'(\frac {a+b}2)\right]\]
\end{enumerate}


\section*{14}
\begin{enumerate}
\item \begin{enumerate}
\item Pour que la formule soit exacte pour $f(x) = 1$, il faut $\omega_1+\omega_2 = 1$. Et pour qu'elle soit exacte pour $f(x) = x$, il faut $\omega_2\alpha =\frac 12$. Ce qui amène à $\omega_1 = 1 -\frac{1}{2\alpha}$ et $\omega_2 = \frac{1}{2\alpha}$.
\item Pour qu'elle soit exacte pour $f(x) = x^2$, il faut $\omega_2\alpha^2 =\frac 13$, donc $\alpha=\frac 23$. Et donc $\omega_1 = \frac{1}{4}$ et $\omega_2 = \frac{3}{4}$.
On a en posant $x=hu$,
\[\int_0^hf(x)\de x = h\int_0^1f(hu)\de u \simeq \frac h4\left[f(0)+3f(\frac 23h)\right]\]
\end{enumerate}
\item\begin{enumerate}
\item On applique la formule de Taylor-Lagrange à l'ordre 2 en 0 :
\[f(x)=f(0)+f'(0)x+\frac 12 f''(0)x^2+\frac 16f'''(c)x^3\]
avec $c$ compris entre 0 et $x$. On pose
\[P(x)=f(0)+f'(0)x+\frac 12 f''(0)x^2\]
qui est bien polynômial de degré au plus 2 et $R(x)=\frac 16f'''(c)x^3$. On vérifie aisément l'inégalité voulue.
\item De la majoration $|R(x)|\leq \frac{M_3}{6} x^3$, on tire :
\[\left|\int_0^hR(x)\de x\right|\leq \int_0^h|R(x)|\de x\leq \int_0^h\frac{M_3}{6} x^3\de x=\frac{M_3 h^4}{24}\]
Et, en tenant de compte que $R(0) = 0$ (inégalité pour $x = 0$), on a aussi :
\[|Q(R) = \left|\frac h4R(0)+\frac{3h}{4}R(\frac 23h)\right|\leq \frac{3h}{4}\frac{M_3}{6}(\frac 23h)^3 = \frac{M_3 h^4}{27}\]
On a enfin
\[|E(R)|=|I(R)-Q(R)|\leq |I(R)|+|Q(R)|\leq \frac{17}{216}M_3 h^4\]
\item Par linéarité, $I(f) = I(P) + I(R)$ et $Q(f) = Q(P) + Q(R)$. Comme la
formule est exacte pour $P$ (puisqu'elle l'est sur tout polynôme de degré 2), $I(P) =Q(P)$ et donc $E(f) = E(R)$. Et donc
\[|E(f)|\leq \frac{17}{216}M_3 h^4\]
\end{enumerate}
\item\begin{enumerate}
\item On pose $x_i = a + ih$. On a
\[\int_a^bf(x)\de x = \sum_{i=0}^{N-1}\int_{x_i}^{x_i+h}f(x)\de x =\sum_{i=0}^{N-1}\int_{0}^{h}f(x_i+u)\de u \simeq \frac h4\sum_{i=0}^{N-1} \left[f(x_i)+3f(x_i+\frac 23h)\right]\]  
\item On note $M_3 = \displaystyle \sup_{a\leq x\leq b}|f^{(3)}(x)|$.

L'erreur sur chaque intervalle sera majorée par $\frac{17}{216}M_3 h^4$  et donc l'erreur totale par $\frac{17}{216}M_3 N h^4=\frac{17}{216}M_3 (b-a) h^3$ .

\end{enumerate}
\end{enumerate}
\section*{15}
\section*{16}




\section*{17}{\bf Polynômes orthogonaux : }
\begin{enumerate}
\item Par intégration par parties
\item On trouve $\|P_n\|^2=\frac{2}{2n+1}$
\item \[\begin{array}{ccl}
P_3(X)&=&\frac{1}{3!2^3}\frac{\de^3}{\de X^3}\left[(X^2-1)^3\right]\\
&=&\frac{1}{6\times 8}\frac{\de^2}{\de X^2}\left[3(X^2-1)^2\times 2X\right]\\
&=&\frac{1}{8}\frac{\de}{\de X}\left[2(X^2-1)\times 2X^2+(X^2-1)^2\right]\\
&=&\frac{1}{8}\frac{\de}{\de X}\left[5X^4-6X^2+1\right]\\
&=&\frac{1}{2}\left(5X^3-3X\right)
\end{array}
\]
\item Les points $\xi_i$, $i=1,2,3$ sont les racines de $P_3$ d'où 
\[\xi_1=-\sqrt{\frac{3}{5}}, \qquad \xi_2=0, \qquad  \xi_3=\sqrt{\frac{3}{5}}\]

 Les 3 poids  associés se calculent par la formule :
 \[\forall i =1,2,3 \quad \omega_i=\frac{2}{(1-\xi_i^2)P'_n(\xi_i)^2}\]
 D'où
 \[\omega_1 = \frac 59,\quad  \omega_2 = \frac 89,\quad  \omega_3 = \frac 59\]
Et donc

\[\int_{-1}^1f(\xi)\de \xi \simeq  \frac 19\left[5f(-\sqrt{\frac{3}{5}})+ 8f(0)+ 5f(\sqrt{\frac{3}{5}})\right]\]
Par changement de variable $x=\frac{a+b}{2}+\frac{b-a}{2}\xi$:
\[\int_a^bf(x)\de x \simeq  \frac{b-a}{18}\left[5f(\frac{a+b}{2}-\sqrt{\frac{3}{5}}\frac{b-a}{2})+ 8f(\frac{a+b}{2})+ 5f(\frac{a+b}{2}+\sqrt{\frac{3}{5}}\frac{b-a}{2})\right]\]
et si $x_i=a+ih$ où $h=\frac{b-a}{n}$:
\[\int_{x_i}^{x_{i+1}}f(x)\de x \simeq  \frac{h}{18}\left[5f(x_{i+\frac{1}{2}}-\sqrt{\frac{3}{20}}h)+ 8f(x_{i+\frac{1}{2}})+ 5f(x_{i+\frac{1}{2}}+\sqrt{\frac{3}{20}}h)\right]\]
Finalement
\[\int_a^bf(x)\de x \simeq  \frac{b-a}{18 n}\sum_{i=0}^{n-1}\left[5f(a+(i-0.1127)\frac{b-a}{n})+ 8f(a+(i+0.5)\frac{b-a}{n})+ 5f(a+(i+0.8873)\frac{b-a}{n})\right]\]
\end{enumerate}


\end{document}