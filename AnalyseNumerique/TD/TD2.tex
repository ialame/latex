\documentclass[a4paper]{article} 
\usepackage[francais]{babel}
\usepackage[utf8]{inputenc} % Required for including letters with accents
\usepackage[T1]{fontenc} % Use 8-bit encoding that has 256 glyphs

\usepackage{amsthm}
\usepackage{amsmath}
\usepackage{amssymb}
\usepackage{mathrsfs}
\usepackage{graphicx}
\usepackage{geometry}
\usepackage{stmaryrd}
\usepackage{tikz}

\def \de {{\rm d}}



\title{TD 2 Analyse numérique (B1-TP1)}
\author{Ibrahim ALAME}
\date{19/02/2024}
\begin{document}
\maketitle

%\section{Schéma d'Euler explicite}
%On considère le problème de Cauchy suivant
%\[\left\{\begin{array}{ccl}
%x'(t)&=&f(t,x(t)), \qquad t\in [t_0,t_0 + T]\\
%x(t_0)&=&x_0
%\end{array}\right.\]
%
%où $t_0$, $T$, $x_0 \in \mathbb{R}$ et $f : [t_0, t_0 + T]\to\mathbb{R}$ sont donnés.
%
%On suppose de plus qu'il existe $L > 0$ tel que pour tout $t \in [t_0, t_0 + T]$, et pour tous $x$, $y \in \mathbb{R}$,
%\[|f(t, x)-f(t, y)| \leq L|x-y|\]
%
%\begin{enumerate}
%\item  Donner le schéma d'Euler explicite à pas de temps constant correspondant à ce problème.
%\item Jusqu'à quel ordre ce schéma est-t-il convergent ?
%\item Applications :
%pour les deux problèmes suivants :
%\[A.\left\{\begin{array}{ccl}
%x'(t)&=&t\sin(x(t)), \qquad t\in [0, T]\\
%x(0)&=&\frac {\pi}2
%\end{array}\right. \qquad B.\left\{\begin{array}{ccl}
%x'(t)&=&t^2+x(t)+1, \qquad t\in [1, T]\\
%x(1)&=&1
%\end{array}\right.\]
%
%\begin{enumerate}
%\item   Écrire le schéma d'Euler explicite en prenant un pas de temps constant.
%\item  Écrire les 2 premières itérations en prenant comme pas de temps h = 0.1.
%\item  Est-ce que ce schéma converge vers chacune des solutions de ces problèmes ?
%\end{enumerate}
%\end{enumerate}
%

\section{Méthodes d'Euler explicite, implicite et Runge-Kutta}
Soit le problème de Cauchy suivant
\[({\cal P})\left\{\begin{array}{l}
y'(t)=f(t,y(t)),\quad t\in [0,1],\\
y(0)=1
\end{array}\right.\]
où $f(t, y) = 3t + y$.
\begin{enumerate}
\item  
\begin{enumerate}
\item  Montrer que la fonction f est lipschitzienne par rapport à la deuxième variable et
donner une constante de Lipschitz.
\item Que peut-on dire sur l'existence et l'unicité du problème (P) ?
\end{enumerate}
\item Montrer que $y(t) = 4e^t - 3t - 3$ est l'unique solution de$({\cal P})$.
\item Écrire le schéma d'Euler explicite à ce problème, avec $h = 0.1$, puis évaluer la solution en $t = 0.2$.
\item Écrire le schéma d'Euler implicite à ce problème, avec $h = 0.1$, puis évaluer la solution en $t = 0.2$.
\item Écrire la méthode de Runge-Kutta d'ordre 2 et donner l'approximation de $y(0.2)$ à l'aide d'un pas de discrétisation numérique $h = 0.1$.
\item  Comparer les solutions numériques obtenues par chaque méthode à la valeur exacte.
\end{enumerate}

\section{Majoration de l'erreur}
 On étudie l'équation différentielle $y' = -y$.
Cette équation sera considérée sur $[0; 10]$ avec la condition initiale $y(0) = 1$.
\begin{enumerate}
\item   Déterminer la solution exacte de ce problème de Cauchy. Vérifier qu'on a toujours
\[\forall t\in [0,10],\quad 0<y(t)\leq 1\]
\item   Exprimer pour la méthode d'Euler explicite, l'expression de $y_n$ en fonction de $n$.
\item  Déterminer à quelle condition sur $h$, on peut assurer que pour tout $n$, $0<y_n\leq 1$. 
\item Majorer sommairement $\max_{0\leq n\leq N}\left|y_n-y(t_n)\right|$. (On se limitera au cas où $h\leq\frac 12$).
\item  Reprendre les questions 2 à 4 avec les autres méthodes vues (Euler implicite, Runge-
Kutta d'ordres 2 et 4). Retrouve-t-on des résultats compatibles avec les ordres connus
pour ces méthodes ?
\end{enumerate}

%%%%%%%%%%%%%%%%%%%%%%%%%%%%%%%%%%%%%%%%%%%%%%%%%%%%%%%%%%%%%%%%%%%%%%%%%%%%%%%%%%%
\section{Erreur de la méthode d'Euler}
Résoudre explicitement l'équation $y'=-y+t$ avec condition initiale $y(0) = 0$. Utiliser la méthode d'Euler pour donner la valeur de l'erreur en t = 1.
%%%%%%%%%%%%%%%%%%%%%%%%%%%%%%%%%%%%%%%%%%%%%%%%%%%%%%%%%%%%%%%%%%%%%%%%%%%%%%%%%%%%%%%%%%
% Soit le problème de Cauchy suivant
%\[({\cal P})\left\{\begin{array}{l}
%y'(t)=f(t,y(t)),\quad t\in [0,1],\\
%y(0)=1
%\end{array}\right.\]
%où $f(t; y) = -2y+e^{-t}$.
%
%\begin{enumerate}
%\item Montrer que la fonction $f$ est lipschitzienne par rapport à la deuxième variable et donner une constante de Lipschitz.
%\item Montrer que ce problème admet une solution unique.
%\item Montrer que $y(t) = e^{-t}$ est l'unique solution de ce problème.
%\item Écrire la méthode d'Euler explicite associée à ce problème et donner l'approximation $y(0.3)$ à l'aide d'un pas de discrétisation numérique $h = 0.1$.
%\item Écrire la méthode d'Euler implicite associée à ce problème et donner l'approximation $y(0.3)$ à l'aide d'un pas de discrétisation numérique $h = 0.1$.
%\item Écrire la méthode de Runge-Kutta d'ordre 2 et donner l'approximation de $y(0.3)$ à l'aide d'un pas de discrétisation numérique $h = 0.1$.
%\item Comparer les solutions numériques obtenues par chaque méthode à la valeur exacte.
%\end{enumerate}

%%%%%%%%%%%%%%%%%%%%%%%%%%%%%%%%%%%%%%%%%%%%%%%%%%%%%%%%%%%%%%%%%%%%%%%%%%%%%%%%%%%
\section{Équation différentielle du second ordre}
 Soit l'équation différentielle du second ordre $y'' + ty' + (1 - t)y = 2$, considérée
sur l'intervalle $I = [0; 1]$ assortie des conditions initiales $y(0) = 0$ et $y'(0) = 0$.
\begin{enumerate}
\item  Reformuler cette équation différentielle sous la forme d'un problème de Cauchy.
\item Montrer que ce problème a une et une seule solution.
\item Exposer la formulation de la méthode d'Euler explicite pour ce problème. Calculer, pour $h = 0.1$ la valeur approchée obtenue comme approximation de $y(0.3)$.
\item Exposer la formulation de la méthode d'Euler implicite pour ce problème.
\end{enumerate}


%%%%%%%%%%%%%%%%%%%%%%%%%%%%%%%%%%%%%%%%%%%%%%%%%%%%%%%%%%%%%%%%%%%%%%%%%%%%%%%%%%%%%%
\section{Équation différentielle non linéaire}
 On considère le problème de Cauchy suivant (avec $T > 0$) :
\[({\cal P})\left\{\begin{array}{l}
y'(t)= \sin y(t) + \sin t,\quad \forall t\in[0,T]\\
y(0)=0
\end{array}\right.\]
\begin{enumerate}
\item  Montrer que ce problème de Cauchy a une et une seule solution.
\item  Majorer sommairement $|y''|$ sur $[0; T]$.
\item  Déterminer une majoration de l'erreur $\max_{0\leq n\leq N}\left|y_n-y(t_n)\right|$
 pour la méthode d'Euler explicite.
\item  Pour $T = 1$, quel valeur de $N$ choisir pour garantir un résultat correct avec une précision de $10^{-3}$ ? Même question pour $T = 10$.
\end{enumerate}

%\end{document}
%%%%%%%%%%%%%%%%%%%%%%%%%%%%%%%%%%%%%%%%%%%%%%%%%%%%%%%%%%%%%%%%%%%%%%%%%%%%%%%%%%%

\section{Méthode à un pas}
 On considère un problème de Cauchy $y' = f(t; y)$ avec $y(a) = \alpha$.
La méthode de Heun est une méthode numérique à un pas où le calcul de $y_{n+1}$ à partir de $y_n$ est décrite par :
\[({\cal P})\left\{\begin{array}{l}
\overline{y}_n= y_n + hf(t_n; y_n)\\
y_{n+1} = y_n +\frac h2\left(f(t_n,y_n) + f(t_{n+1},\overline{y}_n)\right)
\end{array}\right.\]

\begin{enumerate}
\item  Expliquer en quoi on peut affirmer que cette méthode est inspirée de la méthode des
trapèzes.
\item   Montrer que cette méthode est consistante.
\item  Montrer que cette méthode est stable.
\end{enumerate}

%%%%%%%%%%%%%%%%%%%%%%%%%%%%%%%%%%%%%%%%%%%%%%%%%%%%%%%%%%%%%%%%%%%%%%%%%%%%%%%%%%%

\section{Méthode de Heun}
La méthode de Heun consiste à approcher, la solution de l'équation différentielle $y'(t) = f (t, y(t))$,  $y(0) = y0$, par le schéma suivant:
\[\left\{\begin{array}{l}
y_{n+1}=y_n+\frac h2\left[f(t_n,y_n)+f(t_n+h,y_n+hf(t_n,y_n))\right]\\
y_0 \mbox{ donné}
\end{array}\right.
\]
Où $h > 0$ un pas de temps donné, $t_n = nh$ pour $n \in\mathbb{N}$  et $y_n$ une approximation de $y(t_n)$.

\begin{enumerate}
\item Montrer que la méthode est consistante. Déterminer son ordre. On considère le problème de CAUCHY

\[(\star)\left\{\begin{array}{l}
y'(t)=-\alpha y(t)+\beta \mbox{ pour }t>0\\
y(0)=y_0
\end{array}\right.
\]
où $y_0$ est une valeur donnée.

\item Monter que la méthode d'Euler modifiée appliquée au problème de Cauchy $(\star)$ converge.
\item Écrire le schéma permettant de calculer $y_{n+1}$ à partir de $y_n$. Sous
quelle condition sur $h$ la relation 
\[\lim_{n\to\infty}y_n=0\]
 a-t-elle lieu ?
\end{enumerate}

%%%%%%%%%%%%%%%%%%%%%%%%%%%%%%%%%%%%%%%%%%%%%%%%%%%%%%%%%%%%%%%%%%%%%%%%%%%%%%%%%%%

\section{Asymptotique, raideur}
Soit $a > 0$, $b \in \mathbb{R}$ et $x_0 \in \mathbb{R}$. On considère le problème de Cauchy suivant
\[A.\left\{\begin{array}{ccl}
x'(t)&=&-a x(t)+b, \qquad t\in \mathbb{R}^+\\
x(0)&=&x_0
\end{array}\right.
\]
\begin{enumerate}
\item  Donner explicitement $x$.
\item  Soit $h > 0$ un pas de temps.
\begin{enumerate}
       \item  Écrire explicitement le schéma d'Euler explicite à pas  $h$. 
\item  On suppose que $h$ est une constante. Donner explicitement $(x_n)_{n\in \mathbb{N}}$.
\item   Quelle condition doit satisfaire $h$ pour que, quel que soit $x_0$, $x_n$ tende quand $n \to +\infty$ vers $\frac ba$?
\item  On suppose que $h=\frac{t}{n}$. Montrer que quel que soit $x_0$, $x_n$ tend quand $n \to +\infty$ vers $x(t)$.
 Exprimer en fonction de $a$, $b$, $x_0$ et $n$ l'erreur d'approximation dans un calcul approché de $x_{/[0,10]}$.
\end{enumerate}


\end{enumerate}




%\section{Crank-Nicolson, Heun}
%On considère le problème de CAUCHY
%
%\[\left\{\begin{array}{l}
%y'(t)=-\alpha y(t) \mbox{ pour }t>0\\
%y(0)=y_0
%\end{array}\right.
%\]
%où $y_0$ est une valeur donnée. Soit $h > 0$ un pas de temps donné, $t_n = nh$ pour $n \in\mathbb{R}$  et $u_n$ une approximation de $y(t_n)$.
%\begin{enumerate}
%\item  Écrire le schéma du trapèze (appelé aussi de CRANK-NICOLSON) permettant de calculer $u_{n+1}$ à partir de $u_n$. Sous
%quelle condition sur $h$ le schéma du trapèze est-il A-stable ? Autrement dit, pour quelles valeurs de $h$ la relation 
%\[\lim_{n\to\infty}u_n=0\]
% a-t-elle lieu ?
%\item À partir du schéma du trapèze, en déduire le schéma de HEUN. Sous quelle condition sur $h$ le schéma de HEUN est-il A-stable ?
%\end{enumerate}
\section{Méthode d'Euler en dimension 2}
 On considère le système différentiel
\[(S) \left\{\begin{array}{lll}
x'(t)&=&-y(t)\\
y'(t)&=&x(t)
\end{array}\right.\quad t\in [0,T]
\]
 que l'on note aussi $X'(t) = AX(t)$, avec
 \[X(t)= \left(\begin{array}{c} x(t)\\ y(t) \end{array} \right) \quad \mbox{ et }\quad
 A = \left(\begin{array}{cc} 0& -1 \\ 1&0 \end{array} \right)\]
\begin{enumerate}
\item Calculer les valeurs propres de la matrice $A$, et déterminer la solution $X(t)$ du système $(S)$ qui satisfait aux conditions initiales $x(0) = 1, y(0) = 0$, {\em (Indication: $x''+x=0$ et $y''+y=0$)}. Tracer l'orbite de $X(t)$, c'est-à-dire la courbe $t \geq 0 \to X(t) \in \mathbb{R}^2$.
\item Pour un entier $N \geq 2$ fixé, on pose $h = T/N$ et l'on applique au système $(S)$ le schéma d'Euler explicite avec le pas $h$ et la donnée initiale $X^0 = (x_0,y_0)^T = (1,0)^T$. Écrire explicitement la relation de récurrence qui définit la suite $X^n = (x_n,y_n)^T$.
\item On note $E^k$ l'erreur $X^k - X(kh)$. Établir la relation 
\[ E^{k+1} = E^k+h A E^k + \frac{h^2}{2}X(kh)+O(h^3), \quad  0\leq k\leq N-1\]
{\em (Indication: faire un DL de $X((k+1)h)$ au voisinage de $kh$ à l'ordre 2)}

 Montrer qu'il existe des constantes $C_1$ et $C_2$ telles que
\[ \|E^{k+1}\| \leq (1+h C_1)\|E^k\| + C_2 h^2,\quad 0\leq n\leq N-1\]
où  $\|\cdot\|$ désigne la norme euclidienne dans $\mathbb{R}^2$. {\em (Indication: inégalité triangulaire avec $\|O(h^3)\|\leq M h^3\leq M h^2$ pour $h$ assez petit)}. En déduire l'existence d'une constante $C$, dépendant de $T$ mais non de $N$, $h$, telle que $\|E^n\| \leq  C h$ pour tout $0 \leq n \leq N$ {\em (Indication: diviser les deux membres de l'inégalité par $(1+h C_1)^{k+1}$ puis appliquer la somme $\sum_{k=0}^{n-1}$. Remarquer que $(1+hC_1)^n\leq (e^{hC_1})^N$, et $\sum_{k=0}^{n-1}\frac{h^2}{(1+h C_1)^{k+1}}\leq \sum_{k=0}^{N-1}h^2 =Nh^2$)}. Étudier la variation de $C$ en fonction de $T$.
\item On fixe maintenant $h > 0$. Montrer que le schéma d'Euler peut s'écrire $X^{n+1} = A_h X^n$, où $A_h$ est une matrice que l'on précisera.  Déterminer les valeurs propres de $A_h$ ainsi que leurs modules.
\item On pose $h=\tan\theta$. Montrer que $\cos\theta A_h$ est une matrice de rotation. En déduire $A_h^n$. calculer $X^n$ pour $0 \leq n \leq N$. Étudier la limite de $\|X^n\| $ quand $n\to\infty$. Esquisser la ligne brisée qui joint les points $(x_0, y_0),\, (x_1, y_1),\, . . .\, , (x_N , y_N )$, et qui est une approximation de l'orbite $X(t)$. Obtient-on une courbe fermée ? Tracer la ligne brisée dans les deux cas particuliers: $\theta=\frac{\pi}{4}$, $N=8$ et $\theta=\frac{\pi}{6}$, $N=12$.
\end{enumerate}
%%%%%%%%%%%%%%%%%%%%%%%%%%%%%%%%%%%%%%%%%%%%%%%%%%%%%%%%%%%%%%%%%%%%%%%%%%%%%%%%%%%%%%
\section{Système différentiel  d'ordre 2}
 Soit le système différentiel à deux inconnues $y(t)$ et $z(t)$ considéré sur un intervalle $[0; T]$ suivant
\[
\left\{\begin{array}{l}
y''-ty'+2z=t\\
y'+e^ty+3z'+2z=4t^2+1
\end{array}\right.
\]
\begin{enumerate}
\item Quelles conditions initiales poser pour constituer avec ce système un problème de Cauchy? L'exprimer sous la forme $Y' = F(t,Y)$.
\item Justifier que ce problème admet une et une seule solution.
\item Exposer le principe des méthodes d'Euler explicite et implicite pour ce système.
\end{enumerate}
%%%%%%%%%%%%%%%%%%%%%%%%%%%%%%%%%%%%%%%%%%%%%%%%%%%%%%%%%%%%%%%%%%%%%%%%%%%%%%%%%%%%%%%

\end{document}