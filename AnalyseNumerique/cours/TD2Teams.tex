\documentclass{beamer}
\usepackage[francais]{babel}
\usepackage[utf8]{inputenc} % Required for including letters with accents
\usepackage[T1]{fontenc} % Use 8-bit encoding that has 256 glyphs
\usepackage{pythontex}
\usepackage{amsthm}
\usepackage{amsmath}
\usepackage{amssymb}
\usepackage{mathrsfs}
\usepackage{graphicx}
\usepackage{geometry}
\usepackage{stmaryrd}
\usepackage{tikz} 
\usetikzlibrary{matrix,decorations.pathreplacing,calc,fit,backgrounds}
\usetikzlibrary{patterns}
%\usetikzlibrary{intersections}
\usepackage {mathtools} 
%%%%%%%%%%%%%%%

%\usepackage[latin1]{inputenc}
%\usepackage[T1]{fontenc}
%\usepackage[francais]{babel}
\usepackage[cache=false]{minted}


%%%%%%%%%%%%%%%


\usepackage{stmaryrd}
%\usepackage{tikz}
%\usetikzlibrary{tikzmark}
\usepackage{empheq}
\usepackage{longtable}
\usepackage{booktabs} 
\usepackage{array}
\usepackage{pstricks}
\usepackage{pst-3dplot}
\usepackage{pst-tree}
\usepackage{pstricks-add}
\usepackage{upgreek}
%\usepackage{epstopdf}
\usepackage{eolgrab}
\usepackage{chngpage}
 \usepackage{calrsfs}
 % Appel du package pythontex 
\usepackage{pythontex}

\usetikzlibrary{decorations.pathmorphing}
\def \de {{\rm d}}
\usepackage{color}
%\usepackage{xcolor}
%\usepackage{textcomp}
\newcommand{\mybox}[1]{\fbox{$\displaystyle#1$}}
\newcommand{\myredbox}[1]{\fcolorbox{red}{white}{$\displaystyle#1$}}
\newcommand{\mydoublebox}[1]{\fbox{\fbox{$\displaystyle#1$}}}
\newcommand{\myreddoublebox}[1]{\fcolorbox{red}{white}{\fcolorbox{red}{white}{$\displaystyle#1$}}}
\usetheme[options]{Boadilla}
\definecolor{purple2}{RGB}{153,0,153} % there's actually no standard purple
\definecolor{green2}{RGB}{0,153,0} % a darker green
\usepackage{xcolor}
%\setbeamercolor{background canvas}{bg=lightgray}
\usepackage{listings}
\definecolor{purple2}{RGB}{153,0,153} % there’s actually no standard purple
\definecolor{green2}{RGB}{0,153,0} % a
\lstset{%
language=Python, % 
basicstyle=\normalsize\ttfamily, % 
% Color settings to match IDLE style 
keywordstyle=\color{orange}, % 
keywordstyle={[2]\color{purple2}}, % 
stringstyle=\color{green2}, 
commentstyle=\color{red}, 
upquote=true, %
}
\lstdefinestyle{Python}{
    language        = Python,
    basicstyle      = \ttfamily,
    keywordstyle    = \color{blue},
    keywordstyle    = [2] \color{teal}, % just to check that it works
    stringstyle     = \color{violet},
    commentstyle    = \color{red}\ttfamily
}
\usepackage{algorithm2e}
\RestyleAlgo{algoruled}
  \SetKw{KwFrom}{from} 
\newenvironment{algo}{
\begin{algorithm}[H]
\DontPrintSemicolon \SetAlgoVlined}
{\end{algorithm}}

\usepackage{varwidth}

\usepackage{etex} 
\usepackage{easybmat} 
\usepackage{lmodern} 


%\tikzset{% 
%  highlight/.style={rectangle,rounded corners,fill=orange!35,draw,thick,inner sep=2pt} 
%} 
%\newcommand{\tikzmark}[2]{\tikz[remember picture,baseline=(#1.base),inner sep=0,outer sep=0pt] \node (#1) {#2};} 



  \title{Analyse numérique matricielle}
  \author{ \textsc{Ibrahim ALAME}}\institute{ESTP}
\date{01/03/2023}
  \begin{document}
  \lstset{
    frame       = single,
    numbers     = left,
    showspaces  = false,
    showstringspaces    = false,
    captionpos  = t,
    caption     = \lstname
}
\pgfmathsetmacro{\myscale}{2}
\pgfkeys{tikz/mymatrixenv/.style={decoration={brace},every left delimiter/.style={xshift=8pt},every right delimiter/.style={xshift=-8pt}}}
\pgfkeys{tikz/mymatrix/.style={matrix of math nodes,nodes in empty cells,
left delimiter={[},right delimiter={]},inner sep=1pt,outer sep=1.5pt,
column sep=8pt,row sep=8pt,nodes={minimum width=20pt,minimum height=10pt,
anchor=center,inner sep=0pt,outer sep=0pt,scale=\myscale,transform shape}}}
\pgfkeys{tikz/mymatrixbrace/.style={decorate,thick}}

\newcommand*\mymatrixbraceright[4][m]{
    \draw[mymatrixbrace] (#1.west|-#1-#3-1.south west) -- node[left=2pt] {#4} (#1.west|-#1-#2-1.north west);
}
\newcommand*\mymatrixbraceleft[4][m]{
    \draw[mymatrixbrace] (#1.east|-#1-#2-1.north east) -- node[right=2pt] {#4} (#1.east|-#1-#2-1.south east);
}
\newcommand*\mymatrixbracetop[4][m]{
    \draw[mymatrixbrace] (#1.north-|#1-1-#2.north west) -- node[above=2pt] {#4} (#1.north-|#1-1-#3.north east);
}
\newcommand*\mymatrixbracebottom[4][m]{
    \draw[mymatrixbrace] (#1.south-|#1-1-#2.north east) -- node[below=2pt] {#4} (#1.south-|#1-1-#3.north west);
}


\tikzset{greenish/.style={
    fill=green!50!lime!60,draw opacity=0.4,
    draw=green!50!lime!60,fill opacity=0.1,
  },
  cyanish/.style={
    fill=cyan!90!blue!60, draw opacity=0.4,
    draw=blue!70!cyan!30,fill opacity=0.1,
  },
  orangeish/.style={
    fill=orange!90, draw opacity=0.8,
    draw=orange!90, fill opacity=0.3,
  },
  brownish/.style={
    fill=brown!70!orange!40, draw opacity=0.4,
    draw=brown, fill opacity=0.3,
  },
  purpleish/.style={
    fill=violet!90!pink!20, draw opacity=0.5,
    draw=violet, fill opacity=0.3,    
  }}



 \begin{frame}
 \begin{center}
 Chapitre 3
 \end{center}
  \titlepage
  \end{frame}

\begin{frame}
\frametitle{Exercice 1}
\begin{enumerate}
\item \[f(t,y)=3t+y\]
\[|f(t,y_1)-f(t,y_2)|=|y_1-y_2|\]
Donc $f$ est 1-Lipschitzienne. D'après le cours $f$ est Lipschitzienne donc le problème de Cauchy admet une et une seule solution.
\item  $y(t)=4 e^t-3t-3 $ vérifie bien l'équadiff $y'=3t+y$ et $y(0)=1$.
\item \[\left\{\begin{array}{l}
y_{k+1}=y_k+hf(t_k,y_k)\\
y_0=1
\end{array}\right.\]
 \[\left\{\begin{array}{l}
y_{k+1}=y_k+h(3t_k+y_k)\\
y_0=1
\end{array}\right.\]
\[\left\{\begin{array}{l}
y_{k+1}=(1+h)y_k+3ht_k\\
y_0=1
\end{array}\right.\]

\end{enumerate}


\end{frame}

\begin{frame}
\frametitle{Exercice 1}
En général $y(t_k)\simeq y_k$

On $t_k=a+k\times h=0+k\times 0.1 \Longrightarrow 0.2=t_2$

On cherche donc  $y(0.2)\simeq y_2$
\[y_{1}=(1+h)y_0+3ht_0\]
\[y_1=1.1\times 1+3\times 0.1\times 0=1.1\]
\[y_{2}=(1+h)y_1+3ht_1\]
\[y_2=1.1\times 1.1+3\times 0.1\times 0.1=1.24\]
\end{frame}

%%%%%%%%%%%%%%%%%%%%%%%%%%%%%%%%%%%%%%%%%%%%%%%%%%%%%%%%%%%%%%%

\begin{frame}
\frametitle{Exercice 1}
 \[\left\{\begin{array}{l}
y_{k+1}=y_k+hf(t_k,y_{k+1})\\
y_0=1
\end{array}\right.\]
 \[\left\{\begin{array}{l}
y_{k+1}=y_k+h(3t_k+y_{k+1})\\
y_0=1
\end{array}\right.\]
\[\left\{\begin{array}{l}
y_{k+1}=\frac{y_k+3ht_k}{1-h}\\
y_0=1
\end{array}\right.\]
 \[y_1=\frac{1+3\times 0.1\times 0}{1-0.1}\simeq 1.11 \]
 \[y_2=\frac{1.11+3\times 0.1\times 0.1}{1-0.1}\simeq 1.27 \]
 \end{frame}
 
 %%%%%%%%%%%%%%%%%%%%%%%%%%%%%%%%%%%%%%%%%%%%%%%%%%%%%%%%%%%%%%%

\begin{frame}
\frametitle{Exercice 1}
 \[\left\{\begin{array}{l}
y_{k+1}=y_k+hf(t_k,y_{k+1})\\
y_0=1
\end{array}\right.\]
 \[\left\{\begin{array}{l}
y_{k+1}=y_k+h(3t_k+y_{k+1})\\
y_0=1
\end{array}\right.\]
\[\left\{\begin{array}{l}
y_{k+1}=\frac{y_k+3ht_k}{1-h}\\
y_0=1
\end{array}\right.\]
 \[y_1=\frac{1+3\times 0.1\times 0}{1-0.1}\simeq 1.11 \]
 \[y_2=\frac{1.11+3\times 0.1\times 0.1}{1-0.1}\simeq 1.27 \]
 \end{frame}
 
 
 \begin{frame}
\frametitle{Exercice 1}
 \[\left\{\begin{array}{l}
 \overline{y}_{i}=y_{i}+h f(t_{i},y_{i})\\
 y_{i+1}=y_{i}+\frac h2\left[f(t_{i},y_{i})+ f(t_{i}+h,\overline{y}_{i})\right]
\end{array}\right.\]

 \[\left\{\begin{array}{l}
 \overline{y}_{i}=y_{i}+h (3t_{i}+y_{i})\\
 y_{i+1}=y_{i}+\frac h2\left[(3t_{i}+y_{i})+ 3(t_{i}+h)+\overline{y}_{i}\right]
\end{array}\right.\]

 \[\left\{\begin{array}{l}
 \bar{y}_0=1+0.1(3\times 0+1)\\
 y_{1}=1+0.05\left[(3\times 0+1)+ 3(0+0.1)+\overline{y}_{0}\right]
\end{array}\right.\]
 \[\left\{\begin{array}{l}
 \bar{y}_0=1.1\\
 y_{1}=1+0.05\left[1+ 0.3+1.1\right]\simeq 1.12
\end{array}\right.\]
 \[\left\{\begin{array}{l}
 \overline{y}_{1}=1.12+0.1(3\times 0.1+1.12)\simeq 1.26\\
 y_{2}=1.12+0.05\left[(3\times 0.1+1.12)+ 3(0.1+0.1)+\overline{y}_{1}\right]\simeq 1.284
\end{array}\right.\]
\[y(0.2)=1.28561\]
 \end{frame}
 
 \begin{frame}
\frametitle{Exercice2}
 \[\left\{\begin{array}{l}
y_{k+1}=y_k+hf(t_k,y_{k})\\
y_0=1
\end{array}\right.\]
 \[\left\{\begin{array}{l}
y_{k+1}=y_k-hy_{k}=(1-h)y_k\\
y_0=1
\end{array}\right.\]
\[y_n=(1-h)^n\]


\[|y_n-y(t_n)|=|(1-h)^n-e^{-t_n}|=|(1-h)^n-e^{-nh}|=|(1-h)^n-(e^{-h})^n|\]
\[|y_n-y(t_n)|\leq n|(1-h)-e^{-h}|\]
  \end{frame}
   \begin{frame}
\frametitle{Exercice2}
Taylor-Lagrange:

\[|f(x)-f(x_0)-(x-x_0)f'(x_0)|\leq M_2\frac{|x-x_0]^2}2\]
\[|f(h)-f(0)-hf'(0)|\leq M_2\frac{h^2}2\]
\[f(t)=e^{-t}\]
\[|e^{-h}-1+h|\leq \frac{h^2}2\]
Finalement
\[|y_n-y(t_n)|\leq n|(1-h)-e^{-h}|\leq n  \frac{h^2}2\]
Or $h=\frac{b-a}{n}=\frac{10-0}{n} \Longrightarrow nh=10$ donc $|y_n-y(t_n)|\leq \frac{50}{n}$
  \end{frame}
  
   \begin{frame}
\frametitle{Exercice2}
 \[\left\{\begin{array}{l}
y_{k+1}=y_k+hf(t_k,y_{k+1})\\
y_0=1
\end{array}\right.\]
 \[\left\{\begin{array}{l}
y_{k+1}=y_k-hy_{k+1}\\
y_0=1
\end{array}\right.\]
\[\left\{\begin{array}{l}
y_{k+1}=\frac{y_k}{1+h}\\
y_0=1
\end{array}\right.\]
\[y_n=\frac 1{(1+h)^n}\]


\[|y_n-y(t_n)|=|(1+h)^{-n}-e^{-t_n}|=|(1+h)^{-n}-(e^{h})^{-n}|\]
\[|y_n-y(t_n)|\leq n|(1+h)-e^{h}|\leq n  \frac{h^2}2\leq \frac{50}{n}\]
  \end{frame}
  
  
  \begin{frame}
\frametitle{Exercice 4}
 On pose $Y_1=y$ et $Y_2=y'$:
 \[Y'=(Y'_1,Y'_2)=(y',y'')=(Y_2,-ty'-(1-t)y+2)\]
 \[Y'=(Y_2,-tY_2-(1-t)Y_1+2)\]
 \[Y'=F(t,Y)\]
  \end{frame}
  
   \begin{frame}
\frametitle{Exercice 4}
 \[f(t,y)=\sin y+\sin t\]
 \[|f(t,y)-f(t,z)|=|(\sin y+\sin t)-(\sin z+\sin t)|=|\sin y-\sin z|\]
  \[|f(t,y)-f(t,z)|\leq |y-z|\]
  
  \[y'=\sin y+\sin t \Longrightarrow y''=y'\cos y +\cos t\]
  \[|y''|\leq |y'|+1\leq 3\]
  \end{frame}
  \end{document}
   

























