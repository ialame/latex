% !TEX TS-program = pdflatexmk
%%%%%%%%%%%%%%%%%%%%%%%%%%%%%%%%%%%%%%%%%%%%%%%%%%%%%%%%%%%%%%%
%
% Welcome to Overleaf --- just edit your LaTeX on the left,
% and we'll compile it for you on the right. If you open the
% 'Share' menu, you can invite other users to edit at the same
% time. See www.overleaf.com/learn for more info. Enjoy!
%
%%%%%%%%%%%%%%%%%%%%%%%%%%%%%%%%%%%%%%%%%%%%%%%%%%%%%%%%%%%%%%%
\documentclass{article}
\usepackage{minted}
\usepackage{xcolor} % to access the named colour LightGray
\definecolor{LightGray}{gray}{0.9}
\begin{document}
\begin{minted}
[
frame=lines,
framesep=2mm,
baselinestretch=1.2,
bgcolor=LightGray,
fontsize=\footnotesize,
linenos
]
{python}
from tkinter import *
import numpy as np

# coordonnees initiales
x0, y0 = 10, 200
# vitesse initiale
alpha=np.pi/3
V0=50
# 'pas' du temps
h=0.1
z=np.array([x0,y0,V0*np.cos(alpha),-V0*np.sin(alpha)])
def f(z):
    return np.array([z[2],z[3],0,9.81])

def Euler():
    global z
    z=z+h*f(z)
    if z[1]>H-30 or z[1]<0:
        z[3]=-z[3]
    if z[0]>W-30 or z[0]<0:
        z[2]=-z[2]
    # deplacement de la balle a la nouvelle position
    can1.coords(balle, z[0], z[1], z[0] + 30, z[1] + 30)
    # La fenetre fen1 est actualisee en executant la
    # fonction Euler toutes les 10 millisecondes
    fen1.after(10, Euler)

# ========== Programme principal =============
# Creation de la fenetre principale :
fen1 = Tk()
fen1.title("Probleme de tir")
# creation du canvas :
H=W=750
can1 = Canvas(fen1, bg='dark grey', height=H, width=W)
can1.pack()
# creation de la balle
balle = can1.create_oval(x0, y0, x0 + 30, y0 + 30, width=2, fill='red')
# Lancement de la fonction Euler
Euler()
# demarrage de la boucle principale:
fen1.mainloop()
\end{minted}
\end{document}