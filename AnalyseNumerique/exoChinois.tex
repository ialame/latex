\documentclass[12pt,a4paper]{article}
\usepackage[utf8]{inputenc}
\usepackage[french]{babel}
\usepackage[T1]{fontenc}
\usepackage{amsmath}
\usepackage{amsfonts}
\usepackage{amssymb}
\usepackage{graphicx}
\usepackage{tikz}
\usepackage{pgfplots}
\pgfplotsset{compat=1.15}
\usepackage{mathrsfs}
\usetikzlibrary{arrows}
\pagestyle{empty}
%\usepackage{fourier}
\usepackage[left=2cm,right=2cm,top=2cm,bottom=2cm]{geometry}
\author{retranscrit par Léo Shi}
\title{TD n°1 - Analyse numérique - M. Alame}
\begin{document}

\maketitle

\section*{Exercice n°4 : Intégration numérique}

On considère l'intégrale : $$I = \int^2_1 \dfrac{1}{x}\text{d}x$$

\begin{enumerate}
\item On calcule la valeur exacte de $I$ : $$ I = \int^2_1 \dfrac{1}{x}\text{d}x = \left[\ln (x)\right]^2_1 = \ln 2
$$
\item Par la méthode des trapèzes : 
\begin{center}
\begin{tikzpicture}
\draw[thick,-] (0,0) -- (6,0);

\draw (0,-1pt) -- (0,1pt) node[anchor=north] {$x_0=1$};
\draw (2,-1pt) -- (2,1pt) node[anchor=north] {$\dfrac{4}{3}$};
\draw (4,-1pt) -- (4,1pt) node[anchor=north] {$\dfrac{5}{3}$};
\draw (6,-1pt) -- (6,1pt) node[anchor=north] {$x_3=2$};
\draw[thick,<->] (2,0.5) -- (4,0.5) node[anchor=south] {$h = \dfrac{1}{3}$};
\end{tikzpicture}
\end{center}

$$
\int^b_a f(x) \text{d}x = h\left[ \dfrac{f(a)+f(b)}{2} + \sum^2_{i=1} f(x_i) \right] = \dfrac{1}{3}\left[ \dfrac{1+\frac{1}{2}}{2} + \dfrac{3}{4} + \dfrac{3}{5} \right] = \dfrac{7}{10} = 0,7
$$

\item On s'aide d'un dessin : 

\begin{center}
\begin{tikzpicture}
%\draw[step=1cm,gray,very thin] (-1,-1) grid (5,4);
\draw[thick,->] (0,0) -- (4,0) node[anchor=north] {$x$};
\draw[thick,->] (0,0) -- (0,3) node[anchor=west] {$y$};
\filldraw [fill=gray!20,draw=black]
(1,0) -- (1,1)
-- plot [domain=1:2] (\x,1/\x)
-- (2,0) -- cycle;
\draw [domain=0.4:3] plot(\x,1/\x) node[anchor=south west] {$y=\dfrac{1}{x}$};
\draw (0,-1pt) -- (0,1pt) node[anchor=north] {0};
\draw (1,-1pt) -- (1,1pt) node[anchor=north] {1};
\draw (2,-1pt) -- (2,1pt) node[anchor=north] {2};
\draw[thick,-, draw=blue] (1,1) -- (2,0.5);
\draw[thick,-, draw=blue] (2,0.5) -- (2,0);
\draw[thick,-, draw=blue] (1,1) -- (1,0);

\end{tikzpicture}

On a $x \mapsto \dfrac{1}{x}$ qui est \textbf{convexe}, donc la courbe est en dessous de la droite.
\end{center}

\item \fbox{$E=\dfrac{(b-a)^4}{12n^2}f''(\xi)$} avec $\xi\in ]a,b[$, on cherche $n$ tel que $E\leq 10^{-4}$. On a $b=2$ et $a=1$ :
$$
1<\xi<2 \implies \dfrac{1}{2}<\dfrac{1}{\xi} <1 \implies \dfrac{1}{8}<\dfrac{1}{\xi^3} <1
$$
$$
\dfrac{(2-1)^4}{12n^2}\times \dfrac{2}{\xi^3}\leq 10^{-4} \implies \dfrac{2}{8} \times \dfrac{1}{12n^2} \leq \dfrac{1}{12n^2} \times \dfrac{2}{\xi^3} \leq 10^{-4}
$$
D'où $\dfrac{1}{96n^2}\leq 10^{-4}$, soit \fbox{$n\geq 10$}
\end{enumerate}

\section*{Exercice n°11 : Formule de Quadrature}

Soit la formule de quadrature, avec $\alpha \in ]0,1[$ : 

\begin{align*}
\int^1_{-1} f(x) \text{d}x = w_1 f(-\alpha)+w_2 f(\alpha) \qquad (R)
\end{align*}

\begin{enumerate}
\item $(R)$ vraie sur $\mathbb{R}_n[X]$ si et seulement si elle est vraie sur la base $(1,X,X^2, ..., X^n)$, donc pour $n=1$, la relation $(R)$ est vraie si et seulement si elle est vraie sur la base $(1,X)$. 
\begin{itemize}
\item Pour 1, on a alors : 
\begin{align*}
\int^1_{-1}1 \text{d}x = w_1 + w_2 \implies w_1 + w_2 = 2
\end{align*}
\item Pour $X$, on a : 
\begin{align*}
\int^1_{-1} x \text{d}x = w_1(-\alpha) + w_2(\alpha) \implies w_1(-\alpha) + w_2(\alpha) = 0 \implies w_1 = w_2
\end{align*}
\end{itemize}

Soit finalement, on obtient $w_1 = w_2 = 1$, d'où: 
\begin{align*}
  \int^1_{-1} f(x) \text{d}x = f(-\alpha) + f(\alpha)
\end{align*}

\item Lorsque $\alpha = 1$, on a : 
\begin{align*}
\int^1_{-1} f(x) \text{d}x = f(-1) + f(1)
\end{align*}
Pour les polynômes de $\mathbb{R}_2[X] = \mathbb{R}_1[X] + X^2$, d'où pour $f(x)=x^2$ : 
\begin{align*}
\int^1_{-1} x^2 \text{d}x = \dfrac{2}{3}
\end{align*}
Or $f(-1) + f(1) = 2$ et $2 \neq  \dfrac{2}{3}$, donc \underline{la formule n'est pas exacte sur $\mathbb{R}_2[X]$}


\item On a $\dfrac{2}{3} = \alpha^2 \times 2 \implies \alpha = + \sqrt{\dfrac{1}{3}}$ car $\alpha > 0$ ($\alpha \in ]0,1[$). La formule exacte est alors : 
\begin{align*}
\int^1_{-1} f(x) \text{d}x = f\left(\dfrac{-1}{\sqrt{3}} \right) + f\left(\dfrac{1}{\sqrt{3}}\right)
\end{align*}

\item Si $f(x)=x^3$, alors 
\begin{align*}
\int^1_{-1} x^3 \text{d}x = 0 \qquad \text{et} \qquad \left( \dfrac{-1}{\sqrt{3}}\right)^3 + \left( \dfrac{1}{\sqrt{3}}\right)^3 = 0
\end{align*}

Donc \underline{la formule est exacte sur $\mathbb{R}_3[X]$}. La dernière question n'est pas toujours valable, vérifions donc la formule sur $\mathbb{R}_4[X]$ : 

\begin{align*}
\int^1_{-1} x^4 \text{d}x = \dfrac{2}{5} \qquad \text{et} \qquad \left( \dfrac{-1}{\sqrt{3}}\right)^4 + \left( \dfrac{1}{\sqrt{3}}\right)^4 = \dfrac{2}{9}
\end{align*}
Etant donné que $\dfrac{2}{9}\neq \dfrac{2}{5}$, la formule \textbf{n'est pas exacte} pour $\mathbb{R}_4[X]$ (comme prédit ...)

\item \textbf{On cherche une relation entre $\xi$ sur l'intervalle $[-1,1]$ et $x$ sur l'intervalle $[a,b]$}. On suppose que la relation entre $x$ et $\xi$ est \textbf{affine}: 

\begin{align*}
x=\lambda \xi + \mu
\end{align*}



$$
\left\{
    \begin{array}{l}
        a= -\lambda + \mu   \\ 
		b=  \lambda + \mu 
    \end{array}
\right. 
\Longleftrightarrow
\left\{
    \begin{array}{ll}
        \mu = \dfrac{a+b}{2} \\ 
		\lambda = \dfrac{b-a}{2}
    \end{array}
\right.
$$



On pose $x = \dfrac{a+b}{2} + \dfrac{b-a}{2} \xi$, on a donc avec $\xi = \pm \dfrac{1}{\sqrt{3}} $
\begin{align*}
\int^b_a f(x) \text{d}x & = \int^1_{-1} \left( \dfrac{b-a}{2} \right) f\left( \dfrac{a+b}{2} + \dfrac{b-a}{2} \xi\right) \text{d}\xi \\
& = \dfrac{b-a}{2} \left[ f\left( \dfrac{a+b}{2} + \dfrac{b-a}{2\sqrt{3}}\right) + f\left( \dfrac{a+b}{2} - \dfrac{b-a}{2\sqrt{3}}\right) \right]
\end{align*}

\end{enumerate}

\end{document}