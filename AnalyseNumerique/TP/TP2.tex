\documentclass{article}
\usepackage[francais]{babel}
\usepackage[utf8]{inputenc} % Required for including letters with accents
\usepackage[T1]{fontenc} % Use 8-bit encoding that has 256 glyphs
\usepackage{pythontex}
\usepackage{amsthm}
\usepackage{amsmath}
\usepackage{amssymb}
\usepackage{mathrsfs}
\usepackage{graphicx}
\usepackage{geometry}
\usepackage{stmaryrd}
\usepackage{tikz}
\usetikzlibrary{patterns}
%\usetikzlibrary{intersections}
\usepackage[cache=false]{minted}

\usepackage{stmaryrd}
%\usepackage{tikz}
%\usetikzlibrary{tikzmark}
\usepackage{empheq}
\usepackage{longtable}
\usepackage{booktabs} 
\usepackage{array}
\usepackage{pstricks}
\usepackage{pst-3dplot}
\usepackage{pst-tree}
\usepackage{pstricks-add}
\usepackage{upgreek}
%\usepackage{epstopdf}
\usepackage{eolgrab}
\usepackage{chngpage}
 \usepackage{calrsfs}
 % Appel du package pythontex 
\usepackage{pythontex}

\usepackage{algorithm2e}
\RestyleAlgo{algoruled}
  \SetKw{KwFrom}{from} 
\newenvironment{algo}{
\begin{algorithm}[H]
\DontPrintSemicolon \SetAlgoVlined}
{\end{algorithm}}



\usetikzlibrary{decorations.pathmorphing}
\def \de {{\rm d}}
\usepackage{color}
\usepackage{xcolor}
\newcommand{\mybox}[1]{\fbox{$\displaystyle#1$}}
\newcommand{\myredbox}[1]{\fcolorbox{red}{white}{$\displaystyle#1$}}
\newcommand{\mydoublebox}[1]{\fbox{\fbox{$\displaystyle#1$}}}
\newcommand{\myreddoublebox}[1]{\fcolorbox{red}{white}{\fcolorbox{red}{white}{$\displaystyle#1$}}}

\usepackage{xcolor}
%\setbeamercolor{background canvas}{bg=lightgray}
\usepackage{listings}
\definecolor{purple2}{RGB}{153,0,153} % there’s actually no standard purple
\definecolor{green2}{RGB}{0,153,0} % a

 \title{TP1 Analyse numérique (B1-TP1)}
\author{Ibrahim ALAME}
\date{28/02/2024}
\begin{document}
\maketitle

\section{Problème de la chaleur}  
 Soit $OL$ une barre chauffée initialement à la température $\theta(x)=\sin^2(2\pi \frac xL)$, $0\leq x\leq L$. On cherche à refroidir la barre en appliquant une température $0$ aux extrémités $O$ et $L$. Déterminer l'évolution de la température $u(x,t)$ en tout point $x$ de la barre et à chaque instant $t\geq 0$:
 \[\left\{\begin{array}{ll}
 \frac{\partial u}{\partial t}-\kappa \frac{\partial^2 u}{\partial x^2}=0 & \forall (x,t)\in [0,L]\times\mathbb{R}^+\\
 u(x,0)=\theta(x) & \forall x \in [0,L]\\
 u(0,t)=u(L,t)=0 & \forall t\in \mathbb{R}^+
 \end{array}\right.\]
\begin{center}
 \begin{tikzpicture}[scale=1]
  \draw[->] (-0.1,0) -- (5.5,0)  node[right] {$x$};
  \draw[->] (0,-0.1) -- (0,3) node[right] {$\theta$};
   \draw [domain=0:5][line width=0.5,samples=500] plot(\x,{2*sin(2*pi*\x/5 r)^2});
    \draw[line width=2pt,orange] (0,0) -- (5,0) node[below] {$L$};
	\draw[orange] (0,0) node[below] {$O$};
\end{tikzpicture}
 \end{center}
On prendra $\kappa=0.1$, $L=1$ et on choisira un pas de temps $\tau=0.0005$ et on discrétisera l'intervalle $[0,L]$ en $N$ sous-intervalles avec un pas $h=0.01$. L'animation de l'évolution de la température est à afficher dans une fenêtre $800\times 600$  de la librairie python {\tt tkinter}:
\begin{minted}[
mathescape,
framesep=2mm,
baselinestretch=1.2,
fontsize=\footnotesize,
linenos
]{python}
#=============  PROGRAMME PRINCIPAL  ==============
fen1 = Tk()
fen1.title("Equation de la chaleur")
W=800; H=600
can1 = Canvas(fen1,bg = 'dark gray', width=W, height=H)
can1.pack()
DF()
fen1.mainloop()
\end{minted} 

où {\tt DF()} est la fonction qui calcule et affiche la température de la barre à l'instant $t_n=n\tau$ à partir du schéma de la méthode explicite:
\[\frac{u^n_i-u^{n-1}_i}{\tau}-\kappa \frac{u^{n-1}_{i-1}-2u^{n-1}_i+u^{n-1}_{i+1}}{h^2}=0\]
\begin{minted}[
mathescape,
framesep=2mm,
baselinestretch=1.2,
fontsize=\footnotesize,
linenos
]{python}
def DF():
    global X, U0, H, W
    can1.delete('all')
    U=np.zeros(N+1)
    for i in range(1,N):
        U[i]=....
    for i in range(1,N+1):
        can1.create_line(X[i-1]*W,(1-U[i-1])*H,X[i]*W,(1-U[i])*H,fill='blue', width=5)
    U0 = U.copy()

    fen1.after(10, DF)
\end{minted}     
%  U[i]=U0[i]+c*(U0[i-1]-2*U0[i]+U0[i+1])    
\section{Propagation des ondes}
On considère une corde de masse linéïque $\rho$  fixée en ses deux extrémités $O$ et $L$ et tendue avec une grande force $F$. On pose $c=\sqrt\frac{F}{\rho}$. A l'instant $t=0$, on déplace verticalement chaque point d'abscisse $x$  d'une distance $f(x)$ à une vitesse initiale $g(x)$. En petites perturbations, le déplacement $u(x,t)$ en tout point $x$ de la corde et à chaque instant $t\geq 0$ est solution du problème des ondes suivant:
 \[\left\{\begin{array}{ll}
 \frac{\partial^2 u}{\partial t^2}-c^2 \frac{\partial^2 u}{\partial x^2}=0 & \forall (x,t)\in [0,L]\times\mathbb{R}^+\\
 u(x,0)=f(x) & \forall x \in [0,L]\\
  \frac{\partial u}{\partial t}(x,0)=g(x) & \forall x \in [0,L]\\
 u(0,t)=u(L,t)=0 & \forall t\in \mathbb{R}^+
 \end{array}\right.\]
Résoudre numériquement ce problème par la méthode des différences finies, en prenant les données numériques suivantes:
\[L=1\mbox{ m},\quad c=3\mbox{ m/s},\quad f(x)=e^{-500 (x-L/2)^2},\quad g(x)=0,\]

afficher le résultat en animation graphique dans une fenêtre {\tt tkinter} de hauteur $H=800$px et de largeur $W=800$px.
\section{Équation de diffusion}


  \end{document}