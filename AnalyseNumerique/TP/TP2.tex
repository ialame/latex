\documentclass{article}
\usepackage[francais]{babel}
\usepackage[utf8]{inputenc} % Required for including letters with accents
\usepackage[T1]{fontenc} % Use 8-bit encoding that has 256 glyphs
\usepackage{pythontex}
\usepackage{amsthm}
\usepackage{amsmath}
\usepackage{amssymb}
\usepackage{mathrsfs}
\usepackage{graphicx}
\usepackage{geometry}
\usepackage{stmaryrd}
\usepackage{tikz}
\usetikzlibrary{patterns}
%\usetikzlibrary{intersections}
\usepackage[cache=false]{minted}

\usepackage{stmaryrd}
%\usepackage{tikz}
%\usetikzlibrary{tikzmark}
\usepackage{empheq}
\usepackage{longtable}
\usepackage{booktabs} 
\usepackage{array}
\usepackage{pstricks}
\usepackage{pst-3dplot}
\usepackage{pst-tree}
\usepackage{pstricks-add}
\usepackage{upgreek}
%\usepackage{epstopdf}
\usepackage{eolgrab}
\usepackage{chngpage}
 \usepackage{calrsfs}
 % Appel du package pythontex 
\usepackage{pythontex}

\usepackage{algorithm2e}
\RestyleAlgo{algoruled}
  \SetKw{KwFrom}{from} 
\newenvironment{algo}{
\begin{algorithm}[H]
\DontPrintSemicolon \SetAlgoVlined}
{\end{algorithm}}



\usetikzlibrary{decorations.pathmorphing}
\def \de {{\rm d}}
\usepackage{color}
\usepackage{xcolor}
\newcommand{\mybox}[1]{\fbox{$\displaystyle#1$}}
\newcommand{\myredbox}[1]{\fcolorbox{red}{white}{$\displaystyle#1$}}
\newcommand{\mydoublebox}[1]{\fbox{\fbox{$\displaystyle#1$}}}
\newcommand{\myreddoublebox}[1]{\fcolorbox{red}{white}{\fcolorbox{red}{white}{$\displaystyle#1$}}}

\usepackage{xcolor}
%\setbeamercolor{background canvas}{bg=lightgray}
\usepackage{listings}
\definecolor{purple2}{RGB}{153,0,153} % there’s actually no standard purple
\definecolor{green2}{RGB}{0,153,0} % a

 \title{TP1 Analyse numérique (B1-TP1)}
\author{Ibrahim ALAME}
\date{28/02/2024}
\begin{document}
\maketitle

\section{Problème de la chaleur}  
 Soit $OL$ une barre chauffée initialement à la température $\theta(x)=\sin^2(2\pi \frac xL)$, $0\leq x\leq L$. On cherche à refroidir la barre en appliquant une température $0$ aux extrémités $O$ et $L$. Déterminer l'évolution de la température $u(x,t)$ en tout point $x$ de la barre et à chaque instant $t\geq 0$:
 \[\left\{\begin{array}{ll}
 \frac{\partial u}{\partial t}-\kappa \frac{\partial^2 u}{\partial x^2}=0 & \forall (x,t)\in [0,L]\times\mathbb{R}^+\\
 u(x,0)=\theta(x) & \forall x \in [0,L]\\
 u(0,t)=u(L,t)=0 & \forall t\in \mathbb{R}^+
 \end{array}\right.\]
\begin{center}
 \begin{tikzpicture}[scale=1]
  \draw[->] (-0.1,0) -- (5.5,0)  node[right] {$x$};
  \draw[->] (0,-0.1) -- (0,3) node[right] {$\theta$};
   \draw [domain=0:5][line width=0.5,samples=500] plot(\x,{2*sin(2*pi*\x/5 r)^2});
    \draw[line width=2pt,orange] (0,0) -- (5,0) node[below] {$L$};
	\draw[orange] (0,0) node[below] {$O$};
\end{tikzpicture}
 \end{center}
On prendra $\kappa=0.1$, $L=1$ et on choisira un pas de temps $\tau=0.0005$

\begin{minted}[
mathescape,
framesep=2mm,
baselinestretch=1.2,
fontsize=\footnotesize,
linenos
]{python}
# Initialisation des constantes
L=1; kappa = 0.1
h=0.01; tau= 0.0005
c = kappa * tau / h**2
\end{minted} 
 et on discrétisera l'intervalle $[0,L]$ en $N$ sous-intervalles avec un pas $h=0.01$ ($N+1$ points de discrétisation). 
 \begin{minted}[
mathescape,
framesep=2mm,
baselinestretch=1.2,
fontsize=\footnotesize,
linenos
]{python}
# Coordonnées en espace
N = int(L / h)
X = np.linspace(0, L, N+1)
\end{minted} 
 L'animation de l'évolution de la température est à afficher dans une fenêtre $800\times 600$  de la librairie python {\tt tkinter}:
\begin{minted}[
mathescape,
framesep=2mm,
baselinestretch=1.2,
fontsize=\footnotesize,
linenos
]{python}
#=============  PROGRAMME PRINCIPAL  ==============
fen1 = Tk() # Création de la fenêtre principale
fen1.title("Equation de la chaleur")
W=800; H=600
# Définir une zone graphique $W \times H$
can1 = Canvas(fen1,bg = 'dark gray', width=W, height=H)
can1.pack() # Positionne can1 dans la fenêtre fen1
DF() # Fonction de calcul en Diff Finies et affichage graphique
fen1.mainloop() # Boucle principale de la fenêtre
\end{minted} 
où {\tt DF()} est la fonction qui calcule et affiche la température de la barre à l'instant $t_n=n\tau$ à partir du schéma de la méthode explicite:
\[\frac{u^n_i-u^{n-1}_i}{\tau}-\kappa \frac{u^{n-1}_{i-1}-2u^{n-1}_i+u^{n-1}_{i+1}}{h^2}=0\]
\begin{minted}[
mathescape,
framesep=2mm,
baselinestretch=1.2,
fontsize=\footnotesize,
linenos
]{python}
def DF():
    global U0 # on désigne par U0[i] la solution à l'instant précédent  $u(x_i,t_{n-1})=u^{n-1}_i$
    can1.delete('all') # on efface la courbe de l'instant t-1
    U=np.zeros(N+1)# U[i] est la solution à l'instant actuel  $u(x_i,t_{n})=u^n_i$
    for i in range(1,N):
        U[i]=...
    # courbe à l'instant t; segment entre deux points consécutifs: create_line($x_{i-1},y_{i-1},x_i,y_i$)
    for i in range(1,N+1):
        can1.create_line(X[i-1]*W,(1-U[i-1])*H,X[i]*W,(1-U[i])*H,fill='blue', width=5)
    U0 = U.copy()
    fen1.after(10, DF) # réexécute la fonction $DF()$ à l'instant suivant après 10ms
\end{minted}     
%  U[i]=U0[i]+c*(U0[i-1]-2*U0[i]+U0[i+1])   
\subsubsection*{Transformation géométrique entre le repère informatique $(Oxy)$ et le repère mathématique $(\Omega X Y)$  }

\begin{center}
 \begin{tikzpicture}[scale=1]
  \draw[dashed, olive, fill=blue!30, fill opacity=0.6](5,0) -- ++(0,-4)--++(-5,0)--++(0,4);
  \draw[->,orange] (-0.1,0) -- (5.5,0)  node[right] {$x$};
  \draw[->,blue] (-0.1,-2) -- (5.5,-2)  node[right] {$X$};
  \draw[->,orange] (0,0.1) -- (0,-4.5) node[left] {$y$};
\draw[->,blue] (-0.05,-4) -- (-0.05,0.5) node[above] {$Y$};
\draw[olive,dashed] (5,0) -- ++(0,-4)--++(-5,0)  node[left] {$H$};

\draw[olive] (5,0) node[above] {$W$};
\draw[blue] (0,-4) node[above right] {$-1$};
 \draw[blue] (0,0) node[below right] {$+1$};
  \draw[blue] (5,-2) node[below left] {$+1$};
	\draw[orange] (0,0) node[left] {$O$};
		\draw[blue] (0,-2) node[left] {$\Omega$};
	\draw(-4,-2) node{$\left\{\begin{array}{l}
x=X\cdot W \\
y=(1-Y)\cdot H/2
\end{array}\right.$};
\end{tikzpicture}
\end{center}
\section{Propagation des ondes}
On considère une corde de masse linéïque $\rho$  fixée en ses deux extrémités $O$ et $L$ et tendue avec une grande force $F$. On pose $c=\sqrt\frac{F}{\rho}$. A l'instant $t=0$, on déplace verticalement chaque point d'abscisse $x$  d'une distance $f(x)$ à une vitesse initiale $g(x)$. En petites perturbations, le déplacement $u(x,t)$ en tout point $x$ de la corde et à chaque instant $t\geq 0$ est solution du problème des ondes suivant:
 \[\left\{\begin{array}{ll}
 \frac{\partial^2 u}{\partial t^2}-c^2 \frac{\partial^2 u}{\partial x^2}=0 & \forall (x,t)\in [0,L]\times\mathbb{R}^+\\
 u(x,0)=f(x) & \forall x \in [0,L]\\
  \frac{\partial u}{\partial t}(x,0)=g(x) & \forall x \in [0,L]\\
 u(0,t)=u(L,t)=0 & \forall t\in \mathbb{R}^+
 \end{array}\right.\]
Résoudre numériquement ce problème par la méthode des différences finies:
\[\frac{u^n_i-2u^{n-1}_i+u^{n-2}_i}{\tau^2}-c^2\frac{u^{n-1}_{i+1}-2u^{n-1}_{i}+u^{n-1}_{i-1}}{h^2}=0\]
en prenant les données numériques suivantes:
\[L=1\mbox{ m},\quad c=3\mbox{ m/s},\quad f(x)=e^{-500 (x-L/2)^2},\quad g(x)=0,\]
afficher le résultat en animation graphique dans une fenêtre {\tt tkinter} de hauteur $H=800$px et de largeur $W=800$px.
\section{Équation de diffusion}
On considère l'équation d'advection suivante:
$$ \left\{ 
    \begin{array}{l}
\displaystyle \frac{\partial u}{\partial t}+ V\frac{\partial u}{\partial x}=0, \quad 0 \leq x \leq 1,\quad t \in \mathbb{R}^+\\
u(x+1,t)=u(x,t)\\
u(x,0)=u_0(x)= \cos(\pi x)^8
    \end{array}  
    \right. 
  $$
\begin{enumerate}
\item   Pour résoudre ce problème, programmer le {\em schéma explicite centré}, suivants:
  \[\frac{u^n_i-u^{n-1}_i}{\tau}+V \frac{u^{n-1}_{i+1}-u^{n-1}_{i-1}}{2h}=0\]
\item {\em Schéma de Lax-Wendroff}
On fait un développement limité à l'ordre 2 en temps de $u(x,t)$:
\[u(x_i,t_{n+1}) = u(x_i,t_{n})+\tau \frac{\partial u}{\partial t}(x_i,t_n)+\frac{\tau^2}{2}\frac{\partial^2 u}{\partial t^2}(x_i,t_n)+o(\tau^2)\]
Nous allons passer les dérivées partielles par rapport au temps à des dérivées partielles par rapport à l'espace:
\[\frac{\partial u}{\partial t}=- V\frac{\partial u}{\partial x}\]
et 
\[\frac{\partial^2 u}{\partial t^2}=- V\frac{\partial}{\partial t}\left[\frac{\partial u}{\partial x}\right]=- V\frac{\partial}{\partial x}\left[\frac{\partial u}{\partial t}\right]=+ V^2\frac{\partial^2 u}{\partial x^2}\]
Finalement
\[u(x_i,t_{n+1})= u(x_i,t_{n})-\tau V\frac{\partial u}{\partial x}(x_i,t_n)+\frac{\tau^2 V^2}{2}\frac{\partial^2 u}{\partial x^2}(x_i,t_n)+o(\tau^2)\]
D'où le schéma de Lax-Wendroff:
\[\frac{u^n_i-u^{n-1}_i}{\tau}+V \frac{u^{n-1}_{i+1}-u^{n-1}_{i-1}}{2h}-\frac{\tau V^2}{2}\frac{u^{n-1}_{i+1}-2u^{n-1}_{i}+u^{n-1}_{i-1}}{h^2}=0\]
Programmer ce schéma et comparer avec la méthode explicite centrée.
\end{enumerate}


  \end{document}