\documentclass[a4paper]{article} 
\usepackage[francais]{babel}
\usepackage[utf8]{inputenc} % Required for including letters with accents
\usepackage[T1]{fontenc} % Use 8-bit encoding that has 256 glyphs
\usepackage{pythontex}
\usepackage{amsthm}
\usepackage{amsmath}
\usepackage{amssymb}
\usepackage{mathrsfs}
\usepackage{graphicx}
\usepackage{geometry}
\usepackage{stmaryrd}
\usepackage{tikz}
\usetikzlibrary{patterns}

\def \de {{\rm d}}

\usepackage{geometry}
 \geometry{
 a4paper,
 total={210mm,297mm},
 left=20mm,
 right=20mm,
 top=20mm,
 bottom=20mm,
 }

\title{Les livrables du projet numérique: Treillis  de barres ou de poutres}
\author{Ibrahim ALAME}
\date{21/03/2024}
\begin{document}
\maketitle
 
 \newcommand{\appui}[3]%
{\fill[fill=gray]   (#1,#2) -- (#1-#3*0.8,#2+#3*0.5)--(#1-#3*0.8,#2-#3*0.5)--cycle;
\fill[fill=gray] [pattern=north east lines]
     (#1-#3*0.8,#2+#3*0.5)
     --(#1-#3*0.8,#2-#3*0.5)
     -- (#1-#3*0.8-#3*0.5,#2-#3*0.5)
     -- (#1-#3*0.8-#3*0.5,#2+#3*0.5)
     -- cycle;

}
\subsection*{Datalore}
Datalore est une plateforme collaborative de notebooks des données scientifiques. Elle est axée sur le code et la collaboration. Disponible en ligne pour un usage personnel et sur site pour les entreprises.

Datalore fonctionne dans les navigateurs, est compatible avec Jupyter et propose une assistance au codage intelligente pour les notebooks en Python, SQL, R et Scala. Les équipes peuvent partager des notebooks par lien, les modifier ensemble en temps réel, et organiser des projets dans des espaces de travail. 

Pour les rendus des séances d'avancement de projet, les élèves doivent transformer les notebooks des livrables en rapports en se servant de HTML et/ou LATEX et les partager avec l'enseignant avant la fin de chaque séance.
\subsection*{Livrable 1}
 Pour la structure de treillis proposée à votre sous-groupe, établir les équations d'équilibre et déterminer, par un calcul manuel, les efforts dans chacune des barres, le déplacement de chaque nœud et les réaction des appuis puis tracer le système déformé.
 \subsubsection*{Exemple 1}
 \begin{verbatim}
Points = [[ L , 0 ], [ L , L], [0, L ], [ 0 , 0]]
Barres = [[ 0 , 1 ], [ 0 , 2 ], [ 0 , 3 ]]
Conditions = [[ 1 , 0 , 0 ],[ 2 , 0 , 0 ] ,[ 3 , 0 , 0 ]] 
Forces = [[0,0,-P]]
 \end{verbatim}
\begin{center}
\begin{tikzpicture}[scale=15]
\draw[orange, double distance = 1pt] (0.2,0) - - (0.2,0.2);
\draw[orange, double distance = 1pt] (0.2,0) - - (0,0.2);
\draw[orange, double distance = 1pt] (0.2,0) - - (0,0);
\path[fill=orange!40]  (0.2,0) circle (.075mm);
\path[fill=orange!40]  (0.2,0.2) circle (.075mm);
\path[fill=orange!40]  (0,0.2) circle (.075mm);
\path[fill=orange!40]  (0,0) circle (.075mm);
\draw[blue, double distance = 1pt] (0.20672893218813454,-0.01672893218813452) - - (0.2,0.2);
\draw[blue, double distance = 1pt] (0.20672893218813454,-0.01672893218813452) - - (0.0,0.2);
\draw[blue, double distance = 1pt] (0.20672893218813454,-0.01672893218813452) - - (0.0,0.0);
\path[fill=black]  (0.20672893218813454,-0.01672893218813452) circle (.075mm) [fill=gray];
\path[fill=black]  (0.2,0.2) circle (.075mm) [fill=gray];
\path[fill=black]  (0.0,0.2) circle (.075mm) [fill=gray];
\path[fill=black]  (0.0,0.0) circle (.075mm) [fill=gray];
\draw[very thick,red,-latex] (0.20672893218813454,-0.01672893218813452)--++(0.0,-0.06666666666666667);
\end{tikzpicture}
\end{center}
\subsubsection*{Exemple 2}
 \begin{verbatim}
Points = [[ 0 , L ], [ L , 0], [L, L ], [ 2*L , L]]
Barres = [[ 0 , 1 ], [ 1 , 2 ], [1 , 3 ]]
Conditions = [[ 0 , 0 , 0 ],[ 2 , 0 , 0 ] ,[ 3 , 0 , 0 ]] 
Forces = [[1,P,0]]
\end{verbatim}

\begin{center}
\begin{tikzpicture}[scale=15]
\draw[orange, double distance = 1pt] (0,0.2) - - (0.2,0);
\draw[orange, double distance = 1pt] (0.2,0) - - (0.2,0.2);
\draw[orange, double distance = 1pt] (0.2,0) - - (0.4,0.2);
\path[fill=orange!40]  (0,0.2) circle (.075mm);
\path[fill=orange!40]  (0.2,0) circle (.075mm);
\path[fill=orange!40]  (0.2,0.2) circle (.075mm);
\path[fill=orange!40]  (0.4,0.2) circle (.075mm);
\draw[blue, double distance = 1pt] (0.0,0.2) - - (0.21414213562373097,-0.012445079348883236);
\draw[blue, double distance = 1pt] (0.21414213562373097,-0.012445079348883236) - - (0.2,0.2);
\draw[blue, double distance = 1pt] (0.21414213562373097,-0.012445079348883236) - - (0.4,0.2);
\path[fill=black]  (0.0,0.2) circle (.075mm) [fill=gray];
\path[fill=black]  (0.21414213562373097,-0.012445079348883236) circle (.075mm) [fill=gray];
\path[fill=black]  (0.2,0.2) circle (.075mm) [fill=gray];
\path[fill=black]  (0.4,0.2) circle (.075mm) [fill=gray];
\draw[very thick,red,-latex] (0.21414213562373097,-0.012445079348883236)--++(0.06666666666666667,0.0);
\end{tikzpicture}
\end{center}
\subsubsection*{Exemple 3}
 \begin{verbatim}
Points = [[ 0 , 0 ], [ 0 , L], [L, L ], [ 2*L , 0]]
Barres = [[ 0 , 2 ], [ 1 , 2 ], [2 , 3 ],[0,3]]
Conditions = [[ 0 , 0 , 0 ],[ 1 , 0 , 0 ]] 
Forces = [[3,0,-P/2]]
 \end{verbatim}

\begin{center}
\begin{tikzpicture}[scale=15]
\draw[orange, double distance = 1pt] (0,0) - - (0.2,0.2);
\draw[orange, double distance = 1pt] (0,0.2) - - (0.2,0.2);
\draw[orange, double distance = 1pt] (0.2,0.2) - - (0.4,0);
\draw[orange, double distance = 1pt] (0,0) - - (0.4,0);
\path[fill=orange!40]  (0,0) circle (.075mm);
\path[fill=orange!40]  (0,0.2) circle (.075mm);
\path[fill=orange!40]  (0.2,0.2) circle (.075mm);
\path[fill=orange!40]  (0.4,0) circle (.075mm);
\draw[blue, double distance = 1pt] (0.0,0.0) - - (0.21880000000000002,0.18465786437626905);
\draw[blue, double distance = 1pt] (0.0,0.2) - - (0.21880000000000002,0.18465786437626905);
\draw[blue, double distance = 1pt] (0.21880000000000002,0.18465786437626905) - - (0.4076,-0.058284271247461945);
\draw[blue, double distance = 1pt] (0.0,0.0) - - (0.4076,-0.058284271247461945);
\path[fill=black]  (0.0,0.0) circle (.075mm) [fill=gray];
\path[fill=black]  (0.0,0.2) circle (.075mm) [fill=gray];
\path[fill=black]  (0.21880000000000002,0.18465786437626905) circle (.075mm) [fill=gray];
\path[fill=black]  (0.4076,-0.058284271247461945) circle (.075mm) [fill=gray];
\draw[very thick,red,-latex] (0.4076,-0.058284271247461945)--++(0.0,-0.066);
\end{tikzpicture}
\end{center}
\subsubsection*{Exemple 4}
 \begin{verbatim}
Points = [[ 0 , 0 ], [ 0 , L], [L, 0 ], [ 2*L , L]]
Barres = [[ 0 , 2 ], [ 1 , 2 ], [2 , 3 ],[1,3]]
Conditions = [[ 0 , 0 , 0 ],[ 1 , 0 , 0 ]] 
Forces = [[2,0,-P/2]]
 \end{verbatim}

\begin{center}
\begin{tikzpicture}[scale=15]
\draw[orange, double distance = 1pt] (0,0) - - (0.2,0);
\draw[orange, double distance = 1pt] (0,0.2) - - (0.2,0);
\draw[orange, double distance = 1pt] (0.2,0) - - (0.4,0.2);
\draw[orange, double distance = 1pt] (0,0.2) - - (0.4,0.2);
\path[fill=orange!40]  (0,0) circle (.075mm);
\path[fill=orange!40]  (0,0.2) circle (.075mm);
\path[fill=orange!40]  (0.2,0) circle (.075mm);
\path[fill=orange!40]  (0.4,0.2) circle (.075mm);
\draw[blue, double distance = 1pt] (0.0,0.0) - - (0.2038,-0.02794213562373097);
\draw[blue, double distance = 1pt] (0.0,0.2) - - (0.2038,-0.02794213562373097);
\draw[blue, double distance = 1pt] (0.2038,-0.02794213562373097) - - (0.4176,0.17585786437626905);
\draw[blue, double distance = 1pt] (0.0,0.2) - - (0.4176,0.17585786437626905);
\path[fill=black]  (0.0,0.0) circle (.075mm) [fill=gray];
\path[fill=black]  (0.0,0.2) circle (.075mm) [fill=gray];
\path[fill=black]  (0.2038,-0.02794213562373097) circle (.075mm) [fill=gray];
\path[fill=black]  (0.4176,0.17585786437626905) circle (.075mm) [fill=gray];
\draw[very thick,red,-latex] (0.2038,-0.02794213562373097)--++(0.0,-0.066);
\end{tikzpicture}
\end{center}

\subsubsection*{Exemple 5}
 \begin{verbatim}
Points = [[ 0 , 0 ], [ L , 0], [L/2, L ], [ L/2 , 2*L]]
Barres = [[ 0 , 2 ], [ 1 , 2 ], [2,3]]
Conditions = [[ 0 , 0 , 0 ],[ 1 , 1 , 0 ],[ 3 , 0 , 0 ]] 
Forces = [[2,P/4,0]]
 \end{verbatim}

\begin{center}
\begin{tikzpicture}[scale=15]
\draw[orange, double distance = 1pt] (0,0) - - (0.1,0.2);
\draw[orange, double distance = 1pt] (0.2,0) - - (0.1,0.2);
\draw[orange, double distance = 1pt] (0.1,0.2) - - (0.1,0.4);
\path[fill=orange!40]  (0,0) circle (.075mm);
\path[fill=orange!40]  (0.2,0) circle (.075mm);
\path[fill=orange!40]  (0.1,0.2) circle (.075mm);
\path[fill=orange!40]  (0.1,0.4) circle (.075mm);
\draw[blue, double distance = 1pt] (0.0,0.0) - - (0.1635754248593737,0.1862);
\draw[blue, double distance = 1pt] (0.31317542485937366,0.0) - - (0.1635754248593737,0.1862);
\draw[blue, double distance = 1pt] (0.1635754248593737,0.1862) - - (0.1,0.4);
\path[fill=black]  (0.0,0.0) circle (.075mm) [fill=gray];
\path[fill=black]  (0.31317542485937366,0.0) circle (.075mm) [fill=gray];
\path[fill=black]  (0.1635754248593737,0.1862) circle (.075mm) [fill=gray];
\path[fill=black]  (0.1,0.4) circle (.075mm) [fill=gray];
\draw[very thick,red,-latex] (0.1635754248593737,0.1862)--++(0.066,0.0);
\end{tikzpicture}
\end{center}

\subsubsection*{Exemple 6}
 \begin{verbatim}
Points = [[ 0 , 0 ], [ L , 0], [L, L/2 ], [ 2*L , 0]]
Barres = [[ 0 , 1 ], [ 0 , 2 ], [1,2],[1,3],[2,3]]
Conditions = [[ 0 , 0 , 0 ],[ 2 , 1 , 0 ]] 
Forces = [[1,-P,-P]]
 \end{verbatim}

\begin{center}
\begin{tikzpicture}[scale=15]
\draw[orange, double distance = 1pt] (0,0) - - (0.2,0);
\draw[orange, double distance = 1pt] (0,0) - - (0.2,0.1);
\draw[orange, double distance = 1pt] (0.2,0) - - (0.2,0.1);
\draw[orange, double distance = 1pt] (0.2,0) - - (0.4,0);
\draw[orange, double distance = 1pt] (0.2,0.1) - - (0.4,0);
\path[fill=orange!40]  (0,0) circle (.075mm);
\path[fill=orange!40]  (0.2,0) circle (.075mm);
\path[fill=orange!40]  (0.2,0.1) circle (.075mm);
\path[fill=orange!40]  (0.4,0) circle (.075mm);
\draw[blue, double distance = 1pt] (0.0,0.0) - - (0.1988,-0.009399999999999999);
\draw[blue, double distance = 1pt] (0.0,0.0) - - (0.21100000000000002,0.1);
\draw[blue, double distance = 1pt] (0.1988,-0.009399999999999999) - - (0.21100000000000002,0.1);
\draw[blue, double distance = 1pt] (0.1988,-0.009399999999999999) - - (0.4076,-0.028799999999999996);
\draw[blue, double distance = 1pt] (0.21100000000000002,0.1) - - (0.4076,-0.028799999999999996);
\path[fill=black]  (0.0,0.0) circle (.075mm) [fill=gray];
\path[fill=black]  (0.1988,-0.009399999999999999) circle (.075mm) [fill=gray];
\path[fill=black]  (0.21100000000000002,0.1) circle (.075mm) [fill=gray];
\path[fill=black]  (0.4076,-0.028799999999999996) circle (.075mm) [fill=gray];
\draw[very thick,red,-latex] (0.1988,-0.009399999999999999)--++(-0.06666666666666667,-0.06666666666666667);
\end{tikzpicture}
\end{center}

\subsubsection*{Exemple 7}
 \begin{verbatim}
Points = [[ 0 , L ], [ r3*L , 0], [r3*L, L ], [ 2*r3*L , L]]
Barres = [[ 0 , 2 ], [ 1 , 2 ], [1,3],[2,3]]
Conditions = [[ 0 , 0 , 0 ],[ 1 , 0 , 0 ]] 
Forces = [[3,0,-P/4]]
\end{verbatim}


\begin{center}
\begin{tikzpicture}[scale=15]
\draw[orange, double distance = 1pt] (0,0.2) - - (0.34641016151377546,0.2);
\draw[orange, double distance = 1pt] (0.34641016151377546,0) - - (0.34641016151377546,0.2);
\draw[orange, double distance = 1pt] (0.34641016151377546,0) - - (0.6928203230275509,0.2);
\draw[orange, double distance = 1pt] (0.34641016151377546,0.2) - - (0.6928203230275509,0.2);
\path[fill=orange!40]  (0,0.2) circle (.075mm);
\path[fill=orange!40]  (0.34641016151377546,0) circle (.075mm);
\path[fill=orange!40]  (0.34641016151377546,0.2) circle (.075mm);
\path[fill=orange!40]  (0.6928203230275509,0.2) circle (.075mm);
\draw[blue, double distance = 1pt] (0.0,0.2) - - (0.36915220862038156,0.2088);
\draw[blue, double distance = 1pt] (0.34641016151377546,0.0) - - (0.36915220862038156,0.2088);
\draw[blue, double distance = 1pt] (0.34641016151377546,0.0) - - (0.7383044172407631,0.1364192378864669);
\draw[blue, double distance = 1pt] (0.36915220862038156,0.2088) - - (0.7383044172407631,0.1364192378864669);
\path[fill=black]  (0.0,0.2) circle (.075mm) [fill=gray];
\path[fill=black]  (0.34641016151377546,0.0) circle (.075mm) [fill=gray];
\path[fill=black]  (0.36915220862038156,0.2088) circle (.075mm) [fill=gray];
\path[fill=black]  (0.7383044172407631,0.1364192378864669) circle (.075mm) [fill=gray];
\draw[very thick,red,-latex] (0.7383044172407631,0.1364192378864669)--++(0.0,-0.06666666666666666);
\end{tikzpicture}
\end{center}

\subsubsection*{Exemple 8}
 \begin{verbatim}
Points = [[ 0 , 0 ], [ 0 , L], [L, 0 ], [ L , L]]
Barres = [[ 0 , 1 ], [ 1 , 2 ], [1,3],[2,3]]
Conditions = [[ 0 , 0 , 0 ],[ 2 , 0 , 0 ]] 
Forces = [[3,P/2,0]]
 \end{verbatim}



\begin{center}
\begin{tikzpicture}[scale=15]
\draw[orange, double distance = 1pt] (0,0) - - (0,0.2);
\draw[orange, double distance = 1pt] (0,0.2) - - (0.2,0);
\draw[orange, double distance = 1pt] (0,0.2) - - (0.2,0.2);
\draw[orange, double distance = 1pt] (0.2,0) - - (0.2,0.2);
\path[fill=orange!40]  (0,0) circle (.075mm);
\path[fill=orange!40]  (0,0.2) circle (.075mm);
\path[fill=orange!40]  (0.2,0) circle (.075mm);
\path[fill=orange!40]  (0.2,0.2) circle (.075mm);
\draw[blue, double distance = 1pt] (0.0,0.0) - - (0.010342135623730948,0.21380000000000002);
\draw[blue, double distance = 1pt] (0.010342135623730948,0.21380000000000002) - - (0.2,0.0);
\draw[blue, double distance = 1pt] (0.010342135623730948,0.21380000000000002) - - (0.22414213562373095,0.2088);
\draw[blue, double distance = 1pt] (0.2,0.0) - - (0.22414213562373095,0.2088);
\path[fill=black]  (0.0,0.0) circle (.075mm) [fill=gray];
\path[fill=black]  (0.010342135623730948,0.21380000000000002) circle (.075mm) [fill=gray];
\path[fill=black]  (0.2,0.0) circle (.075mm) [fill=gray];
\path[fill=black]  (0.22414213562373095,0.2088) circle (.075mm) [fill=gray];
\draw[very thick,red,-latex] (0.22414213562373095,0.2088)--++(0.066,0.0);
\end{tikzpicture}
\end{center}
\subsubsection*{Exemple 9}
 \begin{verbatim}
Points = [[ 0 , 0 ], [ 0 , L], [L, 0 ], [ L , L], [ L , 2*L]]
Barres = [[ 0 , 1 ], [ 0 , 3 ], [1,3],[2,3],[1,4],[3,4]]
Conditions = [[ 0 , 0 , 0 ],[ 2 , 0 , 0 ]] 
Forces = [[4,P/2,0]]
 \end{verbatim}
\begin{center}
\begin{tikzpicture}[scale=15]
\draw[orange, double distance = 1pt] (0,0) - - (0,0.2);
\draw[orange, double distance = 1pt] (0,0) - - (0.2,0.2);
\draw[orange, double distance = 1pt] (0,0.2) - - (0.2,0.2);
\draw[orange, double distance = 1pt] (0.2,0) - - (0.2,0.2);
\draw[orange, double distance = 1pt] (0,0.2) - - (0.2,0.4);
\draw[orange, double distance = 1pt] (0.2,0.2) - - (0.2,0.4);
\path[fill=orange!40]  (0,0) circle (.075mm);
\path[fill=orange!40]  (0,0.2) circle (.075mm);
\path[fill=orange!40]  (0.2,0) circle (.075mm);
\path[fill=orange!40]  (0.2,0.2) circle (.075mm);
\path[fill=orange!40]  (0.2,0.4) circle (.075mm);
\draw[blue, double distance = 1pt] (0.0,0.0) - - (0.029142135623730938,0.21380000000000002);
\draw[blue, double distance = 1pt] (0.0,0.0) - - (0.23294213562373095,0.1988);
\draw[blue, double distance = 1pt] (0.029142135623730938,0.21380000000000002) - - (0.23294213562373095,0.1988);
\draw[blue, double distance = 1pt] (0.2,0.0) - - (0.23294213562373095,0.1988);
\draw[blue, double distance = 1pt] (0.029142135623730938,0.21380000000000002) - - (0.27208427124746193,0.4026);
\draw[blue, double distance = 1pt] (0.23294213562373095,0.1988) - - (0.27208427124746193,0.4026);
\path[fill=black]  (0.0,0.0) circle (.075mm) [fill=gray];
\path[fill=black]  (0.029142135623730938,0.21380000000000002) circle (.075mm) [fill=gray];
\path[fill=black]  (0.2,0.0) circle (.075mm) [fill=gray];
\path[fill=black]  (0.23294213562373095,0.1988) circle (.075mm) [fill=gray];
\path[fill=black]  (0.27208427124746193,0.4026) circle (.075mm) [fill=gray];
\draw[very thick,red,-latex] (0.27208427124746193,0.4026)--++(0.066,0.0);
\end{tikzpicture}
\end{center}
\subsubsection*{Exemple 10}
 \begin{verbatim}
Points = [[ 0 , 0 ], [ 0 , L], [L, 0 ], [ L , L], [ 0 , 2*L]]
Barres = [[ 0 , 1 ], [ 0 , 3 ], [1,3],[2,3],[1,4],[3,4]]
Conditions = [[ 0 , 0 , 0 ],[ 2 , 0 , 0 ]] 
Forces = [[4,P/2,0]]
 \end{verbatim}
\begin{center}
\begin{tikzpicture}[scale=15]
\draw[orange, double distance = 1pt] (0,0) - - (0,0.2);
\draw[orange, double distance = 1pt] (0,0) - - (0.2,0.2);
\draw[orange, double distance = 1pt] (0,0.2) - - (0.2,0.2);
\draw[orange, double distance = 1pt] (0.2,0) - - (0.2,0.2);
\draw[orange, double distance = 1pt] (0,0.2) - - (0,0.4);
\draw[orange, double distance = 1pt] (0.2,0.2) - - (0,0.4);
\path[fill=orange!40]  (0,0) circle (.075mm);
\path[fill=orange!40]  (0,0.2) circle (.075mm);
\path[fill=orange!40]  (0.2,0) circle (.075mm);
\path[fill=orange!40]  (0.2,0.2) circle (.075mm);
\path[fill=orange!40]  (0,0.4) circle (.075mm);
\draw[blue, double distance = 1pt] (0.0,0.0) - - (0.024142135623730937,0.21380000000000002);
\draw[blue, double distance = 1pt] (0.0,0.0) - - (0.23294213562373095,0.1988);
\draw[blue, double distance = 1pt] (0.024142135623730937,0.21380000000000002) - - (0.23294213562373095,0.1988);
\draw[blue, double distance = 1pt] (0.2,0.0) - - (0.23294213562373095,0.1988);
\draw[blue, double distance = 1pt] (0.024142135623730937,0.21380000000000002) - - (0.05828427124746189,0.42760000000000004);
\draw[blue, double distance = 1pt] (0.23294213562373095,0.1988) - - (0.05828427124746189,0.42760000000000004);
\path[fill=black]  (0.0,0.0) circle (.075mm) [fill=gray];
\path[fill=black]  (0.024142135623730937,0.21380000000000002) circle (.075mm) [fill=gray];
\path[fill=black]  (0.2,0.0) circle (.075mm) [fill=gray];
\path[fill=black]  (0.23294213562373095,0.1988) circle (.075mm) [fill=gray];
\path[fill=black]  (0.05828427124746189,0.42760000000000004) circle (.075mm) [fill=gray];
\draw[very thick,red,-latex] (0.05828427124746189,0.42760000000000004)--++(0.066,0.0);
\end{tikzpicture}
\end{center}

 
 
 
 \subsection*{Liverable 2}
 \begin{enumerate}
  \item Ecrire un programme python permettant de:
 \begin{enumerate}
 \item Saisir les données du problème.
 \item Résoudre l'équation d'équilibre.
 \item Afficher le résultat de calcul.
\item Représenter graphiquement le système au repos et le système déformé.
 
 \end{enumerate}

\item Appliquez le programme à votre cas d'étude du livrable 1 pour afficher le résultat de calcul obtenu.
\item  Comparez le avec les résultats du livrable 1 quelle conclusion pouvez-vous en faire ? 
 
 \item Résoudre un exemple de treillis formé de $n$ barres avec $n\geq 10$. Par exemple



  
\begin{center}
\begin{tikzpicture}[scale=1]
\appui{0}{0}{.3};
\appui{0}{-2}{.3};
\draw[double distance = 1pt] (0,0) - - (2,0);
\draw[double distance = 1pt] (2,0) - - (4,0);
\draw[double distance = 1pt] (4,0) - - (6,0);
\draw[double distance = 1pt] (6,0) - - (4,-2);
\draw[double distance = 1pt] (4,-2) - - (2,-2);
\draw[double distance = 1pt] (2,-2) - - (0,-2);
\draw[double distance = 1pt] (4,0) - - (2,-2);
\draw[double distance = 1pt] (2,0) - - (0,-2);
\draw[double distance = 1pt] (2,0) - - (2,-2);
\draw[double distance = 1pt] (4,0) - - (4,-2);
\path[fill=black]  (0,0) circle (.75mm) [fill=gray];
\path[fill=black]  (2,0) circle (.75mm) [fill=gray];
\path[fill=black]  (4,0) circle (.75mm) [fill=gray];
\path[fill=black]  (6,0) circle (.75mm) [fill=gray];
\path[fill=black]  (4,-2) circle (.75mm) [fill=gray];
\path[fill=black]  (2,-2) circle (.75mm) [fill=gray];
\path[fill=black]  (0,-2) circle (.75mm) [fill=gray];

\draw[red,double distance = 1pt] (0.0,0.0) - - (2.0300000000000002,-0.07242640687119303);
\draw[red,double distance = 1pt] (2.0300000000000002,-0.07242640687119303) - - (4.045,-0.1898528137423861);
\draw[red,double distance = 1pt] (4.045,-0.1898528137423861) - - (6.045,-0.24985281374238621);
\draw[red,double distance = 1pt] (6.045,-0.24985281374238621) - - (3.985,-2.189852813742386);
\draw[red,double distance = 1pt] (3.985,-2.189852813742386) - - (1.9849999999999999,-2.087426406871193);
\draw[red,double distance = 1pt] (1.9849999999999999,-2.087426406871193) - - (0.0,-2.0);
\draw[red,double distance = 1pt] (4.045,-0.1898528137423861) - - (1.9849999999999999,-2.087426406871193);
\draw[red,double distance = 1pt] (2.0300000000000002,-0.07242640687119303) - - (0.0,-2.0);
\draw[red,double distance = 1pt] (2.0300000000000002,-0.07242640687119303) - - (1.9849999999999999,-2.087426406871193);
\draw[red,double distance = 1pt] (4.045,-0.1898528137423861) - - (3.985,-2.189852813742386);
\path[fill=black]  (0.0,0.0) circle (.75mm) [fill=gray];
\path[fill=black]  (2.0300000000000002,-0.07242640687119303) circle (.75mm) [fill=gray];
\path[fill=black]  (4.045,-0.1898528137423861) circle (.75mm) [fill=gray];
\path[fill=black]  (6.045,-0.24985281374238621) circle (.75mm) [fill=gray];
\path[fill=black]  (3.985,-2.189852813742386) circle (.75mm) [fill=gray];
\path[fill=black]  (1.9849999999999999,-2.087426406871193) circle (.75mm) [fill=gray];
\path[fill=black]  (0.0,-2.0) circle (.75mm) [fill=gray];
(6.045,-0.24985281374238621)
\draw [blue,->,very thick] (6.045,-0.25)-- (6.045,-2.25);
\end{tikzpicture}
\end{center}
 
 \end{enumerate}
\subsection*{Livrable 3}
\begin{enumerate}
\item Écrire les équations  du système de barres articulées soumises à des efforts extérieurs et à une variation donnée de température. 
\item Améliorer le programme python de la séance précédente, pour qu'il puisse tenir  compte de la dilatation sous l'action d'une élévation de température ou de la contraction dû à une baisse de température. 
\item Appliquer numériquement à l'exemple de votre groupe.
\end{enumerate}
 


\begin{center}
\begin{tikzpicture}[scale=1]
% -- Soleil
  \shade [ball color=gray!10!yellow] (2,4) circle (0.5);
  \node (soleil) at (2,4) {\small Soleil};
  
\draw[double distance = 1pt] (0.0,0.0) - - (0.6666666666666666,2.0);
\draw[double distance = 1pt] (0.6666666666666666,2.0) - - (2.0,2.0);
\draw[double distance = 1pt] (2.0,2.0) - - (3.3333333333333335,2.0);
\draw[double distance = 1pt] (3.3333333333333335,2.0) - - (4.0,0.0);
\draw[double distance = 1pt] (0.0,0.0) - - (2.0,2.0);
\draw[double distance = 1pt] (2.0,2.0) - - (4.0,0.0);
\path[fill=black]  (0.0,0.0) circle (.75mm) [fill=gray];
\path[fill=black]  (0.6666666666666666,2.0) circle (.75mm) [fill=gray];
\path[fill=black]  (2.0,2.0) circle (.75mm) [fill=gray];
\path[fill=black]  (3.3333333333333335,2.0) circle (.75mm) [fill=gray];
\path[fill=black]  (4.0,0.0) circle (.75mm) [fill=gray];

\draw[red,double distance = 1pt] (0.0,0.0) - - (0.5999999999999999,2.1100632683380107);
\draw[red,double distance = 1pt] (0.5999999999999999,2.1100632683380107) - - (2.0,2.2121320343559643);
\draw[red,double distance = 1pt] (2.0,2.2121320343559643) - - (3.4000000000000004,2.1100632683380107);
\draw[red,double distance = 1pt] (3.4000000000000004,2.1100632683380107) - - (4.0,0.0);
\draw[red,double distance = 1pt] (0.0,0.0) - - (2.0,2.2121320343559643);
\draw[red,double distance = 1pt] (2.0,2.2121320343559643) - - (4.0,0.0);
\path[fill=black]  (0.0,0.0) circle (.75mm) [fill=gray];
\path[fill=black]  (0.5999999999999999,2.1100632683380107) circle (.75mm) [fill=gray];
\path[fill=black]  (2.0,2.2121320343559643) circle (.75mm) [fill=gray];
\path[fill=black]  (3.4000000000000004,2.1100632683380107) circle (.75mm) [fill=gray];
\path[fill=black]  (4.0,0.0) circle (.75mm) [fill=gray];
\end{tikzpicture}
\end{center}

\subsection*{Livrable 4} Étudier le problème  en dimension 3 (modélisation, équations d'équilibre, matrice élémentaire, matrice globale, résolution numérique, représentation graphique). Appliquer numériquement à votre exemple.
 
 \subsection*{Livrable 5} 
 Étudier en deux dimensions, le système de treillis des poutres basées sur modèle de Timoshenko. Ces poutres transmettent des efforts normaux ,comme les barres, et aussi des efforts tranchants et des moments de flexion. Elles peuvent être articulées ou encastrées entre elles…

Écrire un programme python, pour ce cas, analogue au précédent. Appliquer numériquement à votre exemple.

\end{document}
