\documentclass{beamer}
\usepackage[francais]{babel}
\usepackage[utf8]{inputenc} % Required for including letters with accents
\usepackage[T1]{fontenc} % Use 8-bit encoding that has 256 glyphs
\usepackage{pythontex}
\usepackage{amsthm}
\usepackage{amsmath}
\usepackage{amssymb}
\usepackage{mathrsfs}
\usepackage{graphicx}
\usepackage{geometry}
\usepackage{stmaryrd}
\usepackage{tikz}
\usetikzlibrary{patterns}
%\usetikzlibrary{intersections}
% Code syntax highlighter
\usepackage[cache=false]{minted}

\usepackage{stmaryrd}
%\usepackage{tikz}
%\usetikzlibrary{tikzmark}
\usepackage{empheq}
\usepackage{longtable}
\usepackage{booktabs} 
\usepackage{array}
\usepackage{pstricks}
\usepackage{pst-3dplot}
\usepackage{pst-tree}
\usepackage{pstricks-add}
\usepackage{upgreek}
%\usepackage{epstopdf}
\usepackage{eolgrab}
\usepackage{chngpage}
 \usepackage{calrsfs}
 % Appel du package pythontex 
\usepackage{pythontex}

\usetikzlibrary{decorations.pathmorphing}
\def \de {{\rm d}}
%\usepackage{color}
%\usepackage{xcolor}
%\usepackage{textcomp}
\newcommand{\mybox}[1]{\fbox{$\displaystyle#1$}}
\newcommand{\myredbox}[1]{\fcolorbox{red}{white}{$\displaystyle#1$}}
\newcommand{\mydoublebox}[1]{\fbox{\fbox{$\displaystyle#1$}}}
\newcommand{\myreddoublebox}[1]{\fcolorbox{red}{white}{\fcolorbox{red}{white}{$\displaystyle#1$}}}
\usetheme[options]{Boadilla}
\definecolor{purple2}{RGB}{153,0,153} % there's actually no standard purple
\definecolor{green2}{RGB}{0,153,0} % a darker green
\usepackage{xcolor}

\definecolor{LightGray}{gray}{0.9}
%\usepackage{listings}

  \title{Programmation C}
  \author{ \textsc{Ibrahim ALAME}}\institute{ESIEE}
\date{07/02/2023}
  \begin{document}

 \begin{frame}
 \begin{center}
 Chapitre 2
 \end{center}
  \titlepage
  \end{frame}

 \begin{frame}
  \frametitle{Fonctions}
  \begin{block}{Définition}
 Une fonction est une suite d'instructions, mais vue du programme principal main(), elle représente une seule action.
Deux types de fonction:
\begin{itemize}
\item Les fonctions prédéfinies ({\tt scanf, printf, strcpy}…)
\item les fonctions personnelles
\end{itemize}
\end{block}

Utilité des fonctions:
\begin{itemize}
\item Supprimer les répétitions de code
\item Structurer des blocs de code indépendants (maintenance)
\item Création de bibliothèques de fonctions
\end{itemize}

  \end{frame}
  
\begin{frame}[fragile]
\frametitle{Quelques exemples}
Une définition de fonction spécifie :
\begin{itemize}
\item Le type de la valeur renvoyée par la fonction
\item Le nom de la fonction
\item Les paramètres (argument) qui sont passés à la fonction
\item Les variables locales et externes utilisées par la fonction
\item D'autre fonctions appelées éventuellement par la fonction
\item Les instructions que doit exécuter la fonction
\end{itemize}

Syntaxe d'une fonction:

\begin{minted}[
%frame=lines,
framesep=2mm,
baselinestretch=1.2,
%bgcolor=LightGray,
fontsize=\footnotesize,
linenos
]{c}
<type_iden_sortie> iden_Fct( <type> ien_1, ... ,<type> iden_N){
        /* déclaration  des variables locales */
        <type>  iden_2,...,iden_Z;
        //
        // Traitement 
        //
        /* renvoi le résultat dans le paramètre de sortie*/
        return(valeur);
    }

\end{minted}

\end{frame}
  
  
 
%%%%%%%%%%%%%%%%%%%%%%%%%%%%%%%%%%%%%%%%%%%%%%%%%%%%%%%%%%%%%%%
\begin{frame}[fragile]
\frametitle{Quelques exemples}
Exemple d'une fonction:
\begin{minted}[
%frame=lines,
framesep=2mm,
baselinestretch=1.2,
%bgcolor=LightGray,
fontsize=\footnotesize,
linenos
]{c}
    Double carre (double C ) {      
        double resultat;   /* déclaration  d'une variable locale */
        resultat = C * C;
        /* renvoi le résultat dans le paramètre de sortie*/
        return (resultat);
    }
\end{minted}
    Autres caractéristiques des fonctions :
\begin{itemize}
\item Une fonction peut en appeler une autre. La fonction appelée doit être déclarée avant celle qui appelle.
\item On ne peut pas déclarer une fonction à l'intérieur d'une autre fonction.
\item Une fonction possède un seul point d'entrée, mais éventuellement plusieurs de sortie. 
\end{itemize}

\end{frame}
  
  
 
%%%%%%%%%%%%%%%%%%%%%%%%%%%%%%%%%%%%%%%%%%%%%%%%%%%%%%%%%%%%%%%

\begin{frame}[fragile]
\frametitle{Où déclarer  une fonction? : avant le main()}
\begin{minted}[
%frame=lines,
framesep=2mm,
baselinestretch=1.2,
%bgcolor=LightGray,
fontsize=\footnotesize,
linenos
]{c}
#include <stdio.h>

double carre (double C ) {      
    double resultat; // déclaration  d'une variable locale
    resultat = C * C;
    return( resultat ); // renvoi le résultat à la sortie
    }
    
int main(){
 	double A, B=2;
  	A = carre (B );
 	printf ("Le carré de %f est %f",B,A);
    return 0;
}
\end{minted}
\end{frame}
%%%%%%%%%%%%%%%%%%%%%%%%%%%%%%%%%%%%%%%%%%%%%%%%%%%%%%%%%%%%%%%%%

\begin{frame}[fragile]
\frametitle{Où déclarer  une fonction? : après le main()}
\begin{minted}[
%frame=lines,
framesep=2mm,
baselinestretch=1.2,
%bgcolor=LightGray,
fontsize=\footnotesize,
linenos
]{c}
#include <stdio.h>

double carre (double C ) ; // PROTOTYPE DE LA FONCTION

int main(){
 	double A, B=2;
  	A = carre (B );
 	printf ("Le carré de %f est %f",B,A);
    return 0;
}

double carre (double C ) {      
    double resultat; // déclaration  d'une  variable locale
    resultat = C * C;
    return( resultat ); // renvoi le résultat à la sortie
}

\end{minted}
\end{frame}
%%%%%%%%%%%%%%%%%%%%%%%%%%%%%%%%%%%%%%%%%%%%%%%%%%%%%%%%%%%%%%%%%



\begin{frame}[fragile]
\frametitle{Déclaration PROTOTYPE}
Le prototype de la fonction définie l'interface de la fonction. Lors de la compilation, le compilateur accepte l'appel à une fonction seulement s'il à rencontré au moins le prototype de la fonction. Dans la déclaration prototype il est possible d'omettre les noms des variables mais pas leur type.

\begin{minted}[
%frame=lines,
framesep=2mm,
baselinestretch=1.2,
%bgcolor=LightGray,
fontsize=\footnotesize,
linenos
]{c}
#include <stdio.h>
double carre (double) ; // PROTOTYPE SANS NOM DE VARIABLE
int main(){
 	double A, B=2;
  	A = carre (B );
 	printf ("Le carré de %f est %f",B,A);
    return 0;
}
double carre (double C ) {      
    double resultat; // déclaration  d'une  variable locale
    resultat = C * C;
    return( resultat ); // renvoi le résultat à la sortie
}

\end{minted}
\end{frame}
%%%%%%%%%%%%%%%%%%%%%%%%%%%%%%%%%%%%%%%%%%%%%%%%%%%%%%%%%%%%%%%%%


\begin{frame}[fragile]
\frametitle{Les fonctions et les échanges de variables }
Deux types:

\begin{itemize}
\item Les fonctions sans passage de paramètres et ne renvoyant rien au programme :    
		
\begin{minted}[
%frame=lines,
framesep=2mm,
baselinestretch=1.2,
%bgcolor=LightGray,
fontsize=\footnotesize,
linenos
]{c}
void UneFonction ( void ); 
\end{minted}
\item Les fonctions avec passage de paramètres et renvoyant un objet au programme :
		
\begin{minted}[
%frame=lines,
framesep=2mm,
baselinestretch=1.2,
%bgcolor=LightGray,
fontsize=\footnotesize,
linenos
]{c}
int UneFonction ( int A ): 
\end{minted} 

\end{itemize}


\end{frame}
%%%%%%%%%%%%%%%%%%%%%%%%%%%%%%%%%%%%%%%%%%%%%%%%%%%%%%%%%%%%%%%%%


\begin{frame}[fragile]
\frametitle{Les fonctions et les échanges de paramètres : Par des variables globales}

    Une variable globale est déclarée au début du programme.
    Cette variable sera visible dans toutes les fonctions

    Exemple
	
\begin{minted}[
%frame=lines,
framesep=2mm,
baselinestretch=1.2,
%bgcolor=LightGray,
fontsize=\footnotesize,
linenos
]{c}
#include <stdio.h>

int X; /* déclaration de X en variable globale */

void main() {
...

}
\end{minted}

\end{frame}
%%%%%%%%%%%%%%%%%%%%%%%%%%%%%%%%%%%%%%%%%%%%%%%%%%%%%%%%%%%%%%%%%



\begin{frame}[fragile]
\frametitle{Les fonctions et les échanges de paramètres : Par des variables locales}
\begin{itemize}
\item Une variable locale est déclarée au début d'une fonction.
\item Cette variable sera visible uniquement dans la fonction.
\item Une variable globale portant le même nom sera invisible dans la fonction.
\item Les variables locales ne sont pas initialisées par défaut et perdent leur valeur à chaque nouvel appel de la fonction
\end{itemize}

\begin{minted}[
%frame=lines,
framesep=2mm,
baselinestretch=1.2,
%bgcolor=LightGray,
fontsize=\footnotesize,
linenos
]{c}
#include <stdio.h>
int X,A;  /* déclaration de X et A en variable globale */
int MaFonction( int A) {  /* déclaration de A en variable locale */ 
    static int X;         /* déclaration de X en variable locale */
	. . .
    return (X);
}
void main(void) {
	. . .
}
\end{minted}

\end{frame}
%%%%%%%%%%%%%%%%%%%%%%%%%%%%%%%%%%%%%%%%%%%%%%%%%%%%%%%%%%%%%%%%%

\begin{frame}[fragile]
\frametitle{Échange entre les fonctions par variables globales}

\begin{minted}[
%frame=lines,
framesep=2mm,
baselinestretch=1.2,
%bgcolor=LightGray,
fontsize=\footnotesize,
linenos
]{c}
#include <stdio.h>                         

int A,B;
void cube(void);

int main(void) {
    A=2; // initialisation de A à 2
    int B; /* déclaration locale de B */
    printf("A=%d, B=%d\n",A,B);	/* affiche : A=4, B=0  */
    cube();
    printf("A=%d, B=%d\n",A,B);	/* affiche : A=4, B=0   */	
}

void cube() { 	
    B = A*A*A;
    return;
}
\end{minted}

\end{frame}
%%%%%%%%%%%%%%%%%%%%%%%%%%%%%%%%%%%%%%%%%%%%%%%%%%%%%%%%%%%%%%%%%

\begin{frame}[fragile]
\frametitle{Échange entre les fonctions par variables locales}

\begin{minted}[
%frame=lines,
framesep=2mm,
baselinestretch=1.2,
%bgcolor=LightGray,
fontsize=\footnotesize,
linenos
]{c}
#include <stdio.h> 

int cube(int);

int main(void) {
    int A=2; //déclaration locale et initialisation de A à 2
    int B; /* déclaration locale de B */
    printf("A=%d, B=%d\n",A,B);	/* affiche : A=4, B=??   */
    B = cube(A);
    printf("A=%d, B=%d\n",A,B);	/* affiche : A=4, B=8   */	
}

int cube(int X) { /* X est un paramètre de la fonction cube */
              /* X récupère la valeur de A du main */
	int resultat;		
    resultat = X*X*X;
    return (resultat);
}
\end{minted}

\end{frame}
%%%%%%%%%%%%%%%%%%%%%%%%%%%%%%%%%%%%%%%%%%%%%%%%%%%%%%%%%%%%%%%%%

\begin{frame}[fragile]
\frametitle{Échange entre les fonctions par variables locales}

\begin{minted}[
%frame=lines,
framesep=2mm,
baselinestretch=1.2,
%bgcolor=LightGray,
fontsize=\footnotesize,
linenos
]{c}
#include <stdio.h> 
void cube(int);
int main(void) {
    int A=2; //déclaration locale et initialisation de A à 2
    int B; /* déclaration locale de B */
    printf("A=%d, B=%d\n",A,B);	/* affiche : A=4, B=??   */
    cube(A);
    printf("A=%d, B=%d\n",A,B);	/* affiche : A=4, B=??   */	
}

void cube(int A) { /* A est un paramètre de la fonction cube */
			/* A récupère la valeur de A du main */
	int B;		
	B = A*A*A;
	printf("A=%d, B=%d\n",A,B);	/* affiche : A=4, B=8   */	
	return ;
}
\end{minted}

\end{frame}
%%%%%%%%%%%%%%%%%%%%%%%%%%%%%%%%%%%%%%%%%%%%%%%%%%%%%%%%%%%%%%%%%

\begin{frame}[fragile]
\frametitle{Échange entre les fonctions par variables locales}

\begin{minted}[
%frame=lines,
framesep=2mm,
baselinestretch=1.2,
%bgcolor=LightGray,
fontsize=\footnotesize,
linenos
]{c}
#include <stdio.h>                         
void cube(int, int*);
int main(void) {
    int A=2; /* déclaration locale de A et B */
    int B;
    printf("A=%d, B=%d\n",A,B);	/* affiche : A=4, B=??   */
    cube( A , &B);//On donne à la fonction cube le contenu de A 
                  // et l'adresse de de la variable B
    printf("A=%d, B=%d\n",A,B);	/* affiche : A=4, B=??   */	
}

void cube(int A, int *B) {/* A est un paramètre de la fonction cube */
			/* B récupère un pointeur sur un entier */
    printf("> A=%d, B=%d\n",A,*B);	/* affiche : A=4, B=??   */
    *B = A*A*A;
    printf("> A=%d, B=%d\n",A,*B);	/* affiche : A=4, B=8   */
    return;
}

\end{minted}

\end{frame}
%%%%%%%%%%%%%%%%%%%%%%%%%%%%%%%%%%%%%%%%%%%%%%%%%%%%%%%%%%%%%%%%%
\begin{frame}
\frametitle{Échange entre les fonctions par variables locales\\Introduction: }
\begin{itemize}
\item Une fonction récursive est une fonction qui s'appelle elle-même. 
\begin{itemize}
\item Directement (si la fonction $P$ appelle directement $P$ , on dit que la récursivité est directe). 
\item  Indirectement à travers une ou plusieurs fonctions relais (si $P$ appelle une fonction $P_1$ , qui appelle une fonction $P_2$ , ... , qui appelle une fonction $P_n$ et qui enfin appelle $P$ , on dit qu'il s'agit d'une récursivité indirecte).
\end{itemize}
\end{itemize}
\end{frame}
  \end{document}
   

























