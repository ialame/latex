\documentclass[a4paper]{article} 
\usepackage{amsthm}
\usepackage{amsmath}
\usepackage{amssymb}
\usepackage{mathrsfs}
\usepackage{graphicx}
\usepackage{geometry}
\usepackage{stmaryrd}
\usepackage{tikz}

\def \de {{\rm d}}

\usepackage{listings}

\definecolor{darkWhite}{rgb}{0.94,0.94,0.94}
\usepackage[cache=false]{minted} 
\usepackage{xcolor}
%\setbeamercolor{background canvas}{bg=lightgray}
\definecolor{LightGray}{gray}{0.9}
\definecolor{monOrange}{rgb}{0.97,0.35,0.04}


\title{TP1 corrigé: Les listes chainées}
\author{Ibrahim ALAME}
\date{20/12/2023}
\begin{document}
\maketitle


\begin{minted}[
mathescape,
framesep=2mm,
baselinestretch=1.2,
fontsize=\footnotesize,
bgcolor=LightGray,
%linenos
]{c}
#include <stdio.h>
#include <stdlib.h>
typedef struct Element Element;
struct Element{
    int nombre;
    Element *suivant;
};
typedef struct Liste Liste;
struct Liste{
    Element *premier;
};
Liste *initialisation(){
    Liste* liste = malloc(sizeof(*liste));
    Element *element = malloc(sizeof(*element));
    if(liste == NULL || element ==NULL) exit(EXIT_FAILURE);
    element->nombre =0;
    element->suivant = NULL;
    liste->premier = element;
    return liste;
}
void insertion(Liste *liste,int nvNombre){
    /* Création du nouvel élément */
    Element *nouveau = malloc(sizeof(Element));
    if(liste == NULL || nouveau == NULL) exit(EXIT_FAILURE);
    nouveau->nombre = nvNombre;
    /* Insertion de l'élément au début de la liste */
    nouveau->suivant = liste->premier;
    liste->premier = nouveau;
}
void suppression(Liste *liste){
    if(liste == NULL) exit(EXIT_FAILURE);
    if(liste->premier != NULL){
        Element *aSupprimer = liste->premier;
        liste->premier = liste->premier->suivant;
        free(aSupprimer);
    }
}
void afficherListe(Liste *liste){
    if(liste == NULL) exit(EXIT_FAILURE);
    Element *actuel = liste->premier;
    while (actuel != NULL){
        printf("%d -> ",actuel->nombre);
        actuel = actuel->suivant;
    }
    printf("NULL\n");
}
int main(){
    Liste* L = initialisation();
    insertion(L,4);
    insertion(L,8);
    insertion(L,15);
    afficherListe(L);
    suppression(L);
    afficherListe(L);
    
    return 0;
}
\end{minted}



\end{document}
