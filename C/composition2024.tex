\documentclass{article}
\usepackage[francais]{babel}
\usepackage[utf8]{inputenc} % Required for including letters with accents
\usepackage[T1]{fontenc} % Use 8-bit encoding that has 256 glyphs
\usepackage{pythontex}
\usepackage{amsthm}
\usepackage{amsmath}
\usepackage{amssymb}
\usepackage{mathrsfs}
\usepackage{graphicx}
\usepackage{geometry}
\usepackage{stmaryrd}
\usepackage{tikz}
\usetikzlibrary{patterns}
%\usetikzlibrary{intersections}
\usetikzlibrary{calc} 
%\usepackage{tkz-tab}
\usepackage{stmaryrd}
%\usepackage{tikz}
%\usetikzlibrary{tikzmark}
\usepackage{empheq}
\usepackage{longtable}
\usepackage{booktabs} 
\usepackage{array}
\usepackage{pstricks}
\usepackage{pst-3dplot}
\usepackage{pst-tree}
\usepackage{pstricks-add}
\usepackage{upgreek}
%\usepackage{epstopdf}
\usepackage{eolgrab}
\usepackage{chngpage}
 \usepackage{calrsfs}
 % Appel du package pythontex 
\usepackage{pythontex}

\usetikzlibrary{decorations.pathmorphing}
\def \de {{\rm d}}
\def \ch {{\rm ch}}
\def \sh {{\rm sh}}
\def \th {{\rm th}}
\def \sinc {{\rm sinc}}
\def \cotan {{\rm cotan}}

\usepackage{color}
%\usepackage{xcolor}
%\usepackage{textcomp}
\newcommand\hr{\par\vspace{-.5\ht\strutbox}\noindent\hrulefill\par}
\newcommand{\mybox}[1]{\fbox{$\displaystyle#1$}}
\newcommand{\myredbox}[1]{\fcolorbox{red}{white}{$\displaystyle#1$}}
\newcommand{\mydoublebox}[1]{\fbox{\fbox{$\displaystyle#1$}}}
\newcommand{\myreddoublebox}[1]{\fcolorbox{red}{white}{\fcolorbox{red}{white}{$\displaystyle#1$}}}

\definecolor{purple2}{RGB}{153,0,153} % there's actually no standard purple
\definecolor{green2}{RGB}{0,153,0} % a darker green
\usepackage{xcolor}
%\setbeamercolor{background canvas}{bg=lightgray}
\usepackage{fancyvrb}
\newcolumntype{M}[1]{>{\centering}m{#1}}
\usepackage{listings}
\definecolor{purple2}{RGB}{153,0,153} % there’s actually no standard purple
\definecolor{green2}{RGB}{0,153,0} % a

\usepackage[cache=false]{minted}
\definecolor{LightGray}{gray}{0.9}
\definecolor{monOrange}{rgb}{0.97,0.35,0.04}

\usepackage{geometry}
 \geometry{
 a4paper,
 total={210mm,297mm},
 left=25mm,
 right=25mm,
 top=20mm,
 bottom=20mm,
 }
\usepackage{wrapfig}
\title{}%Concours d'entrée à l'ESTP}
%\author{Ibrahim ALAME}
\date{}
  \begin{document}

%\maketitle

\begin{table}[h]
    \centering
    \begin{tabular}{p{3cm}p{13cm}}
    \begin{tabular}{c}
    \\
        \includegraphics[width=95pt]{ESIEE-Paris.png} 
     \end{tabular}
        & 
        \begin{tabular}{c}
        \\
        \large \bf   E3FE-3E-LE1 : Composition de Programmation C\\ \\
        \large   Ibrahim ALAME  \\ \\
         \large    26 Janvier 2024  \\ \\
        
        \end{tabular}
    \end{tabular}
    \label{tab:paragraphes}
\end{table}

\rule{15.5cm}{0.1pt}

Documents et calculatrice ne sont pas autorisés. Ce sujet est à compléter et à remettre avec la copie. 
\section*{Problème 1: QCM}(à rendre avec la copie à la fin du contrôle)\\
%\begin{center}

  \begin{tabular}{|M{4.5cm}|c|c|}
      \hline
      & Descriptifs  & Réponses \\ \hline
      \begin{verbatim}
int p=0,q=0;
while(p++<10)q++;
printf("%d\n",q);
\end{verbatim} & Quelle est la valeur de {\tt q} à la fin de la boucle {\tt while}?
% rep: 10
 &
    \begin{tabular}{c}
     10\\9 \\ 11\\0
\end{tabular}      \\ \hline
\begin{verbatim}
int k=10,i;
for(i=0;i<k;++i)
printf("%d\n",i++);
\end{verbatim} &Combien de fois l'instruction {\tt printf} s'exécutera ?
% rep: 5
 &
    \begin{tabular}{c}
     4 fois\\5 fois\\ 6 fois\\11 fois
\end{tabular}      \\ \hline
      \begin{verbatim}
int s=0,i,j;
for(i=0;i<3;i++)
for(j=0;j<3;j++)
s+=i*j;
printf("%d\n",s);
\end{verbatim} & Quelle est la valeur affichée par {\tt printf} ?
% rep: 9
 &
\begin{tabular}{c}
     0\\3 \\ 9\\12
\end{tabular}      \\ \hline
      \begin{verbatim}
int* m = "esiee";
int str2int(char* t){
    int s=0;
    while(*t!='\0'){
        s+=*t-'a';
        t++;}
    return s;}
printf("%d",str2int(m));
\end{verbatim} & Quelle est la valeur  affichée par {\tt printf} ?
% rep: 38
 &
    \begin{tabular}{c}
     38\\83 \\68\\86 
\end{tabular}      \\ \hline
      \begin{verbatim}
char t[] = "abcd";
int n;
n=sizeof(t)/sizeof(char);
printf("%d\n",n);
\end{verbatim} & Quelle est la valeur de {\tt n}  affichée par {\tt printf}  ?
% rep: 5
 &
    \begin{tabular}{c}
     4\\5 \\ 8\\10
\end{tabular}      \\ \hline
  \end{tabular}
%\end{center}

\section*{Problème 2 : Récursivité}
Soit la suite $(u_n)$ défini par $u_0=2$ et $\forall n\in \mathbb{N},\quad u_{2n} = u_n  \mbox{ et }u_{2n+1}=-u_n$. 
 Écrire une fonction récursive qui retourne la valeur de la suite $u_n$ pour $n$ donné. Calculer à la main les 10
premières valeurs, puis donner sans calcul les valeurs de $u_{32}$, $u_{256}$, $u_{513}$.

%#include <stdio.h>
%
%int u(int n){
%    if(n==0)return 2;
%    if(n%2==0)
%        return u(n/2);
%    else
%        return -u(n/2);
%}
%
%int main()
%{
%for(int i=0;i<11;i++)
%printf("%d\n",u(i));
%
%    return 0;
%}


\section*{Problème 3 : Pointeur}
Soit $t$ un tableau {\tt char t[] = "bonjour";} et $p$ un pointeur sur $t$,  {\tt char* p=t;}
Quelles valeurs ou adresses fournissent ces expressions:
\begin{enumerate}
\item {\tt *p+3}
% rep: 'e'
\item  {\tt *(p+3)}
% rep: 'j'
\item {\tt p[3]-3}
% rep: 'g'
\item  {\tt *(p++)-'a'}
% rep: 1
\item  {\tt *(++p)-'n'}
% rep: 1 ou 0 ça depond si on tien compte de la ligne prrécédente

\end{enumerate}
%
%char t[] = "bonjour";
%char* p =t;
%printf("*p+3=%c\n",*p+3);
%printf("*(p+3)=%c\n",*(p+3));
%printf("p[3]-3=%c\n",p[3]-3);
%printf("*(p++)-'a'=%d\n",*(p++)-'a');
%printf("*(++p)-'n'=%d\n",*(++p)-'n');


\section*{Problème 4 : Erreurs}
Corriger exactement quatre erreurs (compilation, fonctionnelle), puis commenter
brièvement les instructions du programme suivant:
\begin{minted}[
mathescape,
framesep=2mm,
baselinestretch=1.2,
%fontsize=\footnotesize,
bgcolor=LightGray,
%linenos
]{C}
#include <stdoi.h>
#include <stdlib.h>

int char2int(char c){
    if(c==' ')return 0;
    return c-'a'+1;
}
char int2char(char n){
    if(n==0) return ' ';
    return 'a'+n-1;
}
int str2int(char* t){
    int s=0,p=1;
    while (*t){
        s+= char2int(*t)*p;
        p*=100;
        t++;
    }
    return s;
}
char* int2str(int N){
    int t=1;
    char* T =(char) malloc(sizeof(char));
    while(N!=0){
        T[t-1]=int2char(N%100);
        t++;
        T =(char*) realloc(T,t*sizeof(char));
        N/=100;
    }
    T[t-1]='\0';
    return T;
}
int main() {
    int N=str2int("salut");
    printf("%d \n",N);
    char* s = int2str(N);
    printf("%c \n",s);
    return 0;
}

\end{minted}
%1. #include <stdoi.h>
%2. char* T =(char) malloc(sizeof(char));
%3. printf("%c \n",s);
%4. free(s);
A la fin de l'exécution, qu'affiche-t-on sur la sortie standard ?
%  2021120119 
%  salut 
\end{document}
