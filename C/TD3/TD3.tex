\documentclass[a4paper]{article} 
\usepackage[francais]{babel}
\usepackage[utf8]{inputenc} % Required for including letters with accents
\usepackage[T1]{fontenc} % Use 8-bit encoding that has 256 glyphs

\usepackage{amsthm}
\usepackage{amsmath}
\usepackage{amssymb}
\usepackage{mathrsfs}
\usepackage{graphicx}
\usepackage{geometry}
\usepackage{stmaryrd}
\usepackage{tikz}

\def \de {{\rm d}}



\title{Programmation C : TD 2}
\author{Ibrahim ALAME}
\date{18/10/2023}
\begin{document}
\maketitle

\begin{enumerate}
\item Écrire, de deux façons différentes, une fonction qui ne renvoie aucune valeur et qui détermine
 la valeur maximale et la valeur minimale d'un tableau d'entiers (à un indice) et taille quelconque. 
 Il faudra donc prévoir 4 arguments :
 \begin{itemize}
 \item En utilisant uniquement le {\em formalisme tableau}.
 \item En utilisant  le {\em formalisme pointeur}, chaque fois que cela est possible.
 \end{itemize}
Écrire un petit programme d'essai.

\item Écrire un programme C permettant à un enseignant de saisir le nombre d'étudiant d'une classe dont il est en charge et allouer dynamiquement l'espace juste nécessaire pour contenir leurs notes. Donner la main à l'enseignant pour saisir ses notes, puis calculer et afficher la moyenne de la classe.
 
 \item  Écrire un programme C permettant l'allocation dynamique de la mémoire requise pour enregistrer $N$ chaînes de caractères de $M$ caractère chacune. Tester votre programme en saisissant 3 mots.
\item Écrire un programme C permettant de saisir une chaine de caractère de taille quelconque.
 \item  Écrire un programme C permettant l'allocation dynamique de la mémoire requise pour enregistrer $N$ chaînes de caractères. Tester votre programme en saisissant 3 mots.
 \item  Écrire un programme C permettant l'allocation dynamique de la mémoire requise pour enregistrer plusieurs chaînes de caractères.  Tester votre programme en saisissant 3 mots.
\end{enumerate}
%%%%%%%%%%%%%%%%%%%%%%%%%%%%%%%%%%%%%%%%%%%%%%%%%%%%%%%%%%%%%%%%%%%%%%%%%%%%%%%


\end{document}