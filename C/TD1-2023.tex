\documentclass[a4paper]{article} 
\usepackage[francais]{babel}
\usepackage[utf8]{inputenc} % Required for including letters with accents
\usepackage[T1]{fontenc} % Use 8-bit encoding that has 256 glyphs

\usepackage{amsthm}
\usepackage{amsmath}
\usepackage{amssymb}
\usepackage{mathrsfs}
\usepackage{graphicx}
\usepackage{geometry}
\usepackage{stmaryrd}
\usepackage{tikz}

\def \de {{\rm d}}



\title{TD 1  Programmation C 3E-LE1}
\author{Ibrahim ALAME}
\date{09/10/2023}
\begin{document}
\maketitle
\section{Les variables simples, les E/S}
\begin{enumerate}
\item Écrire un programme C qui lit une température exprimée en degrés Farenheit et la convertit en degrés Celsius avec une précision d'un dixième de degré. La formule de conversion est :
\[ C=\frac{5(F-32)}9\]
Exemple :
\begin{verbatim}
> donner le degré en Farenheit : 45
 il correspond à 7.2 degré Celsius
\end{verbatim}
%%%%%%%%%%%%%%%%%%%%%%%%%%%%%%%%%%%%%%%%%%%%%%%%%%%%%%%%%%%%%%%%%%%%%%%%%%%%%%%%%
\item (modulo \%, division entière /)
On indique a l'ordinateur une somme entière comprise entre 0 et 999 euros a remettre. Afficher le nombre
\begin{itemize}
\item de billets de 100 euros,
\item de billets de 20 euros,
\item de pièces de 5 euros,
\item de pièces de 1 euro,
à remettre. Écrire le programme en C.
\end{itemize}
Exemple :
\begin{verbatim}
> combien voulez-vous remettre ? : 423
 vous devez me donner:
 4 billets de 100 euros = 400 euros
 1 billet de 20 euros = 20 euros
 0 billet de 5 euros = 0 euros
 3 pièces de 1 euro = 3 euros
Total = 423 euros
\end{verbatim}
Généraliser.

%%%%%%%%%%%%%%%%%%%%%%%%%%%%%%%%%%%%%%%%%%%%%%%%%%%%%%%%%%%%%%%%%%%%%%%%%%%%%%%%%%
\item Le poids idéal d'une personne est donné par la formule de Lorentz : il est fonction de la taille, exprimée en cm

\begin{itemize}
\item pour un homme, $65 + \frac 34$ de la taille au delà de $150$ cm.
\item pour une femme, $90\%$ de celle d'un homme.
\end{itemize}
Demander la taille en cm puis afficher le poids idéal (à un centième de kilo près) d'un homme et d'une lemme de cette taille.

Exemple :
\begin{verbatim}
> quelle est la taille en cm ?: 174
 le poids idéal selon Lorentz est:
pour un homme : 83
pour une femme : 74.7
\end{verbatim}
%%%%%%%%%%%%%%%%%%%%%%%%%%%%%%%%%%%%%%%%%%%%%%%%%%%%%%%%%%%%%%%%%%%%%%%%%%%%%%%%%%%%%%%%%%
%\item 
%On indique les relevés de 0 a 99999 d'un compteur kilométrique de voiture lors de deux pleins consécutifs ainsi que nombre de litres d'essence nécessaires pour faire ce plein. Afficher alors la consommation aux 100 kms.
%
%Exemple :
%\begin{verbatim}
%> kilométrage avant le plein : 53000
% kilométrage après : 53250
% nombre de litres d'essence : 15
% la consommation de votre véhicule est de 6 litres aux 100 kms
%\end{verbatim}
%
\end{enumerate}
%%%%%%%%%%%%%%%%%%%%%%%%%%%%%%%%%%%%%%%%%%%%%%%%%%%%%%%%%%%%%%%%%%%%%%%%

%%%%%%%%%%%%%%%%%%%%%%%%%%%%%%%%%%%%%%%%%%%%%%%%%%%%%%%%%%%%%%%%%%%%%%%%%%
\section{les structures conditionnelles}
\begin{enumerate}
%\item Afficher l'un des 3 messages suivants en fonction de deux valeurs entières saisies par l'utilisateur :
%\begin{enumerate}
%\item La première valeur est plus grande que la seconde.
%\item Les 2 valeurs sont identiques.
%\item la deuxième valeur est plus grande que la première.
%\end{enumerate}
%Exemple :
%\begin{verbatim}
%> donner 2 valeurs entières SVP: 45 65
%la deuxième valeur est plus grande que la première
%\end{verbatim}
%

\item (Affichage du médium de 3 nombres)

Afficher le nombre ni plus grand, ni plus petit de 3 nombres saisis à l'écran après les avoir classés dans l'ordre croissant.
Exemples
\begin{verbatim}
>  5 8 3
   Le médium de 3 5 8 est 5
>  5 5 3
   Le médium de 3 5 5 est 5
\end{verbatim}
Généraliser

\item Afficher la mention qui corresponde à la note saisie :
\begin{itemize}
\item  si la note est inférieure à 10, alors mention "ajourné"
\item  si la note est supérieure à 10 et strictement inférieure à 12, alors mention "passable"
\item  si la note est supérieure à 12 et strictement intérieure à 14, alors mention "assez bien"
\item  si la note est supérieure à 14 et strictement inférieure à 16, alors mention "bien"
\item  si la note est supérieure à 16 et strictement inférieure à 18, alors mention "très bien"
\item  si la note est supérieure à 18 , alors mention "honorable avec félicitation du jury"
\end{itemize}



\item A partir de la date sous forme de "jour" et "mois" saisis par l'utilisateur, afficher son signe zodiacal correspond. On suppose que la date entrée est correcte.

\begin{itemize}
\item  du 21-3 au 20-4 $\Longrightarrow$ BÉLIER
\item  du 21-4 au 20-5  $\Longrightarrow$ TAUREAU
\item  du 21-5 au 21-6  $\Longrightarrow$ GÉMEAUX
\item  du 22-6 au 22-7  $\Longrightarrow$ CANCER
\item  du 23-7 au 22-8  $\Longrightarrow$ LION
\item  du 23-8 au 22-9  $\Longrightarrow$ VIERGE
\item  du 23-9 au 22-10  $\Longrightarrow$ BALANCE
\item  du 22-11 au 20-12  $\Longrightarrow$ SCORPION
\item  du 23-10 au 21-11  $\Longrightarrow$ SAGITTAIRE
\item  du 21-12 au 20-1  $\Longrightarrow$ CAPRICORNE
\item  du 21-1 au 19-2  $\Longrightarrow$ VERSEAU
\item  du 20-2 au 20-3  $\Longrightarrow$ POISSONS
\end{itemize}


\item (La calculatrice de poche)

Lire une expression simple sous la forme

{\tt Opérandel opérateur Opérande2}

ou {\tt Opérandel} et {\tt Opérande2} sont des nombres réels, {\tt opérateur} est un caractère.
En fonction du signe de l'opérateur $(+,-,*,/,\%)$, afficher le résultat de l'expression où \% est la fonction modulo.\\
Tests d'erreur :

\begin{itemize}
\item  afficher "erreur de signe" si aucun des 5 signes ci-dessus n'est lu.
\item Afficher "division par 0 impossible" si le cas se présente.
\end{itemize}
\end{enumerate}
\end{document}

