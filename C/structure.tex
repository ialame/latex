%%%%%%%%%%%%%%%%%%%%%%%%%%%%%%%%%%%%%%%%%%%%%%%%%%%%%%%%%%%%%%%
%
% Welcome to Overleaf --- just edit your LaTeX on the left,
% and we'll compile it for you on the right. If you open the
% 'Share' menu, you can invite other users to edit at the same
% time. See www.overleaf.com/learn for more info. Enjoy!
%
%%%%%%%%%%%%%%%%%%%%%%%%%%%%%%%%%%%%%%%%%%%%%%%%%%%%%%%%%%%%%%%
\documentclass{beamer}
 \usetheme[options]{Boadilla}
 % paquets pour le français
 \usepackage{tikz}
 \usepackage{multicol}
 \usepackage[T1]{fontenc}
 \usepackage[utf8]{inputenc}
\usepackage{minted}
\definecolor{LightGray}{gray}{0.9}
  \title{Programmation C: Structure}
  \author{ \textsc{Ibrahim ALAME}}\institute{ESIEE}
\date{29/11/2023}
\begin{document}
\maketitle
  
  %%%%%%%%%%%%%%%%%%%%%%%%%%%%%%%%%%%%%%%%%%%%%%%%%%%%%%%%%%%%%%%%%

\begin{frame}[fragile]
\frametitle{Les structures}
\begin{block}{Qu'est-ce qu'une structure ?}
Une structure est un objet composé de plusieurs champs de types différents, qui sert à représenter un objet. Par exemple un client peut être représenté par son nom, son prénom, son année de naissance, son adresse.
\end{block}
\begin{minted}[
%frame=lines,
framesep=2mm,
baselinestretch=1.2,
bgcolor=LightGray,
fontsize=\footnotesize,
% linenos
]{c}  
#include <stdio.h>
struct client{// client est le nom de la structure
    	  char nom[25];
    	  char prenom[20];
    	  int  anneeNaissance;
    	  char adresse[100];
        };
int main(){ 
    //Déclaration de variables :
	struct client Un_Client;// déclaration de la variable Un_Client
	struct client NouveauClient;// déclaration de la variable NouveauClient
    // ....
    return 0; 
}
\end{minted}
\end{frame}
  %%%%%%%%%%%%%%%%%%%%%%%%%%%%%%%%%%%%%%%%%%%%%%%%%%%%%%%%%%%%%%%%%
  
  
\begin{frame}[fragile]
\frametitle{Définition du type et déclaration de variables?}

\begin{minted}[
%frame=lines,
framesep=2mm,
baselinestretch=1.2,
bgcolor=LightGray,
fontsize=\footnotesize,
% linenos
]{c}  
#include <stdio.h>
typedef struct client CLIENT;//création d'un alias CLIENT sur struct client
struct client{// client est le nom de la structure
    	  char nom[25];
    	  char prenom[20];
    	  int  anneeNaissance;
    	  char adresse[100];
        };
int main(){ 
    // Déclaration de variables simplifié :
    CLIENT Un_Client;// déclaration de la variable Un_Client
    CLIENT NouveauClient;// déclaration de la variable NouveauClient

    // ....
    return 0; 
}
\end{minted}
\end{frame}
  


 
%%%%%%%%%%%%%%%%%%%%%%%%%%%%%%%%%%%%%%%%%%%%%%%%%%%%%%%%%%%%%%%
\begin{frame}[fragile]
%\frametitle{Définition du type et déclaration de variables?}
\begin{minted}[
%frame=lines,
framesep=2mm,
baselinestretch=1.2,
bgcolor=LightGray,
fontsize=\footnotesize,
% linenos
]{c}  
#include <stdio.h>
#include <string.h>
typedef struct {
    char nom[10];
    int maths;
    int info;
    float moyenne;
} eleve;
int main(){
    eleve e1, e2;
    strcpy(e1.nom,"Pierre");
    e1.maths =15;
    e1.info =16;
    e1.moyenne =(e1.maths+e1.info)/2.;
    e2=e1;
    strcpy(e2.nom,"Jean");
    printf("Moyenne de %s est : %.2f/20\n",e1.nom,e1.moyenne);
    printf("Moyenne de %s est : %.2f/20\n",e2.nom,e2.moyenne);
    return 0;
}
\end{minted}

\end{frame}
  


 
%%%%%%%%%%%%%%%%%%%%%%%%%%%%%%%%%%%%%%%%%%%%%%%%%%%%%%%%%%%%%%%
\begin{frame}[fragile]
\frametitle{}

\begin{verbatim}
Moyenne de Pierre est : 15.50/20
Moyenne de Jean est : 15.50/20
\end{verbatim}

\end{frame}
 
%%%%%%%%%%%%%%%%%%%%%%%%%%%%%%%%%%%%%%%%%%%%%%%%%%%%%%%%%%%%%%%

\begin{frame}[fragile]
\frametitle{Les pointeurs de structures:}
\begin{minted}[
%frame=lines,
framesep=2mm,
baselinestretch=1.2,
bgcolor=LightGray,
fontsize=\footnotesize,
% linenos
]{c}
#include <stdio.h>
#include <stdlib.h>
typedef struct {
    int maths;
    int info;
} notes;
int main(){
    notes *p = (notes *) malloc(sizeof(notes));
    (*p).maths =15; (*p).info =16;// p->maths=15; p->info=16;
    //   (*structure).champs  <====>  structure->champs
    printf("Maths %d/20 ;  Info %d/20\n",(*p).maths,(*p).info);
    printf("Maths %d/20 ;  Info %d/20\n",p->maths,p->info);
    return 0;
}
\end{minted}
\begin{verbatim}
Maths 15/20 ;  Info 16/20
Maths 15/20 ;  Info 16/20
\end{verbatim}
\end{frame}
%%%%%%%%%%%%%%%%%%%%%%%%%%%%%%%%%%%%%%%%%%%%%%%%%%%%%%%%%%%%%%%%%  
  
 
%%%%%%%%%%%%%%%%%%%%%%%%%%%%%%%%%%%%%%%%%%%%%%%%%%%%%%%%%%%%%%%

\begin{frame}[fragile]
%\frametitle{Pointeurs: L'arithmétique des pointeurs.}
\begin{minted}[
%frame=lines,
framesep=2mm,
baselinestretch=1.2,
bgcolor=LightGray,
fontsize=\footnotesize,
% linenos
]{c}
#include <stdio.h>
#include <stdlib.h>
struct date{
    int jour;
    int mois;
    int annee;
} *p;
int main(){
    p = (struct date *) malloc (sizeof(struct date));
    if(p==NULL){
        printf("Erreur d'allocation mémoire !!!\n");
        exit(-1);
    }
    printf("Donner le jour : "); scanf("%d",&p->jour);
    printf("Donner le mois : "); scanf("%d",&p->mois);
    printf("Donner l'annee : "); scanf("%d",&p->annee);
    
    printf("La date est : %d/%d/%d\n",p->jour,p->mois,p->annee);
    return 0;
}
\end{minted}
\end{frame}
%%%%%%%%%%%%%%%%%%%%%%%%%%%%%%%%%%%%%%%%%%%%%%%%%%%%%%%%%%%%%%%%%

\begin{frame}[fragile]
%\frametitle{Pointeurs: L'arithmétique des pointeurs.}
\begin{verbatim}
Donner le jour : 12
Donner le mois : 7
Donner l'annee : 1963

La date est : 12/7/1963

\end{verbatim}
\end{frame}

%%%%%%%%%%%%%%%%%%%%%%%%%%%%%%%%%%%%%%%%%%%%%%%%%%%%%%%%%%%%%%%%%

\begin{frame}[fragile]
%\frametitle{Exemple}
\begin{minted}[
%frame=lines,
framesep=2mm,
baselinestretch=1.2,
bgcolor=LightGray,
fontsize=\tiny,%footnotesize,
% linenos
]{c}
#include <stdio.h>
#include <stdlib.h>
typedef struct {
    char nom[20];
    char tel[10];
} data;
int main(){
    data* repertoire[500], individu;
    int i=-1,j;
    do{
        printf("\nNom : "); scanf("%s",individu.nom);
        if(individu.nom[0]!='#'){
            printf("\nTéléphone : "); scanf("%s",individu.tel);
            i++;
            repertoire[i] = (data*) malloc (30);
            *(repertoire[i])=individu;
        }else
            break;
    } while(i<500);
    for(int j=0;j<=i;j++) // affichage
        printf("%s : %s\n", repertoire[j]->nom,repertoire[j]->tel);
}
\end{minted}
{\tiny
\begin{verbatim}
Nom : Pierre
Téléphone : 1234567890
Nom : Jean
Téléphone : 2345678901
Nom : #

Pierre : 1234567890
Jean : 2345678901
\end{verbatim}
}
\end{frame}

%%%%%%%%%%%%%%%%%%%%%%%%%%%%%%%%%%%%%%%%%%%%%%%%%%%%%%%%%%%%%%%%%

\end{document}
